\chapter{INTRODUÇÃO}

Em muitas situações de avaliação educacional, existe uma variável subjacente de interesse. Esta variável é muitas vezes algo que não é mensurável, como inteligência, \cite{baker2001}. Essa variável pode ser denominada de latente, porque não pode ser medida diretamente, mas é estimado a partir das respostas observadas dos participantes em uma série de itens \cite{pasquali2003}. 


Na educação, a habilidade ou proficiência é a variável latente de interesse devido à sua relevância central na avaliação do desempenho dos alunos e no aprimoramento do ensino. A habilidade é uma representação abstrata e não observável do conhecimento, compreensão e capacidade de aplicação que os alunos possuem em relação a um determinado domínio. A mensuração dessa habilidade pode ser feita por meio de avaliações, que permitem avaliar o progresso individual dos estudantes, identificar áreas de necessidade de apoio e adaptar estratégias de ensino, para atender às demandas específicas de aprendizado.
\begin{comment}
	A avaliação desempenha um papel de extrema importância no contexto educacional, conforme enfatizado por Gimeno (1994) ao afirmar que ``a função fundamental que a avaliação deve cumprir no processo didático é a de informar ou conscientizar os professores acerca de como caminham os acontecimentos em sua turma, os processos de aprendizagem que desencadeiam em cada um de seus alunos, durante o mesmo.'' Essa citação destaca que a avaliação vai além de simplesmente medir resultados; ela serve como um meio essencial para os educadores compreenderem o desenvolvimento de suas turmas. Através da avaliação, os professores podem identificar as necessidades específicas de cada estudante, adaptar suas abordagens pedagógicas e fornecer suporte personalizado, criando um ambiente educacional que promove o crescimento tanto individual quanto coletivo.



A estimativa da habilidade em simulados do oferece aos estudantes uma compreensão precisa do seu nível de preparação e das áreas que requerem melhoria. Isso permite que os alunos identifiquem seus pontos fortes e fracos, direcionando seus esforços de estudo de forma mais eficaz. Além disso, a estimativa da habilidade ajuda os educadores e instituições a adaptar estratégias de ensino e programas de apoio com base nas necessidades reais dos estudantes, contribuindo para um ensino mais eficaz. Para a gestão educacional, a habilidade estimada em simulados pode fornecer informações valiosas sobre o desempenho dos alunos em nível nacional e regional, orientando políticas educacionais informadas e promovendo a melhoria contínua do sistema educacional. 

\end{comment}


No entanto, é necessário entender que, a estimação da habilidade ou do conhecimento do aluno, depende da qualidade dos itens na prova \cite{BORGATTO2012}, portanto a estimativa de parâmetros relacionado ao item desempenha um papel determinante na qualidade das avaliações e na interpretação dos resultados. Esses parâmetros podem ser calculados tanto com a Teoria Clássica dos Testes (TCT), quanto com a Teoria de Resposta ao Item (TRI), que incluem informações sobre a dificuldade, a discriminação e entre outros parâmetros, na qual permitem que os educadores selecionem questões e provas adequadas para medir a habilidade dos alunos de forma precisa. 

A TCT é uma abordagem adotada anteriormente em avaliações, na qual a mensuração do conhecimento e da dificuldade de um item é realizada por meio de métodos como o escore bruto ou a porcentagem de itens respondidos corretamente \cite{pasquali2003}. A TCT, conforme enfatizado por \citeonline{pasquali2018}, apresenta suas próprias limitações e desafios, um dos quais é a dependência dos parâmetros dos itens em relação à amostra de sujeitos na qual esses parâmetros foram originalmente calculados. A dependência da amostra significa que os parâmetros dos itens, como a dificuldade e a discriminação, podem variar com base na composição da amostra de indivíduos que participaram do teste. Isso pode resultar em estimativas instáveis e imprecisas das habilidades dos alunos, especialmente quando se trabalha com diferentes grupos populacionais.

\begin{comment}
	 É importante ressaltar que, de acordo com Pasquali (2003), a TCT não mede diretamente o traço latente e sim avalia o comportamento observado dos alunos em relação aos itens.
\end{comment}


A TRI, por outro lado, aborda esse problema de maneira mais robusta. Ela se baseia em modelos estatísticos que consideram não apenas as respostas dos alunos, mas também as características dos próprios itens. Esses modelos levam em conta a probabilidade de um aluno responder corretamente a um item, com base em sua habilidade subjacente e nos parâmetros do item. Portanto, os parâmetros do item na TRI são considerados propriedades intrínsecas do item e não dependem da amostra de sujeitos. \cite{pasquali2018}. A TRI permite que educadores avaliem não apenas o desempenho global dos alunos, mas também suas habilidades específicas em diferentes áreas do conhecimento. Isso é particularmente valioso porque reconhece que os alunos têm diferentes pontos fortes e fracos, permitindo uma análise mais detalhada das suas necessidades educacionais. 

A família de modelos da TRI foi concebida para abordar a complexidade inerente à mensuração de variáveis latentes. Em particular, a TRI oferece um conjunto de modelos matemáticos e metodologias que permitem a construção de avaliações ou testes que, por sua vez, servem como instrumentos confiáveis para medir as variáveis latentes em questão. Esses testes são cuidadosamente desenvolvidos a partir de uma coleção de itens, cada um projetado para fornecer informações específicas sobre a(s) habilidade(s) ou traço(s) latente(s) que se deseja medir \cite{pasquali2018}.

A incorporação da TRI na avaliação das habilidades dos alunos tem se destacado por sua capacidade de fornecer medidas mais precisas e personalizadas das competências dos estudantes. Atualmente todas as provas no Instituto Nacional de Estudos e Pesquisas Educacionais Anísio Teixeira (INEP) utilizam TRI, como o Sistema de Avaliação de Rendimento Escolar do Estado de São Paulo (SARESP), o Sistema de Avaliação da Educação Básica (SAEB), o Exame Nacional do Ensino Médio (ENEM), o Exame Nacional para Certificação de Competências de Jovens e Adultos (ENCCEJA) e outros. Atualmente o estado de Mato Grosso está fazendo avaliação da educação pelo CAED/UFJF (Centro de Políticas Públicas e Avaliação da Educação), para avaliar e aumentar os índices do IDEB (Índice de Desenvolvimento da Educação Básica) no estado.

Em destaque, tem-se o Enem, o qual é uma avaliação realizada anualmente no Brasil pelo INEP. Criado em 1998, inicialmente como uma forma de avaliar a qualidade do ensino médio no país, o Enem passou a ter múltiplas funções ao longo do tempo. Em 2009 que o Enem adotou a Teoria de Resposta ao Item (TRI) como método de avaliação, substituindo o modelo de avaliação tradicional. Atualmente, além de servir como uma ferramenta de avaliação do sistema educacional, o Enem é fundamental para acesso ao ensino superior em diversas instituições, por meio do Sistema de Seleção Unificada (SISU), Programa Universidade para Todos (ProUni) e Fundo de Financiamento Estudantil (FIES) \cite{inephistorico}.



  

\begin{comment}
	Atualmente, o Enem é dividido em 5 partes, a redação e 4 áreas de conhecimento: Linguagens, Ciências Humanas, Matemática e Ciências da Natureza. Cada uma dessas áreas é considerado como uma habilidade $\theta$. O ENEM estima cada uma dessas habilidades separadamente. Considerando que cada área está medindo o mesmo $\theta$ \cite{inep2021}.
	
	
	Além disso, ao utilizar modelos estatísticos sofisticados, a TRI é capaz de estimar as habilidades latentes dos alunos de forma mais precisa, levando em consideração a dificuldade dos itens e a capacidade discriminativa de cada questão. Dessa forma, a TRI oferece uma abordagem mais justa e confiável para avaliar o progresso dos alunos, fornecendo informações valiosas que podem ser usadas para direcionar o ensino, identificar alunos que precisam de apoio adicional e melhorar o currículo escolar. Consequentemente, a TRI desempenha um papel essencial na promoção de práticas educacionais eficazes e na melhoria contínua da qualidade da educação.
	
\end{comment}


\section{Objetivo Geral}

O objetivo geral é analisar a qualidade dos itens que compõe um simulado de Ciências da Natureza e estimar a habilidade dos respondentes.

\section{Objetivos Específicos}

\begin{itemize}
	
\item Selecionar o melhor modelo TRI para estimação dos parâmetros dos itens e da habilidade.
	
\item Analisar a qualidade dos Itens e da prova utilizando a TCT e a TRI. Determinar se os itens que compõem a prova são adequados para estimar a habilidade.	

\item Analisar se a prova é adequada para medir a habilidade dos alunos aos quais foi aplicada.

\end{itemize}



\section{Justificativa}

Este trabalho se justifica pela importância crítica dos testes educacionais no contexto atual. No âmbito educacional, a avaliação desempenha um papel fundamental, indo além de simplesmente medir o conhecimento dos alunos. Conforme destacado por \cite{silva2019}, ``Avalia-se para diagnosticar, para qualificar e para planejar atividades e estratégias que percebam processos de ensino-aprendizagem, bem como necessidades individuais e coletivas dos estudantes.'' enfatizando a importância da avaliação no contexto educacional.

Ao adotar a TRI como base para este trabalho, reconhecemos a necessidade de uma abordagem mais sofisticada e precisa para a avaliação. A TRI oferece a capacidade de mensurar as habilidades dos alunos de maneira individualizada, levando em consideração a dificuldade, a discriminação e o acerto casual de cada item \cite{pasquali2018}. Isso permite uma análise mais aprofundada do desempenho dos alunos e uma compreensão mais precisa de suas competências em diferentes áreas do conhecimento.

A TRI é utilizada na educação para mensurar e compreender o desempenho dos alunos e a qualidade de avaliações, itens e testes e a evolução dos alunos através do tempo. Por meio da análise estatística, essa prática permite que educadores e pesquisadores identifiquem quais são os itens e testes com mais capacidade de informação para medir habilidades específicas, avaliem a dificuldade dos itens de teste, identifiquem itens enviesados ou com dificuldade inadequada. Compondo assim, uma prova que forneça a habilidade dos examinados com maior precisão.

Além disso, este trabalho se justifica pela relevância da TRI no contexto educacional brasileiro, onde essa abordagem é adotada nas principais avaliações nacionais, estaduais e municipais, como o ENEM, SAEB e SARESP. A compreensão e aplicação adequada da TRI são cruciais para garantir que essas avaliações sejam justas, confiáveis e válidas, contribuindo assim para a melhoria do sistema educacional como um todo.

Ao possibilitar uma análise mais detalhada do desempenho dos alunos, a TRI contribui significativamente para a adaptação de estratégias de ensino, permitindo a identificação de áreas de melhoria no sistema educacional. Além disso, por meio de suas características, a TRI torna possível medir com maior precisão o nível de habilidade dos alunos, ajustando-se de forma mais eficaz às particularidades de cada estudante, o que favorece uma abordagem mais individualizada e eficaz no processo de ensino-aprendizagem.



\begin{comment}
	
	
	Em última análise, este trabalho visa contribuir para a compreensão e aplicação eficaz da TRI na avaliação educacional, fornecendo orientações  sobre como avaliar a qualidade das questões, identificar níveis de habilidade e construir uma avaliação que estime bem as habilidades dos alunos. 
	
	A realização de simulados com notas baseadas na Teoria de Resposta ao Item (TRI) é de também é importante para os alunos que se preparam para o ENEM. Esses simulados proporcionam uma experiência de avaliação mais próxima do exame real, uma vez que a TRI é o método utilizado para calcular as notas no ENEM. Ao fazer esses simulados, os alunos têm a oportunidade de se familiarizar com o formato das questões, entender como a TRI funciona na prática e, mais importante, obter estimativas de suas pontuações reais no ENEM. Isso não apenas ajuda os estudantes a avaliarem seu nível de preparação e identificar pontos fracos e fortes. Isso não apenas promove uma compreensão mais profunda de seu próprio desempenho, mas também uma ferramenta na tomada de decisões sobre como direcionar seus esforços de estudo e preparação para o exame.
	
	
	
	Em última análise, este trabalho visa contribuir para a compreensão e aplicação eficaz da TRI na avaliação educacional, fornecendo orientações  sobre como avaliar a qualidade das questões, identificar áreas de necessidade de apoio e melhorar a equidade no sistema educacional. Ao fazê-lo, esperamos contribuir para aprimorar a qualidade da educação e, consequentemente, o futuro dos alunos brasileiros.
	
``Sabe-se que avaliar, se tais objetivos foram alcançados, não decorre
de uma simples verificação da aprendizagem. Esse diagnóstico vai muito
além, pois há toda uma conjuntura que propicia a aprendizagem do aluno
ou não. No cotidiano, constata-se que o processo pedagógico ocorre por
meio da relação que se estabelece entre professores, alunos, direção,
administração, estrutura física da escola, comunidade, entre outros, e nessa
relação estão envolvidas as múltiplas dimensões que formam cada ser
humano. Portanto, uma avaliação, que pretenda avaliar a qualidade da
educação oferecida por uma escola, por uma rede ou por um sistema, deve
estar embasada em um modelo que contemple todas as relações possíveis
de serem avaliadas.'' (RODRIGUES, 2006).
\end{comment}




\begin{comment}
...
alguns textos para sevir como base:

``Avalia-se para diagnosticar, para qualificar e para planejar atividades e estratégias que percebam processos de ensino-aprendizagem, bem como necessidades individuais e coletivas dos estudantes.'' (SILVA, 2019)

avaliações
associam-se a um processo interpretativo de dados quantitativos e/ou
qualitativos, supondo um juízo de valor, qualidade ou mérito que tem por
meta diagnosticar e verificar o alcance dos objetivos propostos no processo
ensino-aprendizagem.



...
%DA SILVA, André Felipe Zilio et al. Aplicação do modelo de reposta %nominal da TRI a avaliação educacional de larga escala. Sigmae, v. 8, %n. 2, p. 735-741, 2019.


A analise TCT busca fornecer informações sobre o comportamento dos alunos em relação às opções de resposta e permite avaliar se as alternativas estão sendo utilizadas de maneira apropriada, essa análise permite identificar se alguma alternativa está confundido o aluno.


A avaliação dos parâmetros das questões, como dificuldade, discriminação e acerto casual, é essencial para determinar a qualidade das perguntas presentes na prova. Isso assegura que a avaliação seja justa e precisa, pois a qualidade dos itens é diretamente proporcional à qualidade dos resultados.

A investigação sobre se a prova mede apenas uma única variável latente é crucial para garantir a validade da avaliação. Uma prova que mede múltiplas variáveis latentes ou está sujeita a contaminação de construtos não relacionados pode produzir resultados imprecisos e injustos.

A análise comparativa entre diferentes modelos estatísticos busca identificar o modelo mais adequado para a estimativa das habilidades dos alunos, fornecendo uma base sólida para a interpretação dos resultados.

A identificação de itens âncoras é vital para a equalização, pois fornece pontos de referência confiáveis que permitem a comparação dos resultados entre diferentes edições da avaliação. (VALLE, ??)

A delimitação dos níveis de habilidade dos estudantes e a associação de questões específicas a esses níveis aprimoram a compreensão da progressão de dificuldade da avaliação, permitindo uma análise mais precisa do desempenho dos alunos em diferentes estágios.

Por fim, a estimativa de parâmetros dos itens garante que apenas questões de alta qualidade e relevância sejam incluídas no banco de itens, contribuindo para a criação de avaliações de alta qualidade e aprimorando a validade e confiabilidade do processo de avaliação.

Portanto, os objetivos específicos delineados nesta pesquisa se justificam pela necessidade de aprimorar a qualidade da avaliação educacional, garantindo que as medidas obtidas sejam justas, confiáveis e alinhadas com os objetivos pedagógicos. Isso, por sua vez, pode contribuir para uma educação de maior qualidade e para a tomada de decisões educacionais mais informadas.
\end{comment}

