\chapter{REFERENCIAL TEÓRICO}

\section{Teoria do Traço Latente}


 A avaliação de características subjacentes, também conhecidas como variáveis latentes, constitui uma parte intrínseca da compreensão e mensuração de fenômenos complexos, que não podem ser diretamente observados ou medidos. Questões como a avaliação de níveis de conhecimento, aptidões ou estados emocionais, como a depressão, exemplificam situações que não podem ser medidas diretamente. A necessidade de quantificar essas características subjacentes tem impulsionado o desenvolvimento da Teoria do Traço Latente \cite{pasquali2003fundamentos}. A variável latente é comumente conhecida por diversos termos, tais como: variável hipotética, fator, construto, conceito, estrutura psíquica, traço cognitivo, traço latente, processo cognitivo, processo mental, estrutura mental, nota, escore, habilidade, aptidão, componente cognitivo, tendência, proficiência, entre outros.

O conceito de "Teoria do Traço Latente" engloba uma classe de modelos matemáticos e traços subjacentes não diretamente observáveis. A TRI e a TCT são duas das maneira pela quais a Teoria do Traço latente é aplicado, especificamente no contexto de avaliações e testes, para estimar desempenho e habilidade \cite{pasquali2018}.

\section{Teoria Clássica dos Testes}

Diferente da TRI onde o foco está no Item, na TCT, o todo é mais importante, o escore bruto é a pontuação direta obtida em um teste. É uma medida simples que não leva em consideração as diferenças na dificuldade dos itens ou o comportamento dos respondentes em itens específicos. A TCT pode ser útil para uma análise prévia do instrumento de avaliação e dos itens \cite{pasquali1996}.


\begin{comment}
	Na TCT a dificuldade do Item é calculado a partir da proporção de sujeitos que respondem corretamente tal item. O escore bruto é calculado somando-se o número de respostas corretas.
	
 Neste trabalho, são aplicados alguns procedimentos da Teoria Clássica dos Testes, incluindo a análise da Correlação Ponto-Bisserial e do Coeficiente Alfa de Cronbach.
\end{comment}

\subsection{Dificuldade}

Conforme \citeonline{pasquali1996}, em testes de aptidão, a dificuldade de um item é descrita pela porcentagem (ou proporção) de participantes que respondem corretamente, variando de 0 a 1, sendo quanto mais próximo de 1, mais acertos o item teve, portanto mais fácil o item é. O valor do $ID_i$ está diretamente relacionado à média de acertos do item no teste e é calculado  pela fórmula: 

\[
	ID_i =\dfrac{A_i}{n} 
\]
onde $A_i$ é o total de itens corretos e $n$ é o total de respostas. 

\begin{comment}
O calculo do índice de dificuldade é importante pois permite conhecer a distribuição do grau de dificuldade das questões que segundo \citeonline{pasquali2003}, para que uma avaliação educacional tenha um nível de dificuldade ideal, os índices devem seguir uma distribuição próxima à curva normal, com itens fáceis, médios e difíceis.
\end{comment}

O calculo do índice de dificuldade é importante pois permite conhecer a distribuição do grau de dificuldade das questões, \citeonline{pasquali1996} uma avaliação educacional tenha um nível de dificuldade ideal, itens fáceis, médios e difíceis, seguindo a tabela \ref{tabela-class-ID}

O cálculo do índice de dificuldade permite avaliar a distribuição do grau de dificuldade das questões, assegurando que uma avaliação educacional tenha uma variedade equilibrada de itens. Conforme destaca \citeonline{pasquali1996}, uma prova ideal deve conter uma mistura adequada de itens fáceis, médios e difíceis. A tabela \ref{tabela-class-ID} apresenta a distribuição esperada dos itens por faixa de acertos, sugerindo como  uma avaliação bem construída deve ser.


\begin{table}[H]
		\centering
		\caption{Distribuição ideal dos itens por ID.}
		\label{tabela-class-ID}
		\begin{tabular}{ccc}
			\hline
			\textbf{Faixa} & \textbf{\% de Acertos}  &	  \textbf{Distribuição Esperada}   \\ 
			\hline
			\textbf{I}  & 0 a 20 & 10\%   \\ 
			\hline
			\textbf{II} & 20 a 40 &  20\%   \\
			\hline
			\textbf{III} & 40 a 60 &  40\%   \\ 
			\hline
			\textbf{IV} & 60 a 80 &  20\%   \\ 
			\hline
			\textbf{V} & 80 a 100 & 10\%   \\ 
			\hline
		\end{tabular}\\
		\vspace*{0.5cm}
		\small{\textbf{Fonte:} \citeonline{pasquali1996}, p.83}
\end{table}

\subsection{Discriminação}

Na TCT, há diferentes formas de avaliar a discriminação de um item, que é a capacidade do item de diferenciar entre indivíduos com diferentes níveis de habilidade. Um dos métodos mais comuns é o índice de correlação ponto bisserial, que mede a relação entre o acerto no item e a pontuação total na prova. Outro método usado é discriminação clássica, que é baseada na diferença nas proporções de acertos entre os grupos de alto e baixo desempenho.

\subsubsection{Índice de Discriminação Clássico}

O Índice de Discriminação Clássico ($D_i$) avalia a distribuição de respostas entre dois grupos: o grupo de alto e o grupo de baixo escore.
O índice pode ser calculado por: 

\[D_i =  ACIM_i - ABAI_i\] onde $ACIM_i$ representa a proporção de acertos do grupo superior, no qual é formado pelos 27\% maiores escores e $ACIM_i$   representa a proporção de acertos do grupo inferior, no qual é formado pelos 27\% menores escores. 

 O índice $D_i$ varia de -1 a 1 e é calculado como a diferença entre as proporções de respostas apresentadas pelo grupo superior e pelo grupo inferior. Espera-se um valor positivo para a resposta correta, indicando que o grupo superior selecionou mais frequentemente a resposta correta do que o grupo inferior, o que é um indicativo de qualidade do item  \cite{de1983consideraccoes}. 



\subsubsection{Correlação Ponto Bisserial}

O coeficiente bisserial é uma métrica que avalia a correlação entre uma variável dicotômica e uma quantitativa, ou seja, no caso de avaliações, avalia relação entre o desempenho em um item e o desempenho geral na prova \cite{BORGATTO2012}. Ele desempenha um papel importante na análise preliminar dos itens, auxiliando na identificação de questões que podem apresentar problemas, como respostas incorretas no gabarito \cite{de2000teoria}.

A Correlação Bisserial por pontos é indicada quando os itens são dicotômicos, representada pela equação:


\begin{equation}
	r_{bis} = \frac{\bar{X}_p - \bar{X}_t}{S_t}
	\sqrt{\frac{p_i}{1 - p_i}}
\end{equation}

onde:


\noindent $ \bar{X}_p $ é a média dos escores dos examinados que responderam ao item corretamente;

\noindent $ \bar{X}_t $ é a média global dos escores;

\noindent $ S_t $ é o desvio padrão do teste;

\noindent $ p_i $  é a proporção de indivíduos que acertaram o item, ou seja, a proporção para qual a variável binária é 1.


O valor do ponto bisserial de um item oscila de -1 a 1. \cite{pasquali2003} recomenda um valor acima de 0,30. Quando esse valor é inferior, requer uma avaliação pedagógica. Com base nessa análise, o item pode ser submetido a ajustes no gabarito ou considerado para descarte \cite{andrade2010uso}.


\subsection{Alpha de Cronbach}

A avaliação da consistência interna, ou seja, a medida em que as respostas dos participantes em um conjunto de itens, que teoricamente avaliam a mesma habilidade, traço ou construto. 
Para avaliar essa consistência, é comum recorrer a métodos que examinam a correlação entre os itens \cite{souza2017}. Nesse contexto, destacam-se o coeficiente \textit{alpha} de Cronbach, introduzido por Cronbach (1951), este método é umas das ferramentas estatísticas para a avaliação da confiabilidade de instrumentos de medida. Ele auxilia  a determinar se os itens realmente estão medindo o mesmo construto, já que uma alta consistência entre eles sugere que estão alinhados na mensuração da habilidade desejada.

O coeficiente \textit{Alpha} de Cronbach ($\alpha$) pode ser medido por meio da seguinte equação:

\begin{equation}
	\alpha = \frac{k}{k-1}(1 - \frac{\sum_{i=1}^{k}{s^2_i}}{s_T^2})
\end{equation}

 Em que $k$ é o numero de itens do teste, ${s_i^2}$ a variância do item, e
${s_T^2}$ a variância total do teste.

O coeficiente calcula consistência no intervalo de 0 a 1, sendo que quanto mais próximo de 1 maior a consistência, para \citeonline{pasquali2003}, valores entre 0,7 e 0,9 são considerados bons, acima de 0,9 indica itens repetitivos.

\section{Teoria de Resposta ao Item}

A TRI é um conjunto de modelos matemáticos que procuram representar a probabilidade de um indivíduo dar uma certa resposta a um item, com função do parâmetros desses itens e do traço latente, chamada de Theta ($\theta$), do respondente \cite{de2000teoria}. A TRI considera o dado conjunto de respostas ($\textbf{U}$) de um determinado teste como um elemento capaz de fornecer estimativas para a habilidade $\theta$ avaliada. \cite{baker2001}

A escolha do modelo vai depender da natureza dos itens, do questionário, de qual e quantas variáveis latentes que se pretende medir. Na educação, na maioria das vezes, a variável latente medida é o conhecimento do aluno em determinada área, para isso utiliza-se testes com itens de múltipla escolha, que posteriormente será dicotomizado em certo e errado, por isso, será usado modelos para itens dicotômicos. 





\section{Modelos Unidimensionais para Itens Dicotômicos}

\subsection{Modelo Logístico de 1 parâmetro}

O modelo de \citeonline{rasch1960}, também conhecido como Modelo Logístico de 1 Parâmetro (1PL), é um dos modelos mais simples dentro da TRI. Ele pressupõe que a probabilidade de um indivíduo acertar um item depende apenas da diferença entre a habilidade do indivíduo ($\theta$) e a dificuldade do item ($b_i$). Esse modelo não considera parâmetros de discriminação ou acerto casual, assumindo que todos os itens possuem o mesmo poder de discriminação.


A equação que representa a probabilidade de um indivíduo $ j $ com habilidade $ \theta_j $ dar uma resposta correta para o item $ i $ é representada por:

\[
	 P({U_i}_j = 1|{\theta}_j) =
	\frac{1}{1+e^{-D(\theta_j - b_i)}}
\]
 

$\newline$
com $i = 1, 2, ..., I $ e $ j = 1,2, ... , n $, onde:
\newline

\noindent ${U_i}_j$  é uma variável dicotômica que representa a resposta do indivíduo $ j $ para o item $ i $, assumindo o valor 1 quando para resposta correta e 0 para a resposta incorreta.

\noindent ${\theta}_j$  representa o traço latente, ou habilidade, do $ j $-ésimo indivíduo.

\noindent $ b_i $ representa o parâmetro de dificuldade do $i$-ésimo item.

\noindent $ D $ é uma constante fixada em 1,702, introduzida para que a função forneça resultados parecidos ao da curva normal.

A \autoref{fig:rasch} representa um exemplo de curva característica do Item para o modelo de 1 parâmetro, neste exemplo, o valor da dificuldade ($b_i$) é igual a 2.

\begin{figure}[H]
	\centering
	\caption{Exemplo de curva característica do item para modelo 1PL.}
	\includegraphics[width=12cm]{rasch.png}
	\parbox{\textwidth}{
		\centering % 
		\makebox[12cm][l]{
			\parbox{12cm}{
				\raggedright
				\small \textbf{Fonte}: Elaborado pelos autores.
			}
		}
	}
	\label{fig:rasch}
\end{figure}


O valor b = 2 é o ponto de inflexão da curva e para o modelo 1PL significa que a probabilidade de um indivíduo com habilidade $\theta = 2$ acertar o item é de 50\%. Para indivíduos com habilidade maior que 2, a probabilidade de acerto é maior que 50\% e sobe conforme a habilidade do indivíduo aumenta. 



\subsection{Modelo Logístico de 2 parâmetros}

O Modelo Logístico de 2 Parâmeros, também conhecido como o Modelo de Birnbaum é uma extensão do Modelo de Rasch.
Desenvolvido por Allan \citeonline{birnbaum1968}, o modelo de 2PL conta com 2 Parâmetros Logísticos: dificuldade e discriminação, a segunda refere-se ao quanto o item diferencia respondentes de habilidades diferentes.

A fórmula associada ao modelo de 2 parâmetros é a seguinte:

\[
	\label{eq:2PL}
	P({U_i}_j = 1|{\theta}_j, a_i, b_i) =
	\frac{1}{1+e^{-Da_i(\theta_j- b_i)}}
\]


$\newline$
com $i = 1, 2, ..., I $ e $ j = 1,2, ... , n $, onde:
\newline

\noindent $ a_i $ representa o parâmetro de discriminação do $i$-ésimo item.\\

 A diferença entre o Modelo 1PL e o Modelo 2PL está no fato de que o Modelo de 1PL pressupõe que todos os itens diferenciam igualmente, ou seja, que $a_i = 1$.
 
A \autoref{fig:2PL} é um exemplo de uma curva característica do item onde o parâmetro $a_i$ pode variar, neste caso, vale 1,8.


\begin{figure}[H]
	\centering
	\caption{Exemplo de curvas características do item para modelo 2PL.}
	\includegraphics[width=16cm]{../2PL.png}
	\parbox{\textwidth}{
		\centering % 
		\makebox[16cm][l]{
			\parbox{16cm}{
				\raggedright
				\small \textbf{Fonte}: Elaborado pelos autores.
			}
		}
	}
	\label{fig:2PL}
\end{figure}


\subsection{Modelo Logístico de 3 parâmetros}

Desenvolvido por \citeonline{lord1980}, o Modelo Logístico de 3 Parâmetros, ou Modelo de 3PL, é uma extensão do Modelo de 2 Parâmetros. O Modelo de 3PL adiciona um terceiro Parâmetro a função: o de acerto casual (ou "chute"), que representa a probabilidade de um respondente responder corretamente a um item, mesmo que ele não tenha a habilidade necessários para fazê-lo. 

A equação do Modelo de 3PL, a qual representa a probabilidade de chutar e acertar mais a probabilidade de não chutar e acertar, é dada por:

\begin{equation}\label{eq:3PL}
	P({U_i}_j = 1|{\theta}_j, a_i, b_i, c_i) =
	c_i+(1-c_i)\frac{1}{1+e^{-Da_i(\theta_j- b_i)}}
\end{equation}
$\newline$
com $i = 1, 2, ..., I $ e $ j = 1,2, ... , n $, onde:
\newline

\noindent $c_i$ representa o acerto casual do $i$-ésimo item.\\


A figura \ref{fig:3PL} apresenta os mesmos parâmetros $a$ e $b$ (valendo 1) para 2 curvas com diferentes valores de $c$.

A Figura \ref{fig:3PL} ilustra duas curvas com os mesmos parâmetros de discriminação $a = 1$ e dificuldade $b = 1$, mas com valores distintos para o parâmetro de acerto ao acaso $c$. As curvas evidenciam o impacto desse parâmetro, que ajusta a probabilidade mínima de acerto, mesmo para indivíduos com baixa habilidade. 

\begin{figure}[H]
	\centering
	\caption{Exemplo de curva característica do item para modelo 3PL.}
	\includegraphics[width=16cm]{../3PLcci.png}
	\parbox{\textwidth}{
		\centering % 
		\makebox[16cm][l]{
			\parbox{16cm}{
				\raggedright
				\small \textbf{Fonte}: Elaborado pelos autores.
			}
		}
	}
	\label{fig:3PL}
\end{figure}


\subsection{Função de Informação do Item}

A função de informação do item descreve o poder informativo de um item. Essa função fornece informações sobre a precisão com que um item pode estimar a habilidade latente de um respondente. \cite{de2000teoria} 
A função de informação do item é dada por:

\[
		I_i(\theta) = \dfrac{[\frac{d}{d\theta}P_i(\theta)]^2}{P_i(\theta)Q_i(\theta)}
\]

Para os modelos TRI, a função pode ser descrita como:

\begin{equation}\label{eq:info_item}
	I_i(\theta) = D^2 a_i^2\frac{Q_i(\theta)}{P_i(\theta)} \left[\frac{P_i(\theta) - c_i}{1 - c_i}\right]^2
\end{equation}

onde:

\noindent $I_i(\theta) $ representa a informação fornecida pelo item $i$ no nível de habilidade $\theta$;\\

\noindent $P_i(\theta) = P(U_{ij} = 1| \theta) $ e $ Q_i(\theta) = 1 - P_i(\theta) $ \\

Valores maiores de $a_i$ significam que o item discrimina melhor, gerando uma maior quantidade de informação, especialmente ao redor do ponto em que $\theta$ se aproxima de $b_i$. A equação \ref{eq:info_item} mostra que a quantidade de informação é diretamente proporcional ao quadrado de $a_i$, ou seja, quanto maior o valor de $a_i$, maior será a informação fornecida pelo item em uma dada habilidade $\theta$.

A figura \ref{fig:fii} ilustra a função de informação para um item com $b_i = -2$, observa-se que a informação também é maior quando $\theta$ se aproxima de $b_i$, nesse exemplo, a informação é maior quando $\theta$ se aproxima de $-2$.

\begin{figure}[H]
	\centering
	\caption{Função de Informação do Item.}
	\includegraphics[width=16cm]{../informacao_e_cci.png}
		\parbox{\textwidth}{
		\centering % 
		\makebox[16cm][l]{
			\parbox{16cm}{
				\raggedright
				\small \textbf{Fonte}: Elaborado pelos autores.
			}
		}
	}
	\label{fig:fii}
\end{figure}

\subsection{Função de Informação do Teste}

A informação do teste é obtida pela soma das informações fornecidas pelos itens que compõem a prova \cite{de2000teoria}. Ou seja:

\begin{equation}\label{eq:info_teste}
I(\theta) = \sum_{i=1}^{I}I_i(\theta)
\end{equation}

 O erro-padrão de estimação é calculado como o inverso da raiz quadrada da informação (I) sobre a variável latente ($\theta$), conforme a seguinte fórmula:

\[
EP(\theta) = \dfrac{1}{\sqrt{I(\theta)}}
\]


O erro padrão da medida é inversamente proporcional à informação sobre a variável latente, o que significa que quanto menor a informação, maior será o erro associado. A relação entre a curva de informação do teste e o erro padrão da estimativa pode ser observada na Figura \ref{fig:fft}.

A função de informação do teste é uma medida útil pois permite analisar o quanto o instrumento trás de informação sobre a habilidade e em quais regiões o instrumento estima melhor.

\begin{figure}[H]
	\centering
	\caption{Informação do Teste e Erro Padrão.}
	\includegraphics[width=16cm]{../fft.png}
		\parbox{\textwidth}{
		\centering % 
		\makebox[16cm][l]{
			\parbox{16cm}{
				\raggedright
				\small \textbf{Fonte}: Elaborado pelos autores.
			}
		}
	}
	\label{fig:fft}
\end{figure}



\subsection{Pressupostos dos Modelos Unidimensionais}

Os 3 modelos citados anteriormente são unidimensionais e possuem dois pressupostos principais, na qual os itens de um teste ou simulado devem ser elaborados de forma que atendam esses requisitos. 

\begin{itemize}
	
\item  Unidimensionalidade

Pressupõe que os itens que compõe um teste estão medindo apenas uma traço latente, ou seja, deve haver apenas uma aptidão capaz de realizar os itens do teste. Isso implica que a probabilidade de resposta correta depende apenas da habilidade latente $\theta$ e dos parâmetros do item, sem considerar outras habilidades latentes \cite{pasquali1996}.

\item  Independência Local

Este postulado implica que os itens são respondidos de forma independente para um dado $\theta$. Em outras palavras, as respostas a um item não são afetadas pelas respostas a outros itens após controlar a habilidade latente do indivíduo, assim como diferentes indivíduos do teste são independentes entre si. Isso implica que a probabilidade para respondentes com uma habilidade dada, a probabilidade de resposta a um conjunto de itens é igual aos produtos das probabilidades das respostas do respondente em cada item \cite{pasquali1996}. Matematicamente, isso pode ser expresso como:


\[
 P(U_{1j} = 1, U_{2j} = 1, \cdots, U_{Ij} = 1 |\theta_j) = 
 P(U_{1j} = 1|\theta_j)  P(U_{2j} = 1|\theta_j) \cdots P(U_{Ij} = 1|\theta_j) 
\] 
\[
 = \prod_{i=1}^{I}P(U_{ij} = 1|\theta_j) 
\]

\end{itemize}
\section{Estimação dos Parâmetros}



Um dos desafios do TRI é a estimação dos parâmetros, no geral, temos 3 situações:  (i) Habilidade é conhecida e deseja-se estimar os parâmetros dos itens; (ii) Os parâmetros dos itens são desconhecidos e deseja-se estimar a habilidade; (iii) Ambos os parâmetros são desconhecidos \cite{de2000teoria}.  Nesse contexto, as notações utilizadas para representar os parâmetros a serem estimados são as seguintes:\\


\noindent $ \boldsymbol{\theta} = (\theta_1, \cdots, \theta_n) $  representa o vetor de habilidades dos $n$ indivíduos e

\noindent $ \boldsymbol{\zeta} = (\boldsymbol{\zeta}_1, \cdots, \boldsymbol{\zeta}_I) $ o conjunto de parâmetros dos itens.

\[ \textbf{U}_{n\times I} =  
\begin{bmatrix}
	u_{11} & u_{12} & \cdots & u_{1I} \\
	u_{21} & u_{22} & \cdots & u_{2I} \\
	\vdots & \vdots & \ddots & \vdots\\
	u_{n1} & u_{n2} & \cdots & u_{nI}
\end{bmatrix}
\]

\noindent $\textbf{U}_{n\times I}$ representa a matriz dicotômica de respostas de $ n $ respondentes e $ I $ Itens. A variável $U_{ji}$ é uma variável dicotômica com distribuição Bernoulli, sendo:

\[U_{ji} =    \begin{cases}
	
	  1, & \mbox{resposta correta;}  \\
	
	  0, & \mbox{resposta incorreta.}
	
\end{cases}
\]

Portanto,  

\begin{equation} \label{eq:bern}
P(U_{ji} = u_{ji}|\theta_j, \zeta_i) = P(U_{ji} = 1|\theta_j, \zeta_i)^{u_{ji}}
P(U_{ji} = 0|\theta_j, \zeta_i)^{1 - u_{ji}} = P_{ji}^{u_{ji}}Q_{ji}^{1-u_{ji}}
\end{equation}

Existem diferentes métodos de estimação de parâmetros na TRI, como a máxima verossimilhança, que maximiza a função de verossimilhança dos dados, a máxima verossimilhança marginal, métodos Bayesianos, que utilizam distribuições prévias, etc.

\subsection{Estimador de Máxima Verossimilhança (EMV)}

Considerando a situação (i), onde $ \boldsymbol{\theta} $ é conhecido, dados os pressupostos de independência e unidimensionalidade da TRI e pela equação (\ref{eq:bern}). A função de verossimilhança de  $ \boldsymbol{\zeta} $ pode ser escrita como:
\[
L(\boldsymbol{\zeta}) =  \prod_{j=1}^{n}\prod_{i=1}^{I}P(U_{ij} = u_{ji}|\theta_j, \zeta_i) = \prod_{j=1}^{n}\prod_{i=1}^{I}P_{ji}^{u_{ji}}Q_{ji}^{1-u_{ji}}
\]

Os Estimadores de Máxima Verossimilhança (EMV) para $ \boldsymbol{\zeta}_i = (a_i, b_i , c_i )$, dados por:\\


\[
\begin{aligned}
	\hat{a_i}: & \ D(1 - c_i)\sum_{j=1}^{n}(u_{ji} - P_{ji})(\theta_j - b_i)W_{ij} = 0, \\
	\hat{b_i}: & \ -Da_i(1 - c_i)\sum_{j=1}^{n}(u_{ji} - P_{ji})W_{ij} = 0, \\
	\hat{c_i}: & \ \sum_{j=1}^{n}(u_{ji} - P_{ji})\frac{W_{ij}}{P^*_{ij}} = 0
\end{aligned}
\]

\noindent onde:\\

\noindent $P_{ji} = P(U_{ij}|\theta_j,\zeta_i)$ representada pelo modelo 3PL dado na equação \ref{eq:3PL};\\

\noindent $P^*_{ij} = \{1 + e^{-Da_i(\theta_j - b_j)}\}^{-1} $ representa a probabilidade do $j$-ésimo indivíduo acertar o $i$-ésimo item, outra forma da equação \ref{eq:2PL};\\

\noindent $Q^*_{ij} = 1 - P^*_{ij} $ representa a probabilidade do $j$-ésimo indivíduo não acertar o $i$-ésimo item;\\
\\
\noindent $W_{ij} = \dfrac{P^*_{ij}Q^*_{ij}}{P_{ij}Q_{ij}} $ ;\\

Os parâmetros dos itens também podem ser estimados em forma de quadratura, onde a distribuição da habilidade latente ($\theta$) é aproximada por um conjunto de pontos de quadratura ($\theta_k$)  e pesos $A_k$, $k = 1, \dots, q$ associados. Esses pontos e pesos,  são usados para calcular a função de verossimilhança em relação à habilidade, permitindo a estimativa dos parâmetros sem a necessidade de integrar analiticamente a distribuição de habilidade contínua.

\[
\begin{aligned}
	\hat{a_i}: & \ D(1 - c_i)\sum_{k=1}^{q}[r_{ki} - P_{ki}f_{ki}](\theta_k - b_i)W_{ki} = 0, \\
	\hat{b_i}: & \ -Da_i(1 - c_i)\sum_{k=1}^{q}\left[r_{ki} - P_{ki}f_{ki}\right]W_{ki} = 0, \\
	\hat{c_i}: & \ \sum_{k=1}^{q}(r_{ki} - P_{ki}f_{ki})\frac{W_{ki}}{P^*_{ki}} = 0
\end{aligned}
\]



Na situação (ii), onde $\boldsymbol{\zeta}$ é conhecido, o EMV para $ \theta_j $,  é dado por:\\


\[ \hat{\theta_j} : D\sum_{i=1}^{I}{a_i(1-c_i)(u_{ji}-P_{ji})W_{ji}} = 0  \] 

O desenvolvimento detalhado pode ser encontrado pelo leitor no livro de \citeonline{de2000teoria}.

Os Estimadores de Máxima Verossimilhança (EMV) de $ \boldsymbol{\zeta} $ e $ \boldsymbol{\theta} $ requerem um método iterativo para sua estimação, uma vez que não possuem uma solução direta. Em geral, a estimação dos parâmetros é através do método de Newton-Raphoson, mas também podem ser utilizados os métodos de \textit{Scoring} de Fisher ou o Algoritmo EM para essa finalidade.

\begin{comment}
 \citeonline{de2000teoria}. Sendo assim, considerando $\boldsymbol{\hat{\zeta}}^{(t)}_{i}$
uma estimativa de $\boldsymbol{\hat{\zeta}}$ na iteração $t$, o procedimento de
Newton-Raphson é dado por:\\

\noindent $ \boldsymbol{\hat{\zeta}}^{(t+1)}_{i} = \boldsymbol{\hat{\zeta}}^{(t)}_{i} + [\bold{H}(\boldsymbol{\hat{\zeta}}^{(t)})]^{-1}\bold{h}({\boldsymbol{\hat{\zeta}}^{(t)}})
$, onde $\bold{H}$ representa a matriz Hessiana.\\
\end{comment}


Na situação (iii), na qual tanto $ \boldsymbol{\hat{\zeta}} $ quanto $ \boldsymbol{\theta} $ são desconhecidos, é a situação mais comum, e requer uma estimação conjunta de ambos.
Para tal finalidade, \citeonline{birnbaum1968}, propôs um algoritmo onde os parâmetros são estimados individualmente, usando um do métodos iterativos citados anteriormente.
O processo é dividido em dois estágios. Inicia-se com um chute inicial, uma estimativa grosseira de $ \boldsymbol{\theta} $ considerando que $ \boldsymbol{\zeta} $ é conhecido, após estimada a habilidade, a segunda parte consiste em estimar $ \boldsymbol{\zeta} $ considerando $ \boldsymbol{\theta} $ (estimado na primeira etapa) como conhecido. Esse processo \textit{vai} e \textit{volta} é repetido até a convergência de ambos os parâmetros.



\begin{comment}
Além do método de Estimação de Máxima Verossimilhança (EMV), existem outros métodos de estimação dos parâmetros, que são frequentemente utilizados em contextos de avaliação educacional, estes incluem:
\end{comment}

\subsection{Algoritmo EM}


O algoritmo EM (Esperança-Maximizaçao) introduzido por \citeonline{dempster1977} é uma técnica utilizada particularmente para a estimação por máxima verossimilhança na presença de variáveis latentes, como a habilidade dos indivíduos ($\theta$), que não pode ser diretamente observada.  O foco está em ajustar os parâmetros dos itens de modo que o modelo melhore sua adequação aos dados.

Foi proposto por \citeonline{bock1981} para a estimação de máxima verossimilhança dos parâmetros de itens no contexto da TRI. O objetivo principal é estimar parâmetros dos itens na presença de variáveis não observáveis, como as habilidades latentes dos indivíduos ($\theta$). Neste contexto, $\boldsymbol{u}_{..}$ representa o vetor de dados incompletos (as respostas observadas), enquanto $\boldsymbol{u}_{..},\boldsymbol{\theta}$ constitui o vetor de dados completos (respostas observadas e as habilidades latentes). A função de densidade conjunta dos dados completos é denotada por $f(\boldsymbol{u}_{..}|\boldsymbol{\theta})$, e ela permite modelar o relacionamento entre as respostas dos indivíduos e suas habilidades não observadas \cite{de2000teoria}.

Seja $\boldsymbol{\hat{\zeta}}^{(k)}$ uma estimativa de $\boldsymbol{\zeta}$ na interação $t$, os passos E e M são:
\begin{itemize}
	\item[] \textbf{Passo E:} Com base nas estimativas atuais dos parâmetros dos itens, o algoritmo calcula o valor esperado do logaritmo da verossimilhança, ou seja:
	\[
	E\left[log f(\boldsymbol{u}_{..}, \boldsymbol{\theta}) | \boldsymbol{u}_{..}, \boldsymbol{\hat{\zeta}}^{(k)}\right] 
	\]
	
	\item[] \textbf{Passo M:} 	Obter $\boldsymbol{\hat{\zeta}}^{(k+1)}$ que maximiza a função do Passo E.
\end{itemize}

Se os modelos de TRI fizerem parte da família exponencial de distribuições, como no caso do modelo de Rasch (1PL), o processo de estimação dos parâmetros seria relativamente simples. No entanto, como os modelos de 2 e 3 parâmetros não fazem parte dessa família, uma forma modificada do algoritmo EM deve ser aplicada para estimar os parâmetros \cite{de2000teoria}.

Para calcular o algoritimo EM para os modelos 2PL e 3PL, primeiro, supõe-se que as habilidades dos indivíduos estejam distribuídas em um conjunto de $q$ valores distintos, denotados por 
$\theta_k, k = 1, \dots, q$. Esses valores correspondem a pontos de quadratura, usados para aproximar integrais complexas necessárias para a estimação dos parâmetros.

Para cada item $i$, temos:

\noindent $f_{ki}$: representa o número de indivíduos com habilidade $\overline{\theta}_k$ que responderam ao item.\\
\noindent $r_{ki}$: é o número de indivíduos com habilidade $\overline{\theta}_k$ que acertaram ao item. 

	
\begin{itemize}
	\item[] \textbf{Passo E:} Usam-se os pontos de quadratura $\bar{\theta}_k$ e os pesos \(A_k\), para $k = 1, \dots, q$, junto com as estimativas iniciais dos parâmetros dos itens, $\boldsymbol{\hat{\zeta}}_i$, para gerar \(g_{kj}^*(\bar{\theta}_k)\). Posteriormente, calcula-se \(\bar{r}_{ki}\) e \(\bar{f}_{ki}\).
	
	\item[] \textbf{Passo M:} Com os valores $r$ e $f$ obtidos no Passo E, resolvem-se as equações de estimação dos parâmetros $\boldsymbol{\zeta}_i$, utilizando o algoritmo de Newton-Raphson ou o método de "Scoring" de Fisher.
\end{itemize}

sendo:  \(
	g^*_j(\theta_k) = \dfrac{ P(\boldsymbol{u}_{j.}|\bar{\theta}_k, \boldsymbol{\zeta})A_k
	}{\sum_{k=1}^{q}P(\boldsymbol{u}_{j.}|\bar{\theta}_k)A_k}
\)
	
\begin{comment}
\subsection{Estimação de Máxima Verossimilhança Marginal (EMVM)}

 Proposto por \citeonline{bock1970} este método estima os parâmetros dos itens e as habilidades dos respondentes em duas etapas, levando em consideração as margens das distribuições das habilidades dos respondentes. É útil quando o número de indivíduos é grande, apresentando vantagens computacionais.
 
 A proposta desse método é realizar a estimação em duas etapas: inicialmente, os parâmetros dos itens são estimados, e depois as habilidades dos indivíduos. Como as habilidades não são diretamente observáveis, é necessário aplicar uma técnica que permita remover a dependência direta da verossimilhança em relação às habilidades, facilitando assim o processo de estimação \cite{de2000teoria}.
 
Para isso, assume-se que os respondentes são uma amostra de uma população cuja a habilidade segue uma determinada função de densidade $g(\theta|\boldsymbol{\eta})$, onde $\boldsymbol{\eta}$ é o vetor de parâmetros da distribuição da habilidade, no caso de uma distribuição normal padrão $\boldsymbol{\eta} = (\mu = 0, \sigma = 1) $. Usando a independência entre indivíduos a probabilidade associada ao vetor de respostas $\boldsymbol{U}$, pode ser escrito como:

\[
	P(\boldsymbol{u}_{..}|\boldsymbol{\zeta}) = \prod_{j=1}^{n}P(\boldsymbol{u}_{j.}|\boldsymbol{\zeta}, \boldsymbol{\eta})
\]

A equações de estimação usando EMVM para $ \boldsymbol{\zeta}_i$, são:\\

 \[
	 \hat{a}_i : D(1-c_i)\sum_{j=1}^{s}r_j
	 \int_{\mathbb{R}}\left[(u_{ji} - P_i)(\theta - b_i)W_i\right] g_j^*(\theta)d\theta = 0,
\]
 
\[
	 \hat{b}_i : -Da_i(1-c_i)\sum_{j=1}^{s}r_j
	 \int_{\mathbb{R}}\left[(u_{ji} - P_i)W_i \right] g_j^*(\theta)d\theta = 0,
\]
 
\[
 	\hat{c}_i : \sum_{j=1}^{s}r_j \int_{\mathbb{R}}
 	\left[(u_{ji}  - P_i)\frac{W_i}{P^*_i}\right]
 	 g_j^*(\theta)d\theta = 0
\]

onde:\\ 

\(
	g^*_j(\theta) \equiv
	g(\theta|\boldsymbol{u}_{j.}, \boldsymbol{\zeta}, \boldsymbol{\eta}) =
	\dfrac{
		P(\boldsymbol{u}_{j.}|\theta,\boldsymbol{\zeta})g(\theta|\boldsymbol{\eta})
		}{
		P(\boldsymbol{u}_{j.}|\boldsymbol{\zeta}, \boldsymbol{\eta})
		}
\)



\end{comment}
\subsection{Estimação Bayesiana}

A abordagem bayesiana, proposta por \citeonline{mislevy1986} é usada para estimar os parâmetros da TRI com base na distribuição \textit{a priori} dos parâmetros. Ela incorpora informações prévias sobre os parâmetros e atualiza essas informações com base nas respostas dos participantes. O ENEM e utiliza o método EAP (\textit{Expected a Posteriori}) para estimar as habilidades dos participantes \cite{inep2021}.

Considera-se que a distribuição da habilidade é função de parâmetros com densidade $g(\theta| \boldsymbol{\eta})$. Assume-se que a distribuição \textit{a priori} para $\theta_j$ é normal com vetor de parâmetros
$\boldsymbol{\eta} = (\mu, \sigma^2)$ 

\subsubsection{Estimação das habilidades pela média a posteriori - EAP}

A estimação de \(\theta_j\) pela média da posteriori consiste em obter a esperança da \textit{posteriori}, que pode ser escrita como:

\[
g(\theta | \boldsymbol{u}_{j,\cdot}, \boldsymbol{\zeta}, \boldsymbol{\eta}) = \dfrac{P(\boldsymbol{u}_{j.} | \theta, \boldsymbol{\zeta}) g(\theta | \boldsymbol{\eta})}{P(\boldsymbol{u}_{j.} | \boldsymbol{\zeta}, \boldsymbol{\eta})}
\]

Segue que a esperança da posteriori é

\[
\hat{\theta}_j = E[\theta_j | \boldsymbol{u}_{j.}, \boldsymbol{\zeta}, \boldsymbol{\eta}] = \dfrac{\int_{\mathbb{R}} \theta g(\theta | \boldsymbol{\eta}) P(\boldsymbol{u}_{j.} | \theta, \boldsymbol{\zeta}) d\theta}{\int_{\mathbb{R}} g(\theta |\boldsymbol{\eta}) P(\boldsymbol{u}_{j.} | \theta, \boldsymbol{\zeta}) d\theta}.
\]


\section{Escala}




Na Teoria de Resposta ao Item (TRI), os resultados da habilidade ($\theta$) e dos parâmetros dos itens são geralmente reportados na escala normal padrão com $\mu = 0$ e $\sigma = 1$. No entanto, para algumas aplicações, como no ENEM, esses valores são transformados para uma escala com $\mu = 500$ e $\sigma = 100$, %$\theta_{\text{ENEM}} \sim N(500, 100)$ 
\cite{inep2021procedimento}.

As transformações da habilidade de uma escala padrão para a escala ENEM é dada por:


\[
	\theta_{\text{(500,100)}} = 100 \theta_{(0,1)} + 500
\]


\begin{comment}
\begin{itemize}
	
	\item Habilidade	
	
\item Dificuldade


\[
b_{\text{ENEM}} = 500 + 100 \times b_z 
\]

\item Discriminação

\[
a_{\text{ENEM}} = \frac{a_z}{100}
\]

\item Chute

\[
c_{\text{ENEM}} = c_z
\]

\end{itemize}
\end{comment}


A escala utilizada para os parâmetros de habilidade ($\theta$) e dos itens é flexível, pois o que realmente importa é a ordem relativa entre esses parâmetros \cite{de2000teoria}. Essa ordenação determina as relações entre as habilidades dos indivíduos e as características dos itens, e se mantém consistente independentemente da escala adotada. Ao transformar a escala padrão normal para a escala utilizada pelo ENEM, a hierarquia dos indivíduos em termos de habilidade e a dificuldade dos itens permanecem as mesmas, facilitando a interpretação sem alterar as conclusões essenciais da análise.

\begin{comment}

\section{Itens Âncoras}

A escala de habilidade é construída a partir dos dados coletados por meio do desempenho dos participantes nos itens do teste. Ela atribui um valor numérico que representa o nível de proficiência de cada participante no domínio avaliado. Quanto maior o valor na escala, maior é o grau de habilidade ou competência do indivíduo naquele domínio específico. A inclusão de múltiplos níveis de dificuldade ou complexidade em um teste visa avaliar o desempenho dos participantes. Esses níveis âncoras representam diferentes graus de proficiência ou habilidade, permitindo comparações mais precisas entre os participantes. Essa abordagem auxilia na definição de critérios de desempenho e na avaliação da capacidade do teste em diferenciar os participantes com base em seus distintos níveis de habilidade \cite{valle200}.

O conceito de "item âncora" refere-se a um item específico em um teste ou avaliação que é usado como referência para comparar o desempenho dos participantes em diferentes edições do teste. A ideia é que o item âncora permaneça constante ao longo do tempo e seja usado para garantir a equidade e a comparabilidade dos resultados entre diferentes grupos de participantes ou em diferentes momentos de aplicação do teste. De acordo com \citeonline{de2000teoria}, os 3 requisitos para um item ser considerado "âncora" são:

\begin{enumerate}
	\item Ser respondido corretamente por pelo menos 65\% dos respondentes com aquele nível de habilidade.
	
	\[
	P(U = 1| \theta = Z ) \geq 0,65
	\]
	
	\item Ser respondido corretamente por no máximo 50\% dos respondentes que estão em um nível abaixo.
	
	\[
	P(U = 1| \theta = Y ) < 0,50
	\]
	
	\item A diferença entre a proporção de respondentes de diferentes níveis deve ser de pelo menos 30\%.
	
	\[
	P(U = 1| \theta = Z ) - P(U = 1| \theta = Y ) \geq  0,30
	\]
\end{enumerate}

\end{comment}









