\chapter{RESULTADOS}

\section{Análise TCT}

Segundo a ótica TCT, que analisa o escore bruto, ou seja, a soma de todos os acertos, o simulado apresentou notas que variam entre 2 e 26 pontos em um total de 30 questões. A média geral foi de 16 pontos, com desvio padrão de 4,6 pontos. A figura \ref{fig:hist_acertos} a distribuição de frequência de acertos no teste.

\begin{figure}[H]
	\centering
	\includegraphics[width=16cm]{hist_acertos.png}
	\caption{Distribuição do total de acertos do simulado.}
	\label{fig:hist_acertos}
\end{figure}

\begin{table}[H]
	\IBGEtab{%
		\caption{Índices TCT.}
		\label{tabela-tct}
	}{%
		\begin{tabular*}{.8\textwidth}{@{\extracolsep{\fill}}clccccc@{}}
			\toprule
			Item & Origem & \% Erro & \% Acerto & Discriminação & Bisserial & Cronbach \\ 
			\midrule \midrule
			1 & FAAP & 32,2\% & 67,8\% & 0,552 & 0,468 & 0,730 \\ 
			\midrule
			2 & PUC & 56,0\% & 44,0\% & 0,394 & 0,319 & 0,741 \\ 
			\midrule
			3 & FGV-RJ & 49,4\% & 50,6\% & 0,551 & 0,468 & 0,730 \\ 
			\midrule
			4 & UEA & 21,5\% & 78,5\% & 0,464 & 0,486 & 0,729 \\ 
			\midrule
			5 & UFPR & 61,0\% & 39,0\% & 0,507 & 0,415 & 0,734 \\ 
			\midrule
			6 & UNICENTRO & 63,7\% & 36,3\% & 0,367 & 0,317 & 0,740 \\
			\midrule
			7 & ENEM & 48,6\% & 51,4\% & 0,523 & 0,431 & 0,733 \\ 
			\midrule
			8 & UFMS & 41,4\% & 58,6\% & 0,402 & 0,358 & 0,738 \\ 
			\midrule
			9 & UEA & 73,9\% & 26,1\% & 0,307 & 0,290 & 0,741 \\ 
			\midrule
			10 & EMESCAM & 23,0\% & 77,0\% & 0,390 & 0,402 & 0,734 \\ 
			\midrule
			11 & UFMS & 11,9\% & 88,1\% & 0,366 & 0,488 & 0,731 \\ 
			\midrule
			12 & PUC-RIO & 32,8\% & 67,2\% & 0,564 & 0,507 & 0,727 \\ 
			\midrule
			13 & ENEM-Digital & 54,1\% & 45,9\% & 0,541 & 0,449 & 0,731 \\ 
			\midrule
			14 & IFPR & 61,0\% & 39,0\% & 0,358 & 0,299 & 0,742 \\ 
			\midrule
			15 & ESPCEX & 86,1\% & 13,9\% & 0,155 & 0,205 & 0,744 \\ 
			\midrule
			16 & UNIFESO & 24,4\% & 75,6\% & 0,441 & 0,431 & 0,733 \\ 
			\midrule
			17 & FASM & 25,0\% & 75,0\% & 0,399 & 0,399 & 0,735 \\ 
			\midrule
			18 & ENEM & 22,4\% & 77,6\% & 0,423 & 0,439 & 0,732 \\ 
			\midrule
			19 & UFSCAR & 19,4\% & 80,6\% & 0,391 & 0,431 & 0,733 \\ 
			\midrule
			20 & ENEM & 20,8\% & 79,2\% & 0,297 & 0,306 & 0,740 \\ 
			\midrule
			21 & UFSCAR & 8,1\% & 91,9\% & 0,242 & 0,450 & 0,734 \\ 
			\midrule
			22 & UNICENTRO & 65,7\% & 34,3\% & 0,241 & 0,235 & 0,745 \\ 
			\midrule
			23 & UNICENTRO & 49,2\% & 50,8\% & 0,429 & 0,357 & 0,738 \\ 
			\midrule
			24 & UFAL & 22,1\% & 77,9\% & 0,307 & 0,336 & 0,738 \\ 
			\midrule
			25 & ENEM & 61,1\% & 38,9\% & 0,399 & 0,350 & 0,738 \\ 
			\midrule
			26 & FAMECA & 51,1\% & 48,9\% & 0,390 & 0,305 & 0,742 \\ 
			\midrule
			27 & UFMS & 85,8\% & 14,2\% & -0,0004 & 0,006 & 0,753 \\ 
			\midrule
			28 & UPE & 77,0\% & 23,0\% & -0,068 & -0,061 & 0,760 \\ 
			\midrule
			29 & FAMECA & 57,5\% & 42,5\% & 0,389 & 0,324 & 0,740 \\ 
			\midrule
			30 & UNCISAL & 87,5\% & 12,5\% & 0,078 & 0,119 & 0,747 \\ 
			\bottomrule
		\end{tabular*}
	}{%
		\fonte{Produzido pelos autores.}
	}
\end{table}


A Tabela \ref{tabela-tct} apresenta os principais índices da Teoria Clássica dos Testes (TCT) para os itens. Inicialmente, observa-se que os itens 27 e 28 possuem índices de discriminação negativos, indicando que esses itens foram mais acertados por indivíduos de menor habilidade. Além disso, esses itens apresentam coeficientes bisserial muito baixos, sugerindo que ambos necessitam de revisão. Também se nota que esses itens são difíceis, pois têm uma proporção de acertos (DIFI) muito baixa.


O alfa de Cronbach obtido foi de 0,744, um valor próximo ao recomendado, sendo considerado adequado para a análise. No entanto, foi observado um aumento no alfa de Cronbach ao excluir os itens 21, 27, 28 e 30, o que sugere que esses itens /podem estar impactando negativamente a consistência interna da prova.


Os índices de dificuldade do teste estão distribuídos conforme ilustrado na figura \ref{fig:hist_difi}. O item mais fácil é o 21, com 91,9\% de acertos, enquanto o item mais difícil é o 30, com apenas 12,5\% de acertos. Além disso, o item mais difícil apresenta um valor bisserial abaixo do recomendado. A média do índice de dificuldade (DIFI) foi de 50,6\%, com um desvio padrão de 23,4\%.

\begin{figure}[H]
	\centering
	\includegraphics[width=16cm]{hist_difi.png}
	\caption{Distribuição da dificuldade clássica dos itens.}
	\label{fig:hist_difi}
\end{figure}


\section{Análise TRI}

Para avaliar a adequação dos modelos unidimensionais de TRI aos dados do simulado, foram testados os três modelos distintos: o modelo de um parâmetro logístico (1PL), o modelo de dois parâmetros logísticos (2PL) e o modelo de três parâmetros logísticos (3PL). A escolha do modelo mais apropriado é necessária para garantir a precisão das estimativas das habilidades latentes dos respondentes e a qualidade dos itens do teste.

Para a análise, primeiro foram comparados os modelos usando critérios de informação (AIC, SABIC, HQ, BIC) para avaliar a parcimônia e o ajuste relativo de cada modelo. Em seguida, o teste de adequação M2 foi aplicado para verificar se o modelo selecionado se ajusta bem aos dados observados, complementado por outros índices de ajuste, como RMSEA, TLI, e CFI.

A seguir, são apresentados os resultados das comparações entre os modelos, seguido pela avaliação do ajuste absoluto através do teste M2 e índices complementares. Essas análises conduzem à seleção do modelo mais adequado para interpretação e uso posterior nas análises do teste.


\begin{table}[!htb]
	\IBGEtab{%
		\caption{Comparação dos modelos de TRI.}
		\label{tabela-anova}
	}{%
		\begin{tabular}{ccccccccc}
			\toprule
			Modelo & AIC & SABIC & HQ & BIC & logLik & $X^2$ & df & p \\ 
			\midrule \midrule
			1PL & 22045.13 & 22086.15 & 22099.17 & 22184.58 & -10991.57 &  &  &  \\ 
			\midrule
			2PL & 21657.12 & 21736.51 & 21761.70 & 21927.02 & -10768.56 & 446.01 & 29 & 0.000 \\ 
			\midrule
			3PL & 21663.83 & 21782.92 & 21820.70 & 22068.67 & -10741.91 & 53.29 & 30 & 0.006 \\ 
			\bottomrule
		\end{tabular}
	}{%
		\fonte{Produzido pelos autores.}
	}
\end{table}

A Tabela \ref{tabela-anova} apresenta uma comparação entre modelos de Teoria de Resposta ao Item de um, dois e três parâmetros (1PL, 2PL e 3PL). O modelo 2PL apresenta o menor valor de AIC (21657.12) e o maior valor de log-verossimilhança ($-10768.56$), sugerindo que ele é o modelo mais adequado para os dados entre os modelos testados. O valor significativo de $X^2$
com $p = 0,000$ para a comparação entre os modelos de 1PL e 2PL indica que o 2PL oferece um ajuste significativamente melhor do que o 1PL. No entanto, a comparação entre o 2PL e o 3PL também apresenta um valor significativo de $X^2$ com $p=0,006$, mas com um aumento marginal nos critérios de informação (SABIC, HQ, BIC), indicando que o 3PL não proporciona uma melhoria substancial em relação ao 2PL. 

%A comparação entre o 1PL e o 2PL (X2 = 446,01, df = 29, p < 0,001) mostra que o modelo %2PL se ajusta significativamente melhor do que o modelo 1PL.

\begin{table}[!htb]
	\IBGEtab{%
		\caption{Teste de adequação dos modelos. }
		\label{tabela-m2}
	}{%
		\begin{tabular}{ccccccccc}
			\toprule
			Modelo & M2 & df & p & RMSEA & RMSEA\_5 & RMSEA\_95 & TLI & CFI \\ 
			\midrule \midrule
			1PL & 1167 & 434 & 0.0000 & 0.0504 & 0.0469 & 0.0539 & 0.81 & 0.81 \\ 
			\midrule
			2PL & 485 & 405 & 0.0036 & 0.0173 & 0.0104 & 0.0228 & 0.98 & 0.98 \\ 
			\midrule
			3PL & 371 & 375 & 0.5502 & 0.0000 & 0.0000 & 0.0134 & 1.00 & 1.00 \\ 
			\bottomrule
		\end{tabular}
	}{%
		\fonte{Produzido pelos autores.}
	}
\end{table}

Embora a análise inicial dos critérios de informação (AIC, SABIC, HQ, BIC) tenha sugerido que o modelo de 2PL seria o mais adequado, uma análise mais aprofundada utilizando a estatística M2, conforme a \ref{tabela-m2} revelou que este modelo rejeita a hipótese nula de ajuste adequado aos dados, com um valor p significativo ($p = 0,0036$). Isso indica que o modelo de 2PL não se ajusta bem aos dados observados, assim como o modelo de 1PL, além de apresentar valores de AIC, SABIC, HQ, e BIC menos favoráveis, também não apresenta um ajuste ideal segundo o teste M2, com p < 0,0001 e índices TLI e CFI de 0,81, abaixo do limiar de 0,90 considerado para bom ajuste.

O modelo de 3PL, porém, demonstra um bom ajuste aos dados, com um p-valor não significativo (p = 0,5502), RMSEA próximo de zero, e altos índices de ajuste (TLI = 1,00, CFI = 1,00). Isso sugere que o modelo de 3PL é o mais apropriado para os dados em questão, fornecendo a melhor representação das relações entre os itens do teste e a habilidade latente dos respondentes. Portanto, as análises subsequentes serão baseadas no modelo de 3PL.


\begin{table}[H]
	\IBGEtab{%
		\caption{Parâmetros do modelo 3PL}
		\label{tabela-coef3}
	}{%
		\begin{tabular*}{.5\textwidth}{@{\extracolsep{\fill}}cccc@{}}
			\toprule
			Item  & Discriminação (a) & Dificuldade (b) & Chute (c) \\ 
			\midrule \midrule
			1 & 1.93 & -0.17 & 0.27 \\ 
			2 & 3.08 & 1.17 & 0.34 \\ 
			3 & 1.47 & 0.30 & 0.14 \\ 
			4 & 1.50 & -1.16 & 0.02 \\ 
			5 & 1.12 & 0.77 & 0.08 \\ 
			6 & 0.60 & 1.23 & 0.04 \\ 
			7 & 0.92 & 0.06 & 0.05 \\ 
			8 & 0.62 & -0.60 & 0.00 \\ 
			9 & 0.98 & 2.00 & 0.12 \\ 
			10 & 1.44 & -0.44 & 0.40 \\ 
			11 & 2.16 & -1.52 & 0.00 \\ 
			12 & 1.34 & -0.70 & 0.00 \\ 
			13 & 1.80 & 0.61 & 0.19 \\ 
			14 & 0.49 & 1.01 & 0.01 \\ 
			15 & 0.54 & 3.63 & 0.01 \\ 
			16 & 1.19 & -0.91 & 0.18 \\ 
			17 & 1.00 & -1.31 & 0.00 \\ 
			18 & 1.20 & -1.30 & 0.01 \\ 
			19 & 1.36 & -1.38 & 0.00 \\ 
			20 & 0.68 & -2.14 & 0.01 \\ 
			21 & 2.46 & -1.74 & 0.00 \\ 
			22 & 0.32 & 2.20 & 0.01 \\ 
			23 & 0.68 & -0.04 & 0.00 \\ 
			24 & 0.73 & -1.91 & 0.00 \\ 
			25 & 1.83 & 1.15 & 0.23 \\ 
			26 & 0.51 & 0.10 & 0.00 \\ 
			27 & -1.16 & -3.45 & 0.11 \\ 
			28 & -0.46 & -2.76 & 0.00 \\ 
			29 & 0.87 & 0.93 & 0.14 \\ 
			30 & 3.21 & 2.22 & 0.10 \\ 
			\bottomrule
		\end{tabular*}
	}{%
		\fonte{Produzido pelos autores.}
	}
\end{table}


