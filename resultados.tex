\chapter{RESULTADOS}

\section{Análise TCT}

Segundo a ótica TCT, que analisa o escore bruto, ou seja, a soma de todos os acertos, o simulado apresentou notas que variam entre 2 e 26 pontos em um total de 30 questões. A média geral foi de 16 pontos, com desvio padrão de 4,6 pontos. A figura \ref{fig:hist_acertos} apresenta distribuição de frequência de acertos no teste. A mediana é de 16, indicando que o número de acertos dos participantes está bastante concentrado em torno dessa faixa. O primeiro quartil para os acertos é de 13 e o terceiro quartil é de 19, o que demonstra que a maioria dos respondentes acertaram entre 13 e 19 itens. Nenhum respondente acertou ou errou todo o teste.

\begin{figure}[H]
	\centering
	\caption{Distribuição do total de acertos do simulado.}
	\includegraphics[width=16cm]{hist_acertos.png}
	\label{fig:hist_acertos}
	\parbox{\textwidth}{
	\centering % 
	\makebox[16cm][l]{
		\parbox{16cm}{
			\raggedright
			\small \textbf{Fonte}: Elaborado pelos autores.
		}
	}
}
\end{figure}

\begin{table}[H]
	\centering
		\caption{Índices TCT.}
		\label{tabela-tct}
		\begin{tabular*}{\textwidth}{@{\extracolsep{\fill}}clccccc@{}}
			\toprule
			\textbf{Item} & \textbf{Origem} &
			 \makecell{\textbf{\% Erro} } & 
			 \makecell{\textbf{\% Acerto}\\}&
			  \makecell{\textbf{Discriminação} \\ \textbf{($D_i$)}} & \makecell{\textbf{Ponto} \\ \textbf{Bisserial}} & \makecell{\textbf{Cronbach} \\ \textbf{Excluindo item}} \\ 
\hline \textbf{1 }& FAAP & 32,2\% & 67,8\% & 0,552 & 0,468 & 0,730 \\ 
\hline \textbf{2 }& PUC & 56,0\% & 44,0\% & 0,394 & 0,319 & 0,741 \\ 
\hline \textbf{3 }& FGV-RJ & 49,4\% & 50,6\% & 0,551 & 0,468 & 0,730 \\ 
\hline \textbf{4 }& UEA & 21,5\% & 78,5\% & 0,464 & 0,486 & 0,729 \\ 
\hline \textbf{5 }& UFPR & 61,0\% & 39,0\% & 0,507 & 0,415 & 0,734 \\ 
\hline \textbf{6 }& UNICENTRO & 63,7\% & 36,3\% & 0,367 & 0,317 & 0,740 \\
\hline \textbf{7 }& ENEM & 48,6\% & 51,4\% & 0,523 & 0,431 & 0,733 \\ 
\hline \textbf{8 }& UFMS & 41,4\% & 58,6\% & 0,402 & 0,358 & 0,738 \\ 
\hline \textbf{9 }& UEA & 73,9\% & 26,1\% & 0,307 & \textbf{0,290} & 0,741 \\ 
\hline \textbf{10} & EMESCAM & 23,0\% & 77,0\% & 0,390 & 0,402 & 0,734 \\ 
\hline \textbf{11} & UFMS & 11,9\% & 88,1\% & 0,366 & 0,488 & 0,731 \\ 
\hline \textbf{12} & PUC-RIO & 32,8\% & 67,2\% & 0,564 & 0,507 & 0,727 \\ 
\hline \textbf{13} & ENEM-Digital & 54,1\% & 45,9\% & 0,541 & 0,449 & 0,731 \\ 
\hline \textbf{14} & IFPR & 61,0\% & 39,0\% & 0,358 & \textbf{0,299} & 0,742 \\ 
\hline \textbf{15} & ESPCEX & 86,1\% & 13,9\% & 0,155 & \textbf{0,205} & 0,744 \\ 
\hline \textbf{16} & UNIFESO & 24,4\% & 75,6\% & 0,441 & 0,431 & 0,733 \\ 
\hline \textbf{17} & FASM & 25,0\% & 75,0\% & 0,399 & 0,399 & 0,735 \\ 
\hline \textbf{18} & ENEM & 22,4\% & 77,6\% & 0,423 & 0,439 & 0,732 \\ 
\hline \textbf{19} & UFSCAR & 19,4\% & 80,6\% & 0,391 & 0,431 & 0,733 \\ 
\hline \textbf{20} & ENEM & 20,8\% & 79,2\% & 0,297 & 0,306 & 0,740 \\ 
\hline \textbf{21} & UFSCAR & 8,1\% & 91,9\% & 0,242 & 0,450 & 0,734 \\ 
\hline \textbf{22} & UNICENTRO & 65,7\% & 34,3\% & 0,241 & \textbf{0,235} & 0,745 \\ 
\hline \textbf{23} & UNICENTRO & 49,2\% & 50,8\% & 0,429 & 0,357 & 0,738 \\ 
\hline \textbf{24} & UFAL & 22,1\% & 77,9\% & 0,307 & 0,336 & 0,738 \\ 
\hline \textbf{25} & ENEM & 61,1\% & 38,9\% & 0,399 & 0,350 & 0,738 \\ 
\hline \textbf{26} & FAMECA & 51,1\% & 48,9\% & 0,390 & 0,305 & 0,742 \\ 
\hline \textbf{27} & UFMS & 85,8\% & 14,2\% & 0,000 & \textbf{0,006} & 0,753 \\ 
\hline \textbf{28} & UPE & 77,0\% & 23,0\% & -0,068 & \textbf{-0,061} & 0,760 \\ 
\hline \textbf{29} & FAMECA & 57,5\% & 42,5\% & 0,389 & 0,324 & 0,740 \\ 
\hline \textbf{30} & UNCISAL & 87,5\% & 12,5\% & 0,078 & \textbf{0,119} & 0,747 \\
		\hline  \textbf{Total} &&&&&& 0,744 \\
			\bottomrule
		\end{tabular*}\\
		\vspace*{0.5cm}
		\small{\textbf{Fonte:} Produzido pelos autores.}
\end{table}


A Tabela \ref{tabela-tct} apresenta os principais índices TCT para os itens. O alfa de Cronbach obtido foi de 0,744, um valor próximo ao recomendado, sendo considerado adequado para a análise. No entanto, foi observado um aumento no alfa de Cronbach ao excluir os itens 21, 22, 27, 28 e 30, o que sugere que esses itens podem estar impactando negativamente a consistência interna do instrumento.

Inicialmente, os itens 9, 14, 15, 22, 27, 28 e 30 possuem ponto bisserial abaixo do recomendado de 0,30, o que indica que os itens precisam ser revisados, especialmente o item 28, que possui tanto o ponto bisserial quanto o índice de discriminação clássica (D$_i$) negativos. Em relação ao índice discriminação clássica, os itens foram classificados de acordo com a tabela \ref{class-disc-classica}, os resultados estão apresentados na tabela \ref{class-disc-classica2}.

\begin{table}[!htb]
	\centering
	\caption{Classificação do item de acordo com a discriminação clássica.}
	\label{class-disc-classica2}
	\begin{tabular}{lc}
		\hline
		\textbf{Classificação}  & \textbf{Itens}  \\ 
		\hline
	    Item bom  & 1, 3, 4, 5, 7, 8, 12, 13, 16, 18 e 23.  \\ 
		\hline
		Item bom, mas sujeito a aprimoramento & 2, 6, 9, 10, 11, 14, 17, 19, 23, 24, 25, 26 e 29\\ 
		\hline
		Item marginal, sujeito a reelaboração & 20, 21 e 22\\ 
		\hline
		Item deficiente, que deve ser rejeitado &  15, 27, 28 e 30\\ 
		\hline
	\end{tabular}\\
	\vspace*{0.5cm}
	\small{\textbf{Fonte:} \citeonline{rabelo2013}, p.136}
\end{table}

Os índices de dificuldade do teste estão distribuídos conforme ilustrado na figura \ref{fig:hist_difi}. O item mais fácil é o 21, com 8,1\% de erros, enquanto o item mais difícil é o 30, com 87,5\% de respostas erradas. A média do índice de dificuldade foi de 46,5\%, com um desvio padrão de 23,4\%.

\begin{figure}[H]
	\centering
	\caption{Distribuição da dificuldade clássica dos itens.}
	\includegraphics[width=16cm]{../dificuldade_tct.png}
	\parbox{\textwidth}{
		\centering % 
		\makebox[16cm][l]{
			\parbox{16cm}{
				\raggedright
				\small \textbf{Fonte}: Produzido pelos autores.
			}
		}
	}
	\label{fig:hist_difi}
\end{figure}


Observa-se pela figura \ref{fig:hist_difi} que a distribuição dos índices de dificuldade dos itens não segue uma distribuição normal, que sugere que uma prova deve ter uma maior concentração de itens de dificuldade média, complementada por uma menor quantidade de itens fáceis e difíceis, de forma a se aproximar de uma distribuição normal. Segundo essa recomendação, o ideal seria que a maioria dos itens tivesse uma dificuldade intermediária. A tabela \ref{tabela-dificuldade-obtida} mostra uma comparação entre o recomendado na tabela \ref{tabela-class-ID} e o resultado objetivo, a tabela indica diferença entre o esperados e obtido nas faixas II e III, indicando que faltam itens na faixa III.

\begin{table}[H]
	\centering
		\caption{Distribuição ideal dos itens por ID.}
		\label{tabela-dificuldade-obtida}
		\begin{tabular}{lcccc}
			\hline
				\textbf{Faixa} & \textbf{Total Itens} &\textbf{ Distribuição Esperada} & \textbf{Distribuição Obtida}  & \textbf{Itens} \\ 
			\hline
			 \textbf{I} & 3 & 10\% &  10,0\% & 11, 19 e 21\\ 
			\hline
			\textbf{II} & 9 & 20\% &  30,0\% & 1, 4, 10, 12, 16, 17, 18, 20 e 24 \\
			\hline
			\textbf{III} & 8 & 40\% & 26,7\% & 2, 3, 7, 8, 13, 23, 26 e 29 \\ 
			\hline
			 \textbf{IV}& 7 & 20\% & 23,3\%  & 5, 6, 9, 14, 22, 25 e 28\\ 
			\hline
			 \textbf{V} & 3 & 10\% & 10,0\% & 15, 27 e 30\\ 
			\hline
		\end{tabular}\\
		\vspace*{0.5cm}
		\small{\textbf{Fonte:} Produzido pelos autores.}
\end{table}



\section{Análise TRI}

\subsection{Avaliação do Modelo}

Para avaliar o modelo que melhor explica a variação do conjunto de dados do simulado, foram testados os três modelos distintos: o modelo de um parâmetro logístico (1PL), o modelo de dois parâmetros logísticos (2PL) e o modelo de três parâmetros logísticos (3PL). O modelo bidimensional também foi testado, porém não obteve convergência dos estimadores. 

\begin{table}[!htb]
	\centering
		\caption{Teste Razão de verossimilhança.}
		\label{tabela-anova}
		\begin{tabular}{lcccc}
			\hline
			\textbf{Modelo} &  \textbf{ log-verossimilhança }& $\boldsymbol{\chi^2}$ & \textbf{df} &\textbf{ p-valor }\\ 
			\hline
			\textbf{1PL} \textbf{(1,b,0)} &  -10991,57 &  &  &  \\ 
			\hline
			\textbf{2PL} \textbf{(a,b,0)} & -10768,56 & 446,01 & 29 & 0,000 \\ 
			\hline
			\textbf{3PL} \textbf{(a,b,c)} & -10741,91 & 53,29 & 30 & 0,006 \\ 
			\hline
		\end{tabular}\\
		\vspace*{0.5cm}
		\small{\textbf{Fonte:} Produzido pelos autores.}
\end{table}



Os resultados do teste de razão de verossimilhança, apresentados na \ref{tabela-anova}, indicam que a inclusão de parâmetros adicionais melhora significativamente o ajuste do modelo. O modelo de 2PL mostrou uma melhoria significativa em relação ao modelo de 1PL ($p < 0,001$), e o modelo de 3PL também apresentou um ajuste superior ao 2PL ($p = 0,006$). Esses resultados sugerem que, entre os modelos testados, o 3PL é o mais adequado para representar os dados do simulado, capturando de forma mais precisa as variáveis latentes relacionadas ao desempenho dos respondentes.

\begin{table}[!htb]
	 \centering
		\caption{Teste de adequação dos modelos. }
		\label{tabela-m2}
		\begin{tabular}{lcccccccc}
			\hline
			\textbf{Modelo} & \textbf{M}$_\textbf{2}$ & \textbf{df} &\textbf{ p-valor} & \textbf{RMSEA} & \textbf{RMSEA$_\textbf{5}$} & \textbf{RMSEA$_{\textbf{95}}$} & \textbf{TLI} & \textbf{CFI} \\ 
			\hline 
		\textbf{1PL} & 1167 & 434 & 0,0000 & 0,0504 & 0,0469 & 0,0539 & 0,81 & 0,81 \\ 
		\hline
		\textbf{2PL} & 485 & 405 & 0,0036 & 0,0173 & 0,0104 & 0,0228 & 0,98 & 0,98 \\ 
		\hline
		\textbf{3PL} & 371 & 375 & 0,5502 & 0,0000 & 0,0000 & 0,0134 & 1,00 & 1,00 \\ 
			\hline
		\end{tabular}\\
		\vspace*{0.5cm}
		\small{\textbf{Fonte:} Produzido pelos autores.}
\end{table}


Na tabela \ref{tabela-m2} são apresentados os resultados do teste de adequação dos modelos. O teste M$_2$ mostra que o modelo de 1PL não se ajusta bem, com um valor de p < 0,001 indicando rejeição da hipótese nula de bom ajuste e índices de TLI e CFI de 0,81, abaixo do valor de referência de 0,90. O modelo de 2PL apresenta valores altos para TLI e CFI (0,98), porém, também não passa no teste de adequação,com um p-valor de 0,0036, sugerindo que ele não se ajusta bem aos dados.

O modelo de 3PL demonstra o melhor ajuste entre os três modelos testados. Com um p-valor de 0,5502, que não rejeita a hipótese nula de bom ajuste, os índices TLI e CFI perfeitos (1,00), além de um RMSEA próximo de zero. Esses resultados indicam que o modelo de 3PL tem a melhor representação das relações entre os itens do teste e a habilidade latente dos respondentes.


\subsection{Modelo de 3 Parâmetros}

Os resultados dos parâmetros estimados para o modelo 3PL estão detalhados na tabela \ref{tabela-coef3}.

\begin{table}[!htb]
	\centering
		\caption{Parâmetros do modelo 3PL}
		\label{tabela-coef3}

		\begin{tabular*}{.9\textwidth}{@{\extracolsep{\fill}}clccc@{}}
			\toprule
			\textbf{Item}  & \textbf{Origem} & 
		   \makecell{\textbf{Discriminação} \\\textbf{(a)}}& 
			\makecell{\textbf{Dificuldade} \\ \textbf{(b)}} &
			 \makecell{\textbf{Acerto Casual} \\\textbf{(c)}} 
			   \\ 
		\hline \textbf{1 }& FAAP & 1,94 & -0,17 & 0,27 \\ 
		\hline \textbf{2 }& PUC & 3,08 & 1,17 & 0,34 \\ 
		\hline \textbf{3 }& FGV-RJ & 1,47 & 0,30 & 0,14  \\ 
		\hline \textbf{4 }& UEA & 1,50 & -1,16 & 0,02  \\ 
		\hline \textbf{5 }& UFPR & 1,12 & 0,77 & 0,08 \\ 
		\hline \textbf{6 }& UNICENTRO & 0,60 & 1,23 & 0,04  \\ 
		\hline \textbf{7 }& ENEM & 0,92 & 0,06 & 0,04 \\ 
		\hline \textbf{8 }& UFMS & 0,62 & -0,60 & 0,00 \\ 
		\hline \textbf{9 }& UEA & 0,98 & 2,00 & 0,12  \\ 
		\hline \textbf{10} & UMESCAM & 1,44 & -0,44 & 0,40 \\ 
		\hline \textbf{11} & UFMS & 2,16 & -1,52 & 0,00  \\ 
		\hline \textbf{12} & PUC-RIO & 1,34 & -0,70 & 0,00 \\ 
		\hline \textbf{13} & ENEM-Digital & 1,80 & 0,61 & 0,19 \\ 
		\hline \textbf{14} & IFPR & 0,48 & 1,01 & 0,01  \\ 
		\hline \textbf{15} & ESPECEX & 0,54 & 3,63 & 0,01 \\ 
		\hline \textbf{16} & UNIFESO & 1,19 & -0,91 & 0,18  \\ 
		\hline \textbf{17} & FASM & 1,00 & -1,31 & 0,00  \\ 
		\hline \textbf{18} & ENEM & 1,20 & -1,29 & 0,01 \\ 
		\hline \textbf{19} & UFSCAR & 1,36 & -1,38 & 0,00 \\ 
		\hline \textbf{20} & ENEM & 0,68 & -2,14 & 0,01 \\ 
		\hline \textbf{21} & UFSCAR & 2,46 & -1,74 & 0,00  \\ 
		\hline \textbf{22} & UNICENTRO & 0,32 & 2,20 & 0,01 \\ 
		\hline \textbf{23} & UNICENTRO & 0,68 & -0,04 & 0,00  \\ 
		\hline \textbf{24} & UFAL & 0,73 & -1,91 & 0,00 \\ 
		\hline \textbf{25} & ENEM & 1,83 & 1,15 & 0,23  \\ 
		\hline \textbf{26} & FAMECA & 0,51 & 0,10 & 0,00 \\ 
		\hline \textbf{27} & UFMS & \textbf{-1,16} & -3,45 & 0,11  \\ 
		\hline \textbf{28} & UPE & \textbf{-0,46} & -2,76 & 0,00  \\ 
		\hline \textbf{29} & FAMECA & 0,87 & 0,93 & 0,14  \\ 
		\hline \textbf{30} & UNCISAL & 3,21 & 2,22 & 0,10 \\ 
			\bottomrule
		\end{tabular*}\\
		\vspace*{0.5cm}
		\small{\textbf{Fonte:} Produzido pelos autores.}
\end{table}
\clearpage
\begin{figure}[H]
	\centering
	\caption{Curva característica dos itens.}
	\includegraphics[width=16cm]{../TCCfigura01.png}
	\parbox{\textwidth}{
		\centering % 
		\makebox[16cm][l]{
			\parbox{16cm}{
				\raggedright
				\small \textbf{Fonte}: Produzido pelos autores.
			}
		}
	}
	\label{fig:curva_itens}
\end{figure}

A figura \ref{fig:curva_itens} apresenta as curvas características dos itens com os parâmetros listados na Tabela \ref{tabela-coef3}. Observa-se que os itens 27 e 28 possuem inclinações contrárias, indicativas de valores de discriminação negativos, o que é problemático em avaliações. Esses itens sugerem que participantes com maior habilidade têm menor probabilidade de acertá-los, o que não é esperado em um teste bem construído. Conforme apontado por \citeonline{baker2001}, itens com discriminação negativa indicam que há algum problema no item, seja por estar mal formulado ou por gerar desinformação entre os alunos de maior capacidade. \citeonline{ayala2013theory} reforça que um valor negativo de discriminação é um indicativo de que o item deve ser descartado, uma vez que seu comportamento é inconsistente com o modelo. Portanto, a presença de discriminação negativa nesses itens requer atenção imediata, sendo recomendada a revisão ou a exclusão, já que tais inconsistências comprometem a validade da avaliação e dificultam a mensuração precisa da habilidade dos participantes.

\begin{comment}
	
\subsubsection{Análise dos itens com discriminação negativa}

Itens com discriminação negativa, como os analisados, não conseguem medir adequadamente a habilidade e podem gerar resultados incoerentes. Portanto, é recomendável revisá-los ou descartá-los para garantir a validade da avaliação.


\begin{figure}[!htb]
	\centering
	\caption{Curva característica dos itens com discriminação negativa.}
	\includegraphics[width=16cm]{../itens_disc_negativa.png}
	\parbox{\textwidth}{
		\centering % 
		\makebox[16cm][l]{
			\parbox{16cm}{
				\raggedright
				\small \textbf{Fonte}: Produzido pelos autores.
			}
		}
	}
	\label{fig:itens_disc_negativa}
\end{figure}


Conforme mostrado na Figura \ref{fig:itens_disc_negativa}, a probabilidade de acerto dos participantes diminui à medida que a habilidade aumenta, ou seja, funciona de maneira contra-intuitiva. No contexto da TRI, o parâmetro b representa o nível de habilidade necessário para que um participante tenha $(c+1)/2$ de probabilidade de acertar o item. Entretanto, a presença de discriminação negativa faz com que o item funcione de maneira inversa: participantes com maior habilidade têm menor probabilidade de acerto.
\end{comment}

As figuras \ref{fig:item_27} e \ref{fig:item_28} exibem a proporção de respostas marcadas para cada alternativa em função do total de acertos, com os resultados agrupados em intervalos de 5 acertos.

Ao analisar o gráfico do item 27, observa-se que respondentes com maior número de acertos tenderam a marcar a alternativa D. Apesar de a alternativa correta ser a alternativa A (na Sociologia, a família e a escola não são vistas como instituições opressoras), a letra da música parece sugerir que o narrador se torna um "reprodutor do sistema opressor", levando a alternativa D ser uma interpretação possível se o aluno considerar o conteúdo do texto auxiliar.

Segundo \citeonline{rabelo2013} os itens de múltipla escolha dividem-se em três partes: o texto-base, o enunciado e as alternativas, cada uma dessas partes deve estar bem relacionada às outras, ou seja, devem manter coerência entre si. No caso do item 27, isso não ocorre, pois a alternativa A não se relaciona diretamente com o texto-base, prejudicando a clareza do item e confundindo o objetivo da questão. Esse texto-base gera uma ambiguidade que desvia o foco da análise sociológica apropriada, levando o aluno a acreditar que a alternativa D é a correta. 

%Por fim, a questão abre espaço para múltiplas interpretações, o que pode ser visto como uma falha em um contexto de avaliação, onde se espera que haja uma resposta claramente correta.
	
No caso do item 28, pelo gráfico da figura \ref{fig:item_28}, observa-se que conforme o número de acertos aumenta a alternativa mais marcada é a letra C. Ao analisar o item, verificou-se que ele foi corrigido com o gabarito incorreto, considerando a letra A, quando a alternativa correta deveria ser a letra C. 

\begin{figure}[H]
	\centering
	\caption{Proporção de alternativas marcadas total de acertos do item 27.}
	\includegraphics[width=15cm]{../alternativas2_item27.png}
	\parbox{\textwidth}{
		\centering % 
		\makebox[15cm][l]{
			\parbox{15cm}{
				\raggedright
				\small \textbf{Fonte}: Produzido pelos autores.
			}
		}
	}
	\label{fig:item_27}
\end{figure}


\begin{figure}[H]
	\centering
	\caption{Proporção de alternativas marcadas total de acertos do item 28.}
	\includegraphics[width=15cm]{../alternativas2_item28.png}
	\parbox{\textwidth}{
	\centering % 
	\makebox[15cm][l]{
		\parbox{15cm}{
			\raggedright
			\small \textbf{Fonte}: Produzido pelos autores.
		}
	}
}	
	\label{fig:item_28}
\end{figure}



%%%%%%%%%%%%%%%%%%%%%%%%%%%%%%%%%%%%%%%%%%%%%%%%%%%%%%%%%%%%%%%%%%%%%%%%%%%%%%%

\subsubsection{Modelo de 3 Parâmetros - 2º Ajuste}

 Com base nessas análises, o modelo foi ajustado novamente, removendo o item 27 e corrigindo o item 28. Para a avaliação do modelo o resultados do teste M$_2$ indicam um bom ajuste do modelo. O valor de M$_2$ com p-valor = $0,54$ sugere que o modelo não é significativamente diferente dos dados. O RMSEA é 0, indicando um ajuste excelente. Os índices de ajuste incremental TLI (1) e CFI (1), reforçam que o modelo se ajustou bem aos dados.
 
\begin{table}[!htb]
	\centering
	\caption{Teste de adequação do 2ª Ajuste.}
	\label{tabela-m2-2}
	\begin{tabular}{lcccccccc}
		\hline
		\textbf{Modelo} & \textbf{M}$_\textbf{2}$ & \textbf{df} &\textbf{ p-valor} & \textbf{RMSEA} & \textbf{RMSEA$_\textbf{5}$} & \textbf{RMSEA$_{\textbf{95}}$} & \textbf{TLI} & \textbf{CFI} \\ 
		\hline 
		\textbf{3PL - 2º Ajuste} & 344 & 348 & 0,542 & 0,0000 & 0,0000 & 0,01377 & 1,00 & 1,00 \\ 
		\hline
	\end{tabular}\\
	\vspace*{0.5cm}
	\small{\textbf{Fonte:} Produzido pelos autores.}
\end{table}

\begin{table}[h]
	\centering
	\caption{Parâmetros do modelo 3PL - 2ª Ajuste}
	\label{tabela-coef3-excl}
	\begin{tabular}{clcccc}
		\hline
		\textbf{Item}  & \textbf{Origem} & 
		\makecell{\textbf{Discriminação} \\\textbf{(a)}}& 
		\makecell{\textbf{Dificuldade} \\\textbf{ (b)}} &
		\makecell{\textbf{Acerto Casual} \\\textbf{(c})} &
		\makecell{\textbf{Máxima} \\ \textbf{Informação}} 
		\\ 
		\hline \textbf{22} & UNICENTRO & 0,32 & 2,12 & 0,01 & 0,03 \\ 
		\hline \textbf{14} & IFPR & 0,49 & 0,98 & 0,00 & 0,06 \\ 
		\hline \textbf{15} & ESPECEX & 0,53 & 3,68 & 0,00 & 0,06 \\ 
		\hline \textbf{26} & FAMECA & 0,51 & 0,10 & 0,00 & 0,07 \\ 
		\hline \textbf{6 }& UNICENTRO & 0,59 & 1,23 & 0,04 & 0,08 \\ 
		\hline \textbf{8 }& UFMS & 0,62 & -0,60 & 0,00 & 0,10 \\ 
		\hline \textbf{23} & UNICENTRO & 0,67 & -0,04 & 0,00 & 0,11 \\ 
		\hline \textbf{20} & ENEM & 0,68 & -2,16 & 0,00 & 0,11 \\ 
		\hline \textbf{24} & UFAL & 0,72 & -1,92 & 0,00 & 0,13 \\ 
		\hline \textbf{29} & FAMECA & 0,90 & 0,93 & 0,14 & 0,15 \\ 
		\hline \textbf{28} & UPE & 0,86 & -1,06 & 0,00 & 0,18 \\ 
		\hline \textbf{7 }& ENEM & 0,96 & 0,14 & 0,07 & 0,20 \\ 
		\hline \textbf{9 }& UEA & 1,02 & 2,01 & 0,13 & 0,20 \\ 
		\hline \textbf{10} & UMESCAM & 1,55 & -0,31 & 0,44 & 0,25 \\ 
		\hline \textbf{17} & FASM & 1,00 & -1,31 & 0,00 & 0,25 \\ 
		\hline \textbf{16} & UNIFESO & 1,25 & -0,82 & 0,21 & 0,26 \\ 
		\hline \textbf{5 }& UFPR & 1,11 & 0,76 & 0,08 & 0,26 \\ 
		\hline \textbf{18} & ENEM & 1,18 & -1,31 & 0,00 & 0,35 \\ 
		\hline \textbf{3 }& FGV-RJ & 1,49 & 0,32 & 0,15 & 0,42 \\ 
		\hline \textbf{12} & PUC-RIO & 1,33 & -0,70 & 0,00 & 0,44 \\ 
		\hline \textbf{19} & UFSCAR & 1,37 & -1,37 & 0,00 & 0,47 \\ 
		\hline \textbf{25} & ENEM & 1,83 & 1,15 & 0,23 & 0,54 \\ 
		\hline \textbf{1 }& FAAP & 1,88 & -0,22 & 0,25 & 0,54 \\ 
		\hline \textbf{13} & ENEM-Digital & 1,77 & 0,59 & 0,18 & 0,55 \\ 
		\hline \textbf{4 }& UEA & 1,50 & -1,18 & 0,00 & 0,56 \\ 
		\hline \textbf{11} & UFMS & 2,13 & -1,53 & 0,00 & 1,13 \\ 
		\hline \textbf{2 }& PUC & 2,96 & 1,18 & 0,34 & 1,14 \\ 
		\hline \textbf{21} & UFSCAR & 2,50 & -1,73 & 0,00 & 1,56 \\ 
		\hline \textbf{30} & UNCISAL & 3,41 & 2,18 & 0,10 & 2,37 \\
		\hline
	\end{tabular}\\
	\vspace*{0.5cm}
	\small{\textbf{Fonte:} Produzido pelos autores.}
\end{table}

A tabela \ref{tabela-coef3-excl} apresenta os resultados dos parâmetros para o 2ª ajuste, além disso, foi acrescentado o ponto máximo de informação para cada item, que é o ponto máximo da curva de informação do item  dada pela equação \ref{eq:info_item}. Em primeira análise observa-se mudança nos parâmetros do item 28, onde tal item passou a ter discriminação positiva. 

Os itens 22, 14, 15, 26, 6, 8, 23, 20, 24, 29, 28, 7 são os menos informativos e  possuem o valor da discriminação menor que 1, pela equação \ref{eq:info_item}, valores menores que 1 diminuem a informação do item. Esses itens somam 10,18\% da informação do teste, ou seja, ao remover 12 dos 29 itens, ainda teríamos  89.82\% da informação do instrumento. Com o gráfico da figura \ref{fig:info_itens} podemos ver como as curvas de informação dos itens citados anteriormente são baixas, informando nada ou quase nada sobre a variável latente de interesse.

Conforme enfatizado por \citeonline{baker2001}, itens com baixa discriminação ou com valores muito baixos de informação máxima (ou seja, que não contribuem significativamente para a mensuração da habilidade) devem ser considerados para revisão ou exclusão, pois não agregam valor à avaliação.

\clearpage
\begin{figure}[H]
	\centering
	\caption{Curva de informação dos itens.}
	\includegraphics[width=16cm]{../info_itens.png}
	\parbox{\textwidth}{
		\centering % 
		\makebox[16cm][l]{
			\parbox{16cm}{
				\raggedright
				\small \textbf{Fonte}: Produzido pelos autores.
			}
		}
	}
	\label{fig:info_itens}
\end{figure}

Na análise pela TRI, a dificuldade dos itens ($b$) variou entre -2,15 e 3,67, com o item 15 sendo identificado como o mais difícil e o item 20 como o mais fácil. Diferentemente da TCT, onde a dificuldade varia de 0 a 1, a escala de dificuldade na TRI permite valores de $- \infty$ a $+ \infty$. A Figura \ref{fig:dificuldade_tri} ilustra a distribuição dos valores de dificuldade ($b$) conforme apresentado na Tabela \ref{tabela-coef3-excl}.

\begin{figure}[H]
	\centering
	\caption{Histograma da distribuição da dificuldade pela TRI.}
	\includegraphics[width=15cm]{../hist_dificuldade_tri.png}
	\parbox{\textwidth}{
		\centering % 
		\makebox[15cm][l]{
			\parbox{15cm}{
				\raggedright
				\small \textbf{Fonte}: Produzido pelos autores.
			}
		}
	}	
	\label{fig:dificuldade_tri}
\end{figure}


\subsection{Informação do Teste}

A figura \ref{fig:info} ilustra a curva de informação do teste do segundo ajuste, conforme a equação \ref{eq:info_teste}, a linha pontilhada representa o erro padrão e os pontos representam os itens com o parâmetro de dificuldade posicionado na escala de habilidade. A curva de informação do teste atinge seu pico em 5,71 quando $\theta = -1,46$. A região de $\theta$ com maior informação ($I(\theta) > 5$) é entre as habilidades -1,9 e -0,7, o que indica que o teste é mais preciso para indivíduos com habilidade abaixo da média. 

A maior concentração de informação ocorre no intervalo de aproximadamente -2,5 a 2,5, o que significa que o teste fornece maior precisão para respondentes cujas habilidades estão dentro dessa faixa. Fora desse intervalo, à medida que $\theta$ se afasta em direção a valores muito baixos ou muito altos, a quantidade de informação diminui consideravelmente, o que implica em menor precisão na estimativa de habilidade para esses extremos. Portanto, o teste se mostra eficaz para diferenciar participantes com habilidades intermediárias, mas perde precisão para aqueles com habilidades muito baixas ou muito altas.

\begin{figure}[H]
	\centering
	\caption{Curva de informação e erro padrão do teste.}
	\includegraphics[width=16cm]{../info_modelo2.png}
	\parbox{\textwidth}{
		\centering % 
		\makebox[16cm][l]{
			\parbox{16cm}{
				\raggedright
				\small \textbf{Fonte}: Produzido pelos autores.
			}
		}
	}
	\label{fig:info}
\end{figure}


\subsection{Estimativa das habilidades}

Usando o modelo 3PL excluindo o item 27 da análise, as habilidades foram estimadas, conforme os resultados apresentados na tabela \ref{summary-habilidade} e na figura \ref{fig:info_habilidade}.


\begin{table}[H]		
	\centering
		\caption{Distribuição da Habilidade estimada e total de acertos}
		\label{summary-habilidade}
		\begin{tabular}{lcccccc}
			\hline
			& \textbf{Mínimo} & $\textbf{Q}_1$ & \textbf{Mediana} & \textbf{Média} & $\textbf{Q}_3$ & \textbf{Máximo} \\ 
			\hline
			$\boldsymbol{\hat{\theta}}$ & -2,79 & -0,56 & 0,00 & 0,00 & 0,62 & 2,30 \\ 
			\hline
		\end{tabular}\\
		\vspace*{0.5cm}
		\small{\textbf{Fonte:} Produzido pelos autores.}
\end{table}


\begin{figure}[H]
	\centering
	\caption{Distribuição da habilidade e curva de informação do teste.}
	\includegraphics[width=16cm]{../habilidade_info.png}
		\parbox{\textwidth}{
		\centering % 
		\makebox[16cm][l]{
			\parbox{16cm}{
				\raggedright
				\small \textbf{Fonte}: Produzido pelos autores.
			}
		}
	}
	\label{fig:info_habilidade}
\end{figure}

Podemos observar pela figura \ref{fig:info_habilidade} que a distribuição das habilidades estimadas dos participantes está concentrada majoritariamente (75\%) na região central, entre $\theta = -1$ e $\theta = 1$, o que reflete uma maior frequência de respondentes com habilidades intermediárias. No entanto, a curva de informação do teste atinge seu pico em valores de $\theta$ um pouco mais baixos, sugerindo que o teste tem mais precisão para estimar habilidades abaixo da média.

Este descompasso entre a concentração das habilidades estimadas e a área de maior informação indica uma lacuna no teste. A curva de informação esteja fornecendo melhor precisão para participantes com habilidades mais baixas, há um decréscimo visível na informação na região central do gráfico, onde está a maior parte dos respondentes. 

Para melhorar o teste, seria recomendável a adição de itens que aumentem a informação nessa faixa central. Dessa forma, o teste poderá discriminar melhor entre os participantes que têm habilidades próximas à média, melhorando a precisão das estimativas nessa região. Além disso, adotar itens mais discriminativos ao teste ajudaria a melhorar a capacidade de diferenciar entre níveis de habilidade.

\subsection{Comparação TCT e TRI}

\begin{figure}[!hbt]
	\centering
		\caption{Relação entre o número de acertos e a habilidade estimada pela TRI}
	\includegraphics[width=16cm]{../acertos_habilidade.png}
	\parbox{\textwidth}{
		\centering % 
		\makebox[16cm][l]{
			\parbox{16cm}{
				\raggedright
				\small \textbf{Fonte}: Produzido pelos autores.
			}
		}
	}
	\label{fig:acertos_habilidade}
\end{figure}

O gráfico da figura\ref{fig:acertos_habilidade} mostra a relação entre o número total de acertos e a habilidade estimada pelo modelo TRI. Observa-se a diferença entre as habilidades estimadas dentro do mesmo número de acertos. A tabela \ref{exemplo-10acertos} exemplifica diferentes habilidades estimadas para respondentes com 10 acertos na prova, com o vetor de respostas ordenado do item com menor para o maior valor do parâmetro de dificuldade (b).


\begin{table}[!hbt]
	\centering
		\caption{Vetor de resposta e habilidade estimada para respondentes com 10 acertos}
		\label{exemplo-10acertos}
		\begin{tabular*}{0.7\textwidth}{@{\extracolsep{\fill}}lcc@{}}
			\hline
			 & \textbf{Vetor de Respostas} & $\boldsymbol{\hat{\theta}}$  \\ 
		\hline \textbf{1 } & 00000000010110010110001011010 & -2,18   \\ 
		\hline \textbf{2 } & 10100000011010000010001001110 & -1,60   \\ 
		\hline \textbf{3 } & 01001001100011001001001100000 & -1,57   \\ 
		\hline \textbf{4 } & 11100000010010100010100001100 & -1,53   \\ 
		\hline \textbf{5 } & 00101001100010100001100001100 & -1,24   \\ 
		\hline \textbf{6 } & 10011001110110010000000000100 & -1,19   \\ 
		\hline \textbf{7 } & 11110100100010000010011000000 & -1,19   \\ 
		\hline \textbf{8 } & 11110001100000010001000100100 & -1,13   \\ 
		\hline \textbf{9 } & 10110011000101000000000101100 & -1,12   \\ 
		\hline \textbf{10} & 11101011010000010000100000100 & -1,09  \\ 
		\hline \textbf{11} & 01111100000000001100100001100 & -1,09  \\ 
		\hline \textbf{12} & 10101001100101001100010000000 & -1,08  \\ 
		\hline \textbf{13} & 00101001101010001011010000000 & -1,04  \\ 
		\hline \textbf{14} & 00111100001110000011000100000 & -1,01  \\ 
		\hline \textbf{15} & 10111000101010101000010000000 & -0,92  \\ 
		\hline \textbf{16} & 11110101110000100000100000000 & -0,91  \\ 
		\hline \textbf{17} & 01111100001110110000000000000 & -0,91  \\ 
		\hline \textbf{18} & 11111110110000010000000000000 & -0,86  \\ 
		\hline \textbf{19} & 10111110110000001000000010000 & -0,85  \\ 
		\hline \textbf{20} & 10111101110000100000000010000 & -0,83  \\ 
		\hline \textbf{21} & 10111111010110000000000000000 & -0,75  \\
		\hline
		\end{tabular*}\\
		\vspace*{0.5cm}
		\small{\textbf{Fonte:} Produzido pelos autores.}
\end{table}


A tabela \ref{exemplo-10acertos} demonstra que, com o mesmo número de acertos (10), há uma diferença nas habilidades estimadas. Indivíduos com maior coerência nas resposta, ou seja, aqueles que acertam itens de dificuldade progressiva, recebem uma pontuação maior que  aqueles com menor coerência, ou seja, que acertam itens mais difíceis e erram os mais fáceis.

Em relação aos itens, observou-se que o item 30, foi o único que apresentou problemas tanto na discriminação e na correlação ponto bisserial, porém na TRI apresentou-se como um item que discrimina bem, isso porque na pela TRI, o item consegue discriminar bem indivíduos de alta habilidade (b = 2,18).

Conforme ilustrado nas Figuras  \ref{fig:tct_item30} e \ref{fig:tri_item30}, a TRI indica que o item 30 consegue diferenciar efetivamente respondentes com alta habilidade, enquanto, pela TCT, o item mostrou-se problemático ao considerar apenas o número de acertos. Esses resultados sugerem que, embora o item tenha limitações na análise TCT, ele apresenta um bom desempenho ao avaliar habilidades mais avançadas, conforme os parâmetros da TRI. 

\begin{figure}[H]
	\centering
	\caption{Proporção de alternativas do item 30 por intervalo de número de acertos na TCT.}
	\includegraphics[width=16cm]{../tct_item30.png}
	\parbox{\textwidth}{
		\centering % 
		\makebox[16cm][l]{
			\parbox{16cm}{
				\raggedright
				\small \textbf{Fonte}: Produzido pelos autores.
			}
		}
	}
	\label{fig:tct_item30}
\end{figure}

\begin{figure}[H]
	\centering
	\caption{Proporção de alternativas do item 30 por intervalo de habilidade na TRI.}
	\includegraphics[width=16cm]{../tri_item30.png}
	\parbox{\textwidth}{
		\centering % 
		\makebox[16cm][l]{
			\parbox{16cm}{
				\raggedright
				\small \textbf{Fonte}: Produzido pelos autores.
			}
		}
	}
	\label{fig:tri_item30}
\end{figure}











