\chapter{RESULTADOS}

\section{Análise TCT}

Segundo a ótica TCT, que analisa o escore bruto, ou seja, a soma de todos os acertos, o simulado apresentou notas que variam entre 2 e 26 pontos em um total de 30 questões. A média geral foi de 16 pontos, com desvio padrão de 4,6 pontos. A figura \ref{fig:hist_acertos} a distribuição de frequência de acertos no teste.

\begin{figure}[H]
	\centering
	\includegraphics[width=16cm]{hist_acertos.png}
	\caption{Distribuição do total de acertos do simulado.}
	\label{fig:hist_acertos}
\end{figure}

\begin{table}[H]
	\IBGEtab{%
		\caption{Índices TCT.}
		\label{tabela-tct}
	}{%
		\begin{tabular*}{.8\textwidth}{@{\extracolsep{\fill}}clccccc@{}}
			\toprule
			Item & Origem & \% Erro & \% Acerto & Discriminação & Bisserial & Cronbach \\ 
			\midrule \midrule
			1 & FAAP & 32,2\% & 67,8\% & 0,552 & 0,468 & 0,730 \\ 
			\midrule
			2 & PUC & 56,0\% & 44,0\% & 0,394 & 0,319 & 0,741 \\ 
			\midrule
			3 & FGV-RJ & 49,4\% & 50,6\% & 0,551 & 0,468 & 0,730 \\ 
			\midrule
			4 & UEA & 21,5\% & 78,5\% & 0,464 & 0,486 & 0,729 \\ 
			\midrule
			5 & UFPR & 61,0\% & 39,0\% & 0,507 & 0,415 & 0,734 \\ 
			\midrule
			6 & UNICENTRO & 63,7\% & 36,3\% & 0,367 & 0,317 & 0,740 \\
			\midrule
			7 & ENEM & 48,6\% & 51,4\% & 0,523 & 0,431 & 0,733 \\ 
			\midrule
			8 & UFMS & 41,4\% & 58,6\% & 0,402 & 0,358 & 0,738 \\ 
			\midrule
			9 & UEA & 73,9\% & 26,1\% & 0,307 & 0,290 & 0,741 \\ 
			\midrule
			10 & EMESCAM & 23,0\% & 77,0\% & 0,390 & 0,402 & 0,734 \\ 
			\midrule
			11 & UFMS & 11,9\% & 88,1\% & 0,366 & 0,488 & 0,731 \\ 
			\midrule
			12 & PUC-RIO & 32,8\% & 67,2\% & 0,564 & 0,507 & 0,727 \\ 
			\midrule
			13 & ENEM-Digital & 54,1\% & 45,9\% & 0,541 & 0,449 & 0,731 \\ 
			\midrule
			14 & IFPR & 61,0\% & 39,0\% & 0,358 & 0,299 & 0,742 \\ 
			\midrule
			15 & ESPCEX & 86,1\% & 13,9\% & 0,155 & 0,205 & 0,744 \\ 
			\midrule
			16 & UNIFESO & 24,4\% & 75,6\% & 0,441 & 0,431 & 0,733 \\ 
			\midrule
			17 & FASM & 25,0\% & 75,0\% & 0,399 & 0,399 & 0,735 \\ 
			\midrule
			18 & ENEM & 22,4\% & 77,6\% & 0,423 & 0,439 & 0,732 \\ 
			\midrule
			19 & UFSCAR & 19,4\% & 80,6\% & 0,391 & 0,431 & 0,733 \\ 
			\midrule
			20 & ENEM & 20,8\% & 79,2\% & 0,297 & 0,306 & 0,740 \\ 
			\midrule
			21 & UFSCAR & 8,1\% & 91,9\% & 0,242 & 0,450 & 0,734 \\ 
			\midrule
			22 & UNICENTRO & 65,7\% & 34,3\% & 0,241 & 0,235 & 0,745 \\ 
			\midrule
			23 & UNICENTRO & 49,2\% & 50,8\% & 0,429 & 0,357 & 0,738 \\ 
			\midrule
			24 & UFAL & 22,1\% & 77,9\% & 0,307 & 0,336 & 0,738 \\ 
			\midrule
			25 & ENEM & 61,1\% & 38,9\% & 0,399 & 0,350 & 0,738 \\ 
			\midrule
			26 & FAMECA & 51,1\% & 48,9\% & 0,390 & 0,305 & 0,742 \\ 
			\midrule
			27 & UFMS & 85,8\% & 14,2\% & -0,0004 & 0,006 & 0,753 \\ 
			\midrule
			28 & UPE & 77,0\% & 23,0\% & -0,068 & -0,061 & 0,760 \\ 
			\midrule
			29 & FAMECA & 57,5\% & 42,5\% & 0,389 & 0,324 & 0,740 \\ 
			\midrule
			30 & UNCISAL & 87,5\% & 12,5\% & 0,078 & 0,119 & 0,747 \\ 
			\bottomrule
		\end{tabular*}
	}{%
		\fonte{Produzido pelos autores.}
	}
\end{table}


A Tabela \ref{tabela-tct} apresenta os principais índices da Teoria Clássica dos Testes (TCT) para os itens. Inicialmente, observa-se que os itens 27 e 28 possuem índices de discriminação negativos, indicando que esses itens foram mais acertados por indivíduos de menor habilidade. Além disso, esses itens apresentam coeficientes bisserial muito baixos, sugerindo que ambos necessitam de revisão. Também se nota que esses itens são difíceis, pois têm uma proporção de acertos (DIFI) muito baixa.


O alfa de Cronbach obtido foi de 0,744, um valor próximo ao recomendado, sendo considerado adequado para a análise. No entanto, foi observado um aumento no alfa de Cronbach ao excluir os itens 21, 27, 28 e 30, o que sugere que esses itens /podem estar impactando negativamente a consistência interna da prova.


Os índices de dificuldade do teste estão distribuídos conforme ilustrado na figura \ref{fig:hist_difi}. O item mais fácil é o 21, com 91,9\% de acertos, enquanto o item mais difícil é o 30, com apenas 12,5\% de acertos. Além disso, o item mais difícil apresenta um valor bisserial abaixo do recomendado. A média do índice de dificuldade (DIFI) foi de 50,6\%, com um desvio padrão de 23,4\%.

\begin{figure}[H]
	\centering
	\includegraphics[width=16cm]{hist_difi.png}
	\caption{Distribuição da dificuldade clássica dos itens.}
	\label{fig:hist_difi}
\end{figure}

\newpage
\section{Análise TRI}

\subsection{Avaliação do Modelo}

Para avaliar a adequação dos modelos unidimensionais de TRI aos dados do simulado, foram
testados os três modelos distintos: o modelo de um parâmetro logístico (1PL), o modelo de dois parâmetros logísticos (2PL) e o modelo de três parâmetros logísticos (3PL).

\begin{table}[!htb]
	\IBGEtab{%
		\caption{Teste Razão de verossimilhança.}
		\label{tabela-anova}
	}{%
		\begin{tabular}{cccccc}
			\toprule
			Modelo &   log-verossimilhança & $X^2$ & df & p \\ 
			\midrule \midrule
			1PL &  -10991.57 &  &  &  \\ 
			\midrule
			2PL &  -10768.56 & 446.01 & 29 & 0.000 \\ 
			\midrule
			3PL & -10741.91 & 53.29 & 30 & 0.006 \\ 
			\bottomrule
		\end{tabular}
	}{%
		\fonte{Produzido pelos autores.}
	}
\end{table}



Os resultados do teste de razão de verossimilhança, apresentados na \ref{tabela-anova}, indicam que a inclusão de parâmetros adicionais melhora significativamente o ajuste do modelo. O modelo de 2PL mostrou uma melhoria significativa em relação ao modelo de 1PL ($p < 0,001$), e o modelo de 3PL também apresentou um ajuste superior ao 2PL ($p = 0,006$). Esses resultados sugerem que, entre os modelos testados, o 3PL é o mais adequado para representar os dados do simulado, capturando de forma mais precisa as variáveis latentes relacionadas ao desempenho dos respondentes.

\begin{table}[!htb]
	\IBGEtab{%
		\caption{Teste de adequação dos modelos. }
		\label{tabela-m2}
	}{%
		\begin{tabular}{ccccccccc}
			\toprule
			Modelo & M2 & df & p & RMSEA & RMSEA\_5 & RMSEA\_95 & TLI & CFI \\ 
			\midrule \midrule
			1PL & 1167 & 434 & 0.0000 & 0.0504 & 0.0469 & 0.0539 & 0.81 & 0.81 \\ 
			\midrule
			2PL & 485 & 405 & 0.0036 & 0.0173 & 0.0104 & 0.0228 & 0.98 & 0.98 \\ 
			\midrule
			3PL & 371 & 375 & 0.5502 & 0.0000 & 0.0000 & 0.0134 & 1.00 & 1.00 \\ 
			\bottomrule
		\end{tabular}
	}{%
		\fonte{Produzido pelos autores.}
	}
\end{table}


A \ref{tabela-m2} apresenta os resultados do teste de adequação dos modelos. O teste M2 mostra que o modelo de 1PL não se ajusta bem, com um valor de p < 0,001 indicando rejeição da hipótese nula de bom ajuste e índices de TLI e CFI de 0,81, abaixo do valor de referência de 0,90. O modelo de 2PL apresenta valores altos para TLI e CFI (0,98), porém, também não passa no teste de adequação,com um p-valor de 0,0036, sugerindo que ele não se ajusta bem aos dados.

O modelo de 3PL demonstra o melhor ajuste entre os três modelos testados. Com um p-valor de 0,5502, que não rejeita a hipótese nula de bom ajuste, os índices TLI e CFI perfeitos (1,00), além de um RMSEA próximo de zero. Esses resultados indicam que o modelo de 3PL tem a melhor representação das relações entre os itens do teste e a habilidade latente dos respondentes.


\subsection{Modelo de 3 Parâmetros}

Os resultados dos parâmetros estimados para o modelo de 3 parâmetros logísticos (3PL) estão detalhados na Tabela \ref{tabela-coef3} os 3 parâmetros: discriminação (a), dificuldade (b) e chute (c), permitindo avaliar a qualidade dos itens em relação à diferenciação entre participantes com diferentes níveis de habilidade, o nível de dificuldade de cada item, e a probabilidade de acerto ao acaso. Além disso, a tabela apresenta o ponto máximo de informação para cada item, que é o ponto máximo da curva de informação do item  dada pela equação \ref{eq:info_item}.

\begin{table}[!htb]
	\IBGEtab{%
		\caption{Parâmetros do modelo 3PL}
		\label{tabela-coef3}
	}{%
		\begin{tabular*}{.8\textwidth}{@{\extracolsep{\fill}}clccc@{}}
			\toprule
			Item  & Origem & 
		   \makecell{Discriminação \\(a)}& 
			\makecell{Dificuldade \\ (b)} &
			 \makecell{Acerto Casual \\(c)} 
			   \\ 
			\midrule \midrule
			1 & FAAP & 1.94 & -0.17 & 0.27 \\ 
			2 & PUC & 3.08 & 1.17 & 0.34 \\ 
			3 & FGV-RJ & 1.47 & 0.30 & 0.14  \\ 
			4 & UEA & 1.50 & -1.16 & 0.02  \\ 
			5 & UFPR & 1.12 & 0.77 & 0.08 \\ 
			6 & UNICENTRO & 0.60 & 1.23 & 0.04  \\ 
			7 & ENEM & 0.92 & 0.06 & 0.04 \\ 
			8 & UFMS & 0.62 & -0.60 & 0.00 \\ 
			9 & UEA & 0.98 & 2.00 & 0.12  \\ 
			10 & UMESCAM & 1.44 & -0.44 & 0.40 \\ 
			11 & UFMS & 2.16 & -1.52 & 0.00  \\ 
			12 & PUC-RIO & 1.34 & -0.70 & 0.00 \\ 
			13 & ENEM-Digital & 1.80 & 0.61 & 0.19 \\ 
			14 & IFPR & 0.48 & 1.01 & 0.01  \\ 
			15 & ESPECEX & 0.54 & 3.63 & 0.01 \\ 
			16 & UNIFESO & 1.19 & -0.91 & 0.18  \\ 
			17 & FASM & 1.00 & -1.31 & 0.00  \\ 
			18 & ENEM & 1.20 & -1.29 & 0.01 \\ 
			19 & UFSCAR & 1.36 & -1.38 & 0.00 \\ 
			20 & ENEM & 0.68 & -2.14 & 0.01 \\ 
			21 & UFSCAR & 2.46 & -1.74 & 0.00  \\ 
			22 & UNICENTRO & 0.32 & 2.20 & 0.01 \\ 
			23 & UNICENTRO & 0.68 & -0.04 & 0.00  \\ 
			24 & UFAL & 0.73 & -1.91 & 0.00 \\ 
			25 & ENEM & 1.83 & 1.15 & 0.23  \\ 
			26 & FAMECA & 0.51 & 0.10 & 0.00 \\ 
			27 & UFMS & -1.16 & -3.45 & 0.11  \\ 
			28 & UPE & -0.46 & -2.76 & 0.00  \\ 
			29 & FAMECA & 0.87 & 0.93 & 0.14  \\ 
			30 & UNCISAL & 3.21 & 2.22 & 0.10 \\ 
			\bottomrule
		\end{tabular*}
	}{%
		\fonte{Produzido pelos autores.}
	}
\end{table}

\begin{figure}[!htb]
	\centering
	\includegraphics[width=16cm]{../TCCfigura01.png}
	\caption{Curva característica dos itens.}
	\label{fig:curva_itens}
\end{figure}

A figura \ref{fig:curva_itens} apresenta as curvas características dos itens com os parâmetros listados na Tabela \ref{tabela-coef3}. Observa-se que os itens 27 e 28 possuem inclinações contrárias, indicativas de valores de discriminação negativos, o que é problemático em avaliações. Esses itens sugerem que participantes com maior habilidade têm menor probabilidade de acertá-los, o que não é esperado em um teste bem construído. Conforme apontado por \citeonline{baker2001}, itens com discriminação negativa indicam que há algum problema no item, seja por estar mal formulado ou por gerar desinformação entre os alunos de maior capacidade. \citeonline{ayala2013theory} reforça que um valor negativo de discriminação é um indicativo de que o item deve ser descartado, uma vez que seu comportamento é inconsistente com o modelo. Portanto, a presença de discriminação negativa nesses itens requer atenção imediata, sendo recomendada a revisão ou a exclusão, já que tais inconsistências comprometem a validade da avaliação e dificultam a mensuração precisa da habilidade dos participantes.


\begin{figure}[hb]
	\centering
	\includegraphics[width=16cm]{../itens_disc_negativa.png}
	\caption{Curva característica dos itens com discriminação negativa.}
	\label{fig:itens_disc_negativa}
\end{figure}


Embora os itens 27 e 28 apresentem valores baixos de dificuldade (b), o que sugeriria que seriam itens fáceis, isso não é verdade devido à discriminação negativa (a). Conforme mostrado na Figura \ref{fig:itens_disc_negativa}, a probabilidade de acerto dos participantes diminui à medida que a habilidade aumenta, ou seja, funciona de mandeira contra-intuitiva. No contexto da TRI, o parâmetro b representa o nível de habilidade necessário para que um participante tenha $(c+1)/2$ de probabilidade de acertar o item. Entretanto, a presença de discriminação negativa faz com que o item funcione de maneira inversa: participantes com maior habilidade têm menor probabilidade de acerto.

Essa situação sugere que o item está mal formulado ou gera confusão entre os participantes mais habilidosos, resultando em uma baixa proporção de acertos. Isso compromete a interpretação do valor de b, já que o item não está medindo corretamente a habilidade. Assim, os itens com discriminação negativa não devem ser interpretados como fáceis, mas sim como problemáticos, pois apresentam uma falha fundamental na avaliação da habilidade dos participantes.

Itens com discriminação negativa, como os analisados, não conseguem medir adequadamente a habilidade e podem gerar resultados incoerentes. Portanto, é recomendável revisá-los ou descartá-los para garantir a validade da avaliação. Com base nessas análises, o modelo foi ajustado novamente, removendo-se os itens problemáticos. A tabela \ref{tabela-coef3-excl} apresenta os resultados dos parâmetros para o 2ª ajuste, além disso, foi acrescentado o ponto máximo de informação para cada item, que é o ponto máximo da curva de informação do item  dada pela equação \ref{eq:info_item}, além disso foram verificados itens âncoras, os itens estão ordenados do menos informativo para o mais informativo.



\begin{table}[!hbt]
	\IBGEtab{%
		\caption{Parâmetros do modelo 3PL - 2ª Ajuste}
		\label{tabela-coef3-excl}
	}{%
		\begin{tabular*}{.8\textwidth}{@{\extracolsep{\fill}}clccccc@{}}
			\toprule
			Item  & Origem & 
			\makecell{Discriminação \\(a)}& 
			\makecell{Dificuldade \\ (b)} &
			\makecell{Acerto Casual \\(c)} &
			\makecell{Máxima \\ Informação} &
			\makecell{Item \\ Âncora}
			\\ 
			\midrule \midrule
			22 & UNICENTRO & 0.31 & 2.18 & 0.01 & 0.02 \\ 
			14 & IFPR & 0.48 & 1.00 & 0.00 & 0.06 \\ 
			26 & FAMECA & 0.50 & 0.10 & 0.00 & 0.06 \\ 
			15 & ESPECEX & 0.62 & 3.46 & 0.02 & 0.08 \\ 
			6 & UNICENTRO & 0.63 & 1.25 & 0.05 & 0.09 \\ 
			8 & UFMS & 0.63 & -0.60 & 0.00 & 0.10 \\ 
			23 & UNICENTRO & 0.66 & -0.04 & 0.00 & 0.11 \\ 
			20 & ENEM & 0.69 & -2.13 & 0.00 & 0.12 \\ 
			29 & FAMECA & 0.79 & 0.88 & 0.11 & 0.13 \\ 
			24 & UFAL & 0.74 & -1.89 & 0.00 & 0.14 \\ 
			9 & UEA & 0.96 & 1.98 & 0.12 & 0.18 \\ 
			7 & ENEM & 0.95 & 0.10 & 0.06 & 0.20 \\ 
			10 & UMESCAM & 1.44 & -0.43 & 0.40 & 0.23 \\ 
			17 & FASM & 1.00 & -1.31 & 0.00 & 0.25 \\ 
			16 & UNIFESO & 1.19 & -0.92 & 0.17 & 0.25 \\ 
			5 & UFPR & 1.12 & 0.77 & 0.08 & 0.27 \\ 
			18 & ENEM & 1.20 & -1.30 & 0.00 & 0.36 \\ 
			3 & FGV-RJ & 1.49 & 0.30 & 0.14 & 0.42 & * \\ 
			19 & UFSCAR & 1.33 & -1.39 & 0.00 & 0.44 \\ 
			12 & PUC-RIO & 1.36 & -0.69 & 0.00 & 0.46 \\ 
			1 & FAAP & 1.87 & -0.19 & 0.27 & 0.52 \\ 
			25 & ENEM & 1.82 & 1.14 & 0.23 & 0.53 & * \\ 
			4 & UEA & 1.56 & -1.08 & 0.07 & 0.53 \\ 
			13 & ENEM-Digital & 1.85 & 0.61 & 0.19 & 0.59 \\ 
			11 & UFMS & 2.15 & -1.52 & 0.00 & 1.15 \\ 
			2 & PUC & 3.12 & 1.16 & 0.34 & 1.27 & *\\ 
			21 & UFSCAR & 2.43 & -1.75 & 0.00 & 1.47 \\ 
			30 & UNCISAL & 3.07 & 2.23 & 0.10 & 1.92 & *\\ 
			\bottomrule
		\end{tabular*}
	}{%
		\fonte{Produzido pelos autores.}
		\nota{Ajuste excluindo itens com discriminação negativa.}
			}
\end{table}

Na Figura \ref{fig:info_itens}, nota-se que os itens 22, 14 e 26 oferecem pouca ou nenhuma informação relevante para o teste. Conforme enfatizado por \citeonline{baker2001}, itens com baixa discriminação ou com valores muito baixos de informação máxima (ou seja, que não contribuem significativamente para a mensuração em qualquer nível de habilidade) devem ser considerados para revisão ou exclusão, pois não agregam valor à avaliação.



\begin{figure}[!hbt]
	\centering
	\includegraphics[width=16cm]{../info_itens.png}
	\caption{Curva de informação do teste.}
	\label{fig:info_itens}
\end{figure}

\subsection{Informação do Teste}

A figura \ref{fig:info} ilustra a curva de informação do teste do segundo ajuste, conforme a equação \ref{eq:info_teste}, os pontos representam valores dos parâmetros de dificuldade do item na régua da habilidade. A curva de informação do teste atinge seu pico em 5,45 quando $\theta = -1,44$. A região de $\theta$ com maior informação ($I(\theta) > 5$) é entre -1,8 e -0,9, o que indica que o teste é mais informativo para indivíduos com habilidade abaixo da média. 

A maior concentração de informação ocorre no intervalo de aproximadamente -2,5 a 2,5, o que significa que o teste fornece maior precisão para respondentes cujas habilidades estão dentro dessa faixa. Fora desse intervalo, à medida que $\theta$ se afasta em direção a valores muito baixos ou muito altos, a quantidade de informação diminui consideravelmente, o que implica em menor precisão na estimativa de habilidade para esses extremos. Portanto, o teste se mostra eficaz para diferenciar participantes com habilidades intermediárias, mas perde precisão para aqueles com habilidades muito baixas ou muito altas.

\begin{figure}[H]
	\centering
	\includegraphics[width=16cm]{../info_modelo.png}
	\caption{Curva de informação do teste.}
	\label{fig:info}
\end{figure}

\subsection{Análise dos Itens}



\subsection{Estimativa das habilidades}

Com base no segundo ajuste do modelo logístico de 3 parâmetros (3PL), as habilidades dos respondentes foram estimadas. Esse modelo, após a exclusão de itens problemáticos. A Teoria de Resposta ao Item (TRI) permite que a habilidade  de cada respondente seja estimada em uma escala contínua, considerando o padrão de respostas e os parâmetros dos itens. A estimativa das habilidades leva em conta tanto a dificuldade dos itens (parâmetro b) quanto a discriminação (parâmetro a) e a probabilidade de acerto ao acaso (parâmetro c). Com isso, cada respondente recebe uma estimativa de habilidade, que reflete seu nível de proficiência em relação aos itens aplicados.

A Tabela \ref{summary-habilidade} apresenta a distribuição das habilidades estimadas ($\hat{\theta}$) juntamente com o total de acertos dos participantes. Observa-se que a habilidade mínima estimada é de -2.75, associada a um total de acertos de 2 itens, enquanto a habilidade máxima estimada é de 2.26, correspondente a 26 acertos, sugerindo uma forte relação entre o número de acertos e a habilidade.

A mediana da habilidade é próxima de 0 (0.03), o que indica que metade dos participantes têm habilidades abaixo e metade acima desse valor, sendo um ponto de referência útil por estar no centro da escala normalizada, com média 0 e desvio padrão 1. Já a média da habilidade também é próxima de 0 (-0.00), sugerindo que, em geral, a distribuição das habilidades dos participantes é simétrica ao redor da média.

Para os acertos, a mediana é de 16, e a média de 15,69, indicando que o número de acertos dos participantes está bastante concentrado em torno dessa faixa. O primeiro quartil (1st Qu.) para os acertos é de 13 e o terceiro quartil (3rd Qu.) é de 19, o que demonstra que a maioria dos respondentes acertou entre 13 e 19 itens, com variação moderada.

Além disso, na simulação de um respondente que errou toda a prova, a habilidade estimada foi mínima (-2.77), enquanto para um participante que acertou todos os itens, a habilidade máxima foi de 2.72. Vale lembrar que essas habilidades são expressas em uma escala com média 0 e desvio padrão 1. Caso fosse necessário aproximar os resultados dessa escala ao padrão utilizado no ENEM, seria preciso transformá-la para uma escala com média 500 e desvio padrão 100, conforme o procedimento descrito por \citeonline{inep2021procedimento}.

\begin{table}[!hbt]
	\IBGEtab{%
		\caption{Distribuição da Habilidade estimada e total de acertos}
		\label{summary-habilidade}
	}{%
		\begin{tabular*}{.8\textwidth}{@{\extracolsep{\fill}}rrrrrrr@{}}
			\toprule
		 & Mínimo & $Q_1$ & Mediana & Média & $Q_3$ & Máximo \\ 
			\midrule \midrule
		Habilidade & -2.75 & -0.56 & 0.03 & -0.00 & 0.62 & 2.26 \\ 
		Acertos & 2 & 13 & 16 & 15.69 & 19 & 26 \\ 
			\bottomrule
		\end{tabular*}
	}{%
		\fonte{Produzido pelos autores.}
	}
\end{table}




\begin{figure}[H]
	\centering
	\includegraphics[width=16cm]{../acertos_habilidade.png}
	\caption{Relação entre o número de acertos e a habilidade estimada pela TRI}
	\label{fig:acertos_habilidade}
\end{figure}
