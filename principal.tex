\documentclass[11pt,openright,a4paper,chapter=TITLE,oneside,english,french,spanish,brazil,sumario=tradicional]{abntex2}

\usepackage[brazil]{babel}   % Para hifenar em português
\usepackage{amssymb}
\usepackage{amsmath}
\usepackage{amscd}
\usepackage{t1enc}
\usepackage[T1]{fontenc}		% Selecao de codigos de fonte.
\usepackage{lastpage}			% Usado pela Ficha catalográfica
\usepackage{indentfirst}		% Indenta o primeiro parágrafo de cada seção.
\usepackage{color}				% Controle das cores
\usepackage{graphicx}			% Inclusão de gráficos
\usepackage{microtype} 			% para melhorias de justificação
\usepackage{times}          % Para que o documento aceite a fonte Times Roman
\usepackage{lipsum}				% para geração de dummy text

\usepackage{array, multicol, multirow, makecell} 
\usepackage{graphicx, wrapfig, lscape, rotating, float, booktabs} 
\usepackage[usenames,dvipsnames,svgnames,table]{xcolor} 
\usepackage{tabularx} 
\usepackage{tabulary} 
\usepackage[T1]{fontenc}
\usepackage{placeins}

% No seu documento
\FloatBarrier  % Isso força as figuras a não passarem para a próxima seção



% Pacotes de citações
\usepackage[brazilian,hyperpageref]{backref}	 % Paginas com as citações na bibl
\usepackage[alf]{abntex2cite}	% Citações padrão ABNT
% Configurações do pacote backref
% Usado sem a opção hyperpageref de backref
\renewcommand{\backrefpagesname}{Citado na(s) pagina(s):}
% Texto padrão antes do número das páginas
\renewcommand{\backref}{}
% Define os textos da citação
\renewcommand*{\backrefalt}[4]{
	\ifcase #1 %
		Nenhuma citação no texto.%
	\or
		Citado na página #2.%
	\else
		Citado #1 vezes nas páginas #2.%
	\fi}%
% ---

% Recuo da primeira linha de cada parágrafo
\setlength{\parindent}{1.25cm}

\setlength\afterchapskip{1.5\baselineskip}


\makeatletter
\hypersetup{
     	%pagebackref=true,
		colorlinks=true,       		% false: boxed links; true: colored links
    	linkcolor=black,          	% color of internal links
    	citecolor=black,        		% color of links to bibliography
    	filecolor=magenta,      		% color of file links
		urlcolor=blue,
		bookmarksdepth=4
}
\makeatother


\setlength\cftpartnumwidth{\cftparagraphnumwidth}
\setlength\cftchapternumwidth{\cftparagraphnumwidth}

% Pontinhos em negrito
\renewcommand{\cftchapterleader}{\bfseries\cftdotfill{\cftchapterdotsep}} % Pontinhos em negrito para cap�tulos
\renewcommand{\cftsectionleader}{\bfseries\cftdotfill{\cftsectiondotsep}} % Pontinhos em negrito para se��es
\renewcommand{\cftsubsectionleader}{\bfseries\cftdotfill{\cftsubsectiondotsep}} % Pontinhos em negrito para subse��es
\renewcommand{\cftsubsubsectionleader}{\bfseries\cftdotfill{\cftsubsectiondotsep}}

% Numera��o de p�ginas em negrito
\renewcommand{\cftchapterpagefont}{\bfseries} % Negrito para n�meros de p�ginas de cap�tulos
\renewcommand{\cftsectionpagefont}{\bfseries} % Negrito para n�meros de p�ginas de se��es
\renewcommand{\cftsubsectionpagefont}{\bfseries} % Negrito para n�meros de p�ginas de subse��es
\renewcommand{\cftsubsubsectionpagefont}{\bfseries} % Negrito para n�meros de p�ginas de subse��es

\setlength\cftsectionnumwidth{\cftparagraphnumwidth}
\setlength\cftsubsectionnumwidth{\cftparagraphnumwidth}
\setlength\cftsubsubsectionnumwidth{\cftparagraphnumwidth}
\setlength\cftsectionindent{0pt}
\setlength\cftsubsectionindent{0pt}
\setlength\cftsubsubsectionindent{0pt}
\setlength\cftparagraphindent{0pt}
\renewcommand*{\cftsectionfont}{\bfseries}
\renewcommand*{\cftsubsectionfont}{\bfseries}
\renewcommand*{\cftsubsubsectionfont}{\bfseries}
\renewcommand*{\cftparagraphfont}{\bfseries}

\makeatother

\renewcommand{\ABNTEXchapterfont}{\fontseries{b}\selectfont}
\renewcommand{\ABNTEXchapterfontsize}{\normalsize}
\renewcommand{\ABNTEXsectionfontsize}{\normalsize}
\renewcommand{\contentsname}{\centering{SUMÁRIO}}
\pdfbookmark[0]{\contentsname}{toc}
\renewcommand{\cftchaptername}{\hspace{0em}}
\makeindex

\nobibintoc
\pagestyle{plain}

%+++++++++++++++++++++++++++++++++++++++++++++++++++++++++++++++++++++
%Informações do projeto de TCC:
\titulo{ANÁLISE DE ITENS DE UM SIMULADO DE CIÊNCIAS HUMANAS USANDO A TEORIA DE RESPOSTA AO ITEM}
\newcommand{\tituloingles}{}
\autor{PRISCILA GONÇALVES DALEPIANE}
\orientador{Prof. Dra. Juscelia Dias Mendonça}
\coorientador{} 
\data{\the\year{}} 
\local{CUIABÁ-MT} 


\preambulo{Trabalho apresentado à banca examinadora do Curso de Bacharelado em Estatística da Universidade Federal de Mato Grosso, como requisito parcial, para aprovação da disciplina de Trabalho de Conclusão de Curso 1}


% Insira aqui até cinco palavras chave separadas por ponto. Faça o mesmo para as palavras em inglês. As palavras chave serão automaticamente inseridas no resumo e abstract. O resumo e abstract devem ser aditados no arquivo resumo.tex
%\newcommand{\palavraschave}{Palavra 1. Palavra 2. Palavra 3.}
%\newcommand{\keywords}{Palavra 1. Palavra 2. Palavra 3.}

% Informações da Banca:


%\newcommand{\Data}{ 20 de JKHJK de 2023} 



\begin{document}
%Estilo dos titulos dos capitulos
\chapterstyle{book}
% ----------------------------------------------------------
% ELEMENTOS PRÉ-TEXTO
% ----------------------------------------------------------
\begin{capa}
\begin{figure}[h!]
	\centering
	\includegraphics[height=3.28cm, width=4.18cm]{figcapa.png}
\end{figure}
\vspace*{-0.8cm}	
\center
\textbf{UNIVERSIDADE FEDERAL DE MATO GROSSO\\
INSTITUTO DE CIÊNCIAS EXATAS E DA TERRA\\
BACHARELADO EM ESTATÍSTICA}
\vspace*{2.9cm}


\textbf{\imprimirautor}
\vspace*{2.9cm}

\textbf{\imprimirtitulo}\\

\vspace*{\fill}

\textbf{\imprimirlocal\\
\imprimirdata}

%\vspace*{1cm}
\end{capa}



\thispagestyle{empty}
\begin{center}
	\textbf{\imprimirautor}
\end{center}
\vspace{9cm}

\begin{center}
     \textbf{\imprimirtitulo}
\end{center}
\vspace{1.3cm}

\begin{flushright}

\parbox[t]{8cm}{
\SingleSpacing
	\imprimirpreambulo
}
\vspace{1.1cm}

\parbox[t]{7.5cm}{
	\SingleSpacing
Orientadora: \imprimirorientador
}
\end{flushright}
\vspace*{\fill}


\centerline{\textbf{\imprimirlocal}} 
\centerline{\textbf{\imprimirdata}}
\thispagestyle{empty}
\begin{center}\large
	\imprimirautor
\end{center}
\vspace*{0.2cm}


\begin{center}
	\imprimirtitulo
\end{center}
\vspace*{0.25cm}
\begin{flushright}
	\parbox[t]{9cm}{
	\SingleSpacing
	\imprimirpreambulo
	}
\end{flushright}



\vspace{0.5cm}
\begin{center}
	\noindent APROVADA em \_\_\_\_\_ de \_\_\_\_\_\_\_\_\_\_\_\_ de \_\_\_\_\_\_\_\_
\end{center}
\vspace{1.5cm}

\begin{center}
	BANCA EXAMINADORA
\end{center}
\vspace*{1cm}
\begin{center}
	\begin{minipage}[t]{10cm}
		\hrulefill
		\vspace*{-0.5cm}
		\begin{center}Presidente da Banca/Orientadora: Dra \imprimirorientador 
			\\
			Instituição: Universidade Federal de Mato Grosso\end{center}
		\vspace*{1cm}
		\hrulefill
		\vspace*{-0.5cm}
		\begin{center}Examinador Interno: Dr. Gilmar Jorge de Oliveira Junior\\
			Instituição: Universidade Federal de Mato Grosso\end{center}
		\vspace*{1cm}
		\hrulefill
		\vspace*{-0.5cm}
	\begin{center}Examinador Interno: Dr. Juliano Bortolini\\
	Instituição: Universidade Federal de Mato Grosso\end{center}
	\end{minipage}
\end{center}
\vfill


%\begin{dedicatoria}
   \vspace*{\fill}
   \centering
   \noindent
   \textit{ Elemento opcional, inserido após a folha de aprovação. Não têm título. Geralmente, o autor presta uma homenagem ou dedica seu trabalho a alguém} 
\end{dedicatoria}
%\begin{agradecimentos}
Folha opcional, dirigida àqueles que contribuíram para a elaboração do trabalho. Esta página insere-se após a dedicatória, na qual consta o título "AGRADECIMENTOS". Nesta página, o autor faz agradecimentos a pessoas ou a instituições que deram algum tipo de contribuição ao trabalho.

\end{agradecimentos}
%\begin{epigrafe}
    \vspace*{\fill}
	\begin{flushright}
		\textit{Elemento opcional, onde o aluno apresenta uma citação, seguida da indicação de autoria, relacionada com a matéria tratada no corpo do trabalho. As epígrafes também podem ser apresentadas nas folhas de abertura das seções primárias. A fonte é indicada abaixo da epígrafe, alinhada na margem direita.}
	\end{flushright}
\end{epigrafe}
\begin{resumo}
	
Este trabalho avalia a qualidade de itens e da prova de um simulado de Ciências Humanas por meio da Teoria Clássica dos Testes (TCT) e da Teoria de Resposta ao Item (TRI). A prova foi composta por 30 itens provenientes de instituições públicas e privadas, respondida por 664 pessoas. Foram analisadas apenas respostas de indivíduos que responderam todos os itens. Na análise pela TCT, foi obtido um coeficiente alfa de Cronbach igual de 0,744, indicando boa consistência interna do teste, a análise também apontou 7 itens com correlação ponto bisserial abaixo de 0,30.  Na TRI, foram ajustados modelos logísticos de 1, 2 e 3 parâmetros (1PL, 2PL e 3PL) e o modelo unidimensional de 3 parâmetros apresentou melhor ajuste. A comparação entre os modelos utilizando o teste de razão de verossimilhança. A qualidade do ajuste foi avaliada com o índice M$_2$, que demonstrou um bom ajuste,  enquanto os índices RMSEA, TLI e CFI também indicaram boa adequação do modelo. Na análise dos parâmetros TRI dois itens apresentaram discriminação negativa, indicando problemas. Na investigação, um item havia sido corrigido errado no gabarito e o outro estava mal elaborado, levando o leitor desatento a marcar uma alternativa incorreta. A contribuição de vários itens para a informação da habilidade medida foi praticamente nula, pois para muitos a informação máxima foi próxima de zero. A prova apresentou maior informação para valores baixos de habilidade, sendo que, a habilidade estimada dos examinados foram em sua maioria, medianas e altas,ou seja, a prova foi fácil para esses examinados.

\textbf{Palavras-chaves}: 
\end{resumo}

\begin{resumo}[Abstract]
 \begin{otherlanguage*}{english}
   This is the english abstract.

   \vspace{\onelineskip}

   \noindent
   \textbf{Key-words}: Text, editoration.
 \end{otherlanguage*}
\end{resumo}
% inserir lista de ilustrações
\pdfbookmark[0]{\listfigurename}{lof}
\renewcommand{\listfigurename}{\centering\textbf{LISTA DE FIGURAS}}
\listoffigures*
\cleardoublepage
% inserir lista de tabelas
\newpage
\pdfbookmark[0]{\listtablename}{lot}
\renewcommand{\listtablename}{\centering\textbf{LISTA DE TABELAS}}

\listoftables*
\cleardoublepage
% inserir o sumario
\newpage
\pdfbookmark[0]{\contentsname}{toc}
\tableofcontents*
\cleardoublepage
% ----------------------------------------------------------
% ELEMENTOS TEXTUAIS
% ----------------------------------------------------------
\textual
\chapter{INTRODUÇÃO}

Em muitas situações de avaliação educacional, existe uma variável subjacente de interesse. Esta variável é muitas vezes algo que não é compreendido intuitivamente, como inteligência, \cite{baker2001}. Essa variável pode ser denominada de latente, porque não pode ser medida diretamente, mas é estimado a partir das respostas observadas dos participantes em uma série de itens \cite{pasquali2003}. 


Na educação, a habilidade ou proficiência é a variável latente de interesse devido à sua relevância central na avaliação do desempenho dos alunos e no aprimoramento do ensino. A habilidade é uma representação abstrata e não observável do conhecimento, compreensão e capacidade de aplicação que os alunos possuem em relação a um determinado domínio. A mensuração dessa habilidade pode ser feita por meio de avaliações, que permitem avaliar o progresso individual dos estudantes, identificar áreas de necessidade de apoio e adaptar estratégias de ensino, para atender às demandas específicas de aprendizado.
\begin{comment}
	A avaliação desempenha um papel de extrema importância no contexto educacional, conforme enfatizado por Gimeno (1994) ao afirmar que ``a função fundamental que a avaliação deve cumprir no processo didático é a de informar ou conscientizar os professores acerca de como caminham os acontecimentos em sua turma, os processos de aprendizagem que desencadeiam em cada um de seus alunos, durante o mesmo.'' Essa citação destaca que a avaliação vai além de simplesmente medir resultados; ela serve como um meio essencial para os educadores compreenderem o desenvolvimento de suas turmas. Através da avaliação, os professores podem identificar as necessidades específicas de cada estudante, adaptar suas abordagens pedagógicas e fornecer suporte personalizado, criando um ambiente educacional que promove o crescimento tanto individual quanto coletivo.



A estimativa da habilidade em simulados do oferece aos estudantes uma compreensão precisa do seu nível de preparação e das áreas que requerem melhoria. Isso permite que os alunos identifiquem seus pontos fortes e fracos, direcionando seus esforços de estudo de forma mais eficaz. Além disso, a estimativa da habilidade ajuda os educadores e instituições a adaptar estratégias de ensino e programas de apoio com base nas necessidades reais dos estudantes, contribuindo para um ensino mais eficaz. Para a gestão educacional, a habilidade estimada em simulados pode fornecer informações valiosas sobre o desempenho dos alunos em nível nacional e regional, orientando políticas educacionais informadas e promovendo a melhoria contínua do sistema educacional. 

\end{comment}


No entanto, é necessário entender que, a estimação da habilidade ou do conhecimento do aluno, depende da qualidade dos itens na prova \cite{BORGATTO2012}, portanto a estimativa de parâmetros relacionado ao item desempenha um papel determinante na qualidade das avaliações e na interpretação dos resultados. Esses parâmetros podem ser calculados tanto com a Teoria Clássica dos Testes (TCT), quanto com a Teoria de Resposta ao Item (TRI), que incluem informações sobre a dificuldade, a discriminação e entre outros parâmetros, na qual permitem que os educadores selecionem questões e provas adequadas para medir a habilidade dos alunos de forma precisa. 

A TCT é uma abordagem adotada em escolas e universidades para determinar a nota do aluno, na qual a mensuração do conhecimento e da dificuldade de um item é realizada por meio de métodos como o escore bruto ou a porcentagem de itens respondidos corretamente \cite{pasquali2003}. A TCT, conforme enfatizado por \citeonline{pasquali2018}, apresenta suas próprias limitações e desafios, um dos quais é a dependência dos parâmetros dos itens em relação à amostra de sujeitos na qual esses parâmetros foram originalmente calculados. A dependência da amostra significa que os parâmetros dos itens, como a dificuldade e a discriminação, podem variar com base na composição da amostra de indivíduos que participaram do teste. Isso pode resultar em estimativas instáveis e imprecisas das habilidades dos alunos, especialmente quando se trabalha com diferentes grupos populacionais.

\begin{comment}
	 É importante ressaltar que, de acordo com Pasquali (2003), a TCT não mede diretamente o traço latente e sim avalia o comportamento observado dos alunos em relação aos itens.
\end{comment}


A TRI, por outro lado, aborda esse problema de maneira mais robusta. Ela se baseia em modelos estatísticos que consideram não apenas as respostas dos alunos, mas também as características dos próprios itens. Esses modelos levam em conta a probabilidade de um aluno responder corretamente a um item, com base em sua habilidade subjacente e nos parâmetros do item. Portanto, os parâmetros do item na TRI são considerados propriedades intrínsecas do item e não dependem da amostra de sujeitos. \cite{pasquali2018}. A TRI permite que educadores avaliem não apenas o desempenho global dos alunos, mas também suas habilidades específicas em diferentes áreas do conhecimento. Isso é particularmente valioso porque reconhece que os alunos têm diferentes pontos fortes e fracos, permitindo uma análise mais detalhada das suas necessidades educacionais. 

A incorporação da TRI na avaliação das habilidades dos alunos tem se destacado por sua capacidade de fornecer medidas mais precisas e personalizadas das competências dos estudantes. Atualmente todas as provas no Inep utilizam TRI, como o Sistema de Avaliação de Rendimento Escolar do Estado de São Paulo (SARESP), o Sistema de Avaliação da Educação Básica (SAEB) e o Exame Nacional do Ensino Médio (ENEM). Atualmente o estado de Mato Grosso (MT) está fazendo avaliação da educação pelo Caed (Centro de Políticas Públicas e Avaliação da Educação), para avaliar e aumentar os índices do IDEB (Índice de Desenvolvimento da Educação Básica) no estado.

O Enem é uma avaliação realizada anualmente no Brasil pelo Instituto Nacional de Estudos e Pesquisas Educacionais Anísio Teixeira (Inep). Criado em 1998, inicialmente como uma forma de avaliar a qualidade do ensino médio no país, o Enem passou a ter múltiplas funções ao longo do tempo. Em 2009 que o Enem adotou a Teoria de Resposta ao Item (TRI) como método de avaliação, substituindo o modelo de avaliação tradicional. Atualmente, além de servir como uma ferramenta de avaliação do sistema educacional, o Enem é fundamental para acesso ao ensino superior em diversas instituições, por meio do Sistema de Seleção Unificada (Sisu), Programa Universidade para Todos (ProUni) e Fundo de Financiamento Estudantil (Fies) \cite{inephistorico}.


A família de modelos da TRI foi concebida para abordar a complexidade inerente à mensuração de variáveis latentes. Em particular, a TRI oferece um conjunto de modelos matemáticos e metodologias que permitem a construção de avaliações ou testes que, por sua vez, servem como instrumentos confiáveis para medir as variáveis latentes em questão. Esses testes são cuidadosamente desenvolvidos a partir de uma coleção de itens, cada um projetado para fornecer informações específicas sobre a(s) habilidade(s) ou traço(s) latente(s) que se deseja medir. \cite{pasquali2018}
  

\begin{comment}
	Atualmente, o Enem é dividido em 5 partes, a redação e 4 áreas de conhecimento: Linguagens, Ciências Humanas, Matemática e Ciências da Natureza. Cada uma dessas áreas é considerado como uma habilidade $\theta$. O ENEM estima cada uma dessas habilidades separadamente. Considerando que cada área está medindo o mesmo $\theta$ \cite{inep2021}.
	
	
	Além disso, ao utilizar modelos estatísticos sofisticados, a TRI é capaz de estimar as habilidades latentes dos alunos de forma mais precisa, levando em consideração a dificuldade dos itens e a capacidade discriminativa de cada questão. Dessa forma, a TRI oferece uma abordagem mais justa e confiável para avaliar o progresso dos alunos, fornecendo informações valiosas que podem ser usadas para direcionar o ensino, identificar alunos que precisam de apoio adicional e melhorar o currículo escolar. Consequentemente, a TRI desempenha um papel essencial na promoção de práticas educacionais eficazes e na melhoria contínua da qualidade da educação.
	
\end{comment}


\section{Objetivo Geral}

O objetivo geral é analisar a qualidade dos itens que compõe um simulado de Ciências da Natureza e estimar a habilidade dos respondentes.

\section{Objetivos Específicos}

\begin{itemize}
	
\item Analisar a qualidade dos Itens e da prova utilizando a TCT e a tri. Determinar se os itens que compõem a prova são adequados para estimar a habilidade.	
	
\item Selecionar o melhor modelo TRI para estimação da habilidade: Analisar diferentes modelos estatísticos para estimar as habilidades dos estudantes, visando identificar o modelo que se ajusta de maneira mais adequada aos dados de resposta dos alunos. 

\item Analisar se a prova é adequada para medir a habilidade dos alunos aos quais foi aplicada.

\begin{comment}
\item Estimar os Parâmetros TRI dos Itens: Investigar os parâmetros de dificuldade, discriminação e chute (acerto casual) das questões presentes na prova, visando compreender suas características de medida e avaliar sua qualidade na avaliação da habilidade dos estudantes. Essa análise permite determinar se algumas questões devem ser descartadas devido à sua baixa qualidade ou se podem compor um banco de itens para avaliações futuras. 


%Analise TCT da prova: avaliar a distribuição das alternativas marcadas pelos alunos.
\item Avaliar : Verificar se a prova é unidimensional, ou seja, se está avaliando apenas uma variável latente, assegurando que os itens sejam coesos e estejam medindo a mesma habilidade subjacente. Caso não esteja avaliando a mesma dimensão, avaliar qual a dimensionalidade do teste.
\item Avaliar a Abrangência na faixa da habilidade: Verificar em qual faixa de habilidade o teste se encontra e identificar quais níveis de habilidade precisam de maior representação por meio da inclusão de mais questões.



\item Identificar e definir itens âncoras na prova. Estabelecer escalas interpretáveis de habilidades dos estudantes com base nos itens âncoras, bem como determinar quais questões se enquadram em cada escala.
\end{comment}



\end{itemize}



\section{Justificativa}

Este trabalho se justifica pela importância crítica dos testes educacionais no contexto atual. 	No âmbito educacional, a avaliação desempenha um papel fundamental, indo além de simplesmente medir o conhecimento dos alunos. Conforme destacado por \cite{silva2019}, ``Avalia-se para diagnosticar, para qualificar e para planejar atividades e estratégias que percebam processos de ensino-aprendizagem, bem como necessidades individuais e coletivas dos estudantes.'' enfatizando a importância da avaliação no contexto educacional.

Ao adotar a TRI como base para este trabalho, reconhecemos a necessidade de uma abordagem mais sofisticada e precisa para a avaliação. A TRI oferece a capacidade de mensurar as habilidades dos alunos de maneira individualizada, levando em consideração a dificuldade, a discriminação e o acerto casual de cada item \cite{pasquali2018}. Isso permite uma análise mais aprofundada do desempenho dos alunos e uma compreensão mais precisa de suas competências em diferentes áreas do conhecimento.

A TRI é utilizada na educação para mensurar e compreender o desempenho dos alunos, a eficácia de materiais didáticos e a qualidade de avaliações. Por meio da análise estatística, essa prática permite que educadores e pesquisadores identifiquem quais são as questões mais eficazes para medir habilidades específicas, avaliem a dificuldade dos itens de teste, identifiquem itens enviesados ou com dificuldade inadequada. Compondo assim, uma prova que realmente forneça a habilidade dos examinados.

Além disso, este trabalho se justifica pela relevância da TRI no contexto educacional brasileiro, onde essa abordagem é adotada nas principais avaliações nacionais, estaduais e municipais, como o ENEM, SAEB e SARESP. A compreensão e aplicação adequada da TRI são cruciais para garantir que essas avaliações sejam justas, confiáveis e válidas, contribuindo assim para a melhoria do sistema educacional como um todo.

Ao possibilitar uma análise mais detalhada do desempenho dos alunos, a TRI contribui significativamente para a adaptação de estratégias de ensino, permitindo a identificação de áreas de melhoria no sistema educacional. Além disso, por meio de suas características, a TRI torna possível medir com maior precisão o nível de habilidade dos alunos, ajustando-se de forma mais eficaz às particularidades de cada estudante, o que favorece uma abordagem mais individualizada e eficaz no processo de ensino-aprendizagem.



\begin{comment}
	
	
	Em última análise, este trabalho visa contribuir para a compreensão e aplicação eficaz da TRI na avaliação educacional, fornecendo orientações  sobre como avaliar a qualidade das questões, identificar níveis de habilidade e construir uma avaliação que estime bem as habilidades dos alunos. 
	
	A realização de simulados com notas baseadas na Teoria de Resposta ao Item (TRI) é de também é importante para os alunos que se preparam para o ENEM. Esses simulados proporcionam uma experiência de avaliação mais próxima do exame real, uma vez que a TRI é o método utilizado para calcular as notas no ENEM. Ao fazer esses simulados, os alunos têm a oportunidade de se familiarizar com o formato das questões, entender como a TRI funciona na prática e, mais importante, obter estimativas de suas pontuações reais no ENEM. Isso não apenas ajuda os estudantes a avaliarem seu nível de preparação e identificar pontos fracos e fortes. Isso não apenas promove uma compreensão mais profunda de seu próprio desempenho, mas também pode ser uma ferramenta na tomada de decisões sobre como direcionar seus esforços de estudo e preparação para o exame.
	
	
	
	Em última análise, este trabalho visa contribuir para a compreensão e aplicação eficaz da TRI na avaliação educacional, fornecendo orientações  sobre como avaliar a qualidade das questões, identificar áreas de necessidade de apoio e melhorar a equidade no sistema educacional. Ao fazê-lo, esperamos contribuir para aprimorar a qualidade da educação e, consequentemente, o futuro dos alunos brasileiros.
	
``Sabe-se que avaliar, se tais objetivos foram alcançados, não decorre
de uma simples verificação da aprendizagem. Esse diagnóstico vai muito
além, pois há toda uma conjuntura que propicia a aprendizagem do aluno
ou não. No cotidiano, constata-se que o processo pedagógico ocorre por
meio da relação que se estabelece entre professores, alunos, direção,
administração, estrutura física da escola, comunidade, entre outros, e nessa
relação estão envolvidas as múltiplas dimensões que formam cada ser
humano. Portanto, uma avaliação, que pretenda avaliar a qualidade da
educação oferecida por uma escola, por uma rede ou por um sistema, deve
estar embasada em um modelo que contemple todas as relações possíveis
de serem avaliadas.'' (RODRIGUES, 2006).
\end{comment}




\begin{comment}
...
alguns textos para sevir como base:

``Avalia-se para diagnosticar, para qualificar e para planejar atividades e estratégias que percebam processos de ensino-aprendizagem, bem como necessidades individuais e coletivas dos estudantes.'' (SILVA, 2019)

avaliações
associam-se a um processo interpretativo de dados quantitativos e/ou
qualitativos, supondo um juízo de valor, qualidade ou mérito que tem por
meta diagnosticar e verificar o alcance dos objetivos propostos no processo
ensino-aprendizagem.



...
%DA SILVA, André Felipe Zilio et al. Aplicação do modelo de reposta %nominal da TRI a avaliação educacional de larga escala. Sigmae, v. 8, %n. 2, p. 735-741, 2019.


A analise TCT busca fornecer informações sobre o comportamento dos alunos em relação às opções de resposta e permite avaliar se as alternativas estão sendo utilizadas de maneira apropriada, essa análise permite identificar se alguma alternativa está confundido o aluno.


A avaliação dos parâmetros das questões, como dificuldade, discriminação e acerto casual, é essencial para determinar a qualidade das perguntas presentes na prova. Isso assegura que a avaliação seja justa e precisa, pois a qualidade dos itens é diretamente proporcional à qualidade dos resultados.

A investigação sobre se a prova mede apenas uma única variável latente é crucial para garantir a validade da avaliação. Uma prova que mede múltiplas variáveis latentes ou está sujeita a contaminação de construtos não relacionados pode produzir resultados imprecisos e injustos.

A análise comparativa entre diferentes modelos estatísticos busca identificar o modelo mais adequado para a estimativa das habilidades dos alunos, fornecendo uma base sólida para a interpretação dos resultados.

A identificação de itens âncoras é vital para a equalização, pois fornece pontos de referência confiáveis que permitem a comparação dos resultados entre diferentes edições da avaliação. (VALLE, ??)

A delimitação dos níveis de habilidade dos estudantes e a associação de questões específicas a esses níveis aprimoram a compreensão da progressão de dificuldade da avaliação, permitindo uma análise mais precisa do desempenho dos alunos em diferentes estágios.

Por fim, a estimativa de parâmetros dos itens garante que apenas questões de alta qualidade e relevância sejam incluídas no banco de itens, contribuindo para a criação de avaliações de alta qualidade e aprimorando a validade e confiabilidade do processo de avaliação.

Portanto, os objetivos específicos delineados nesta pesquisa se justificam pela necessidade de aprimorar a qualidade da avaliação educacional, garantindo que as medidas obtidas sejam justas, confiáveis e alinhadas com os objetivos pedagógicos. Isso, por sua vez, pode contribuir para uma educação de maior qualidade e para a tomada de decisões educacionais mais informadas.
\end{comment}


\chapter{REFERENCIAL TEÓRICO}

\section{Teoria do Traço Latente}


 A avaliação de características subjacentes, também conhecidas como variáveis latentes, constitui uma parte intrínseca da compreensão e mensuração de fenômenos complexos, que não podem ser diretamente observados ou medidos. Questões como a avaliação de níveis de conhecimento, aptidões ou estados emocionais, como a depressão, exemplificam situações que não podem ser medidas diretamente. A necessidade de quantificar essas características subjacentes tem impulsionado o desenvolvimento da Teoria do Traço Latente \cite{pasquali2003fundamentos}. A variável latente é comumente conhecida por diversos termos, tais como: variável hipotética, fator, construto, conceito, estrutura psíquica, traço cognitivo, traço latente, processo cognitivo, processo mental, estrutura mental, nota, escore, habilidade, aptidão, componente cognitivo, tendência, proficiência, entre outros.

O conceito de "Teoria do Traço Latente" engloba uma classe de modelos matemáticos e traços subjacentes não diretamente observáveis. A TRI e a TCT são duas das maneira pela quais a Teoria do Traço latente é aplicado, especificamente no contexto de avaliações e testes, para estimar desempenho e habilidade \cite{pasquali2018}.

\section{Teoria Clássica dos Testes}

Diferente da TRI onde o foco está no Item, na TCT, o todo é mais importante, o escore bruto é a pontuação direta obtida em um teste. É uma medida simples que não leva em consideração as diferenças na dificuldade dos itens ou o comportamento dos respondentes em itens específicos. A TCT pode ser útil para uma análise prévia do instrumento de avaliação e dos itens \cite{pasquali1996}.


\begin{comment}
	Na TCT a dificuldade do Item é calculado a partir da proporção de sujeitos que respondem corretamente tal item. O escore bruto é calculado somando-se o número de respostas corretas.
	
 Neste trabalho, são aplicados alguns procedimentos da Teoria Clássica dos Testes, incluindo a análise da Correlação Ponto-Bisserial e do Coeficiente Alfa de Cronbach.
\end{comment}

\subsection{Dificuldade}

Conforme \citeonline{pasquali1996}, em testes de aptidão, a dificuldade de um item é descrita pela porcentagem (ou proporção) de participantes que respondem corretamente, variando de 0 a 1, sendo quanto mais próximo de 1, mais acertos o item teve, portanto mais fácil o item é. O valor do $ID_i$ está diretamente relacionado à média de acertos do item no teste e é calculado  pela fórmula: $ID_i = total\_acertos_i /n$, ou seja, o total de acertos no item $i$ dividido pelo número de respondentes. 

O calculo do índice de dificuldade é importante pois permite conhecer a distribuição do grau de dificuldade das questões que segundo \citeonline{pasquali2003}, para que uma avaliação educacional tenha um nível de dificuldade ideal, os índices devem seguir uma distribuição próxima à curva normal, com itens fáceis, médios e difíceis.

\subsection{Discriminação}

Na TCT, há diferentes formas de avaliar a discriminação de um item, que é a capacidade do item de diferenciar entre indivíduos com diferentes níveis de habilidade. Um dos métodos mais comuns é o índice de correlação ponto bisserial, que mede a relação entre o acerto no item e a pontuação total na prova. Outro método usado é discriminação clássica, que é baseada na diferença nas proporções de acertos entre os grupos de alto e baixo desempenho.

\subsubsection{Índice de Discriminação Clássico}

O Índice de Discriminação Clássico ($D_i$) avalia a distribuição de respostas entre dois grupos: o grupo de alto e o grupo de baixo escore. O grupo inferior é composto pelos respondentes com os 27\% menores escores, onde $ABAI_i$ representa a proporção de acertos do item nesse grupo, enquanto o grupo superior é formado pelos respondentes com os 27\% maiores escores e $ACIM_i$ representa a proporção de acertos do grupo superior. O índice DISC varia de -1 a 1 e é calculado como a diferença entre as proporções de respostas apresentadas pelo grupo superior e pelo grupo inferior, ou seja, $DISC_i = ACIM_i - ABAI_i$. Espera-se um valor positivo para a resposta correta, indicando que o grupo superior selecionou mais frequentemente a resposta correta do que o grupo inferior, o que é um indicativo de qualidade do item  \cite{de1983consideraccoes}. 



\subsubsection{Correlação Ponto Bisserial}

O coeficiente bisserial é uma métrica que avalia a correlação entre uma variável dicotômica e uma quantitativa, ou seja, no caso de avaliações, avalia relação entre o desempenho em um item e o desempenho geral na prova \cite{BORGATTO2012}. Ele desempenha um papel importante na análise preliminar dos itens, auxiliando na identificação de questões que podem apresentar problemas, como respostas incorretas no gabarito \cite{de2000teoria}.

A Correlação Bisserial por pontos é indicada quando os itens são dicotômicos, representada pela equação:


\begin{equation}
	r_{bis} = \frac{\bar{X}_p - \bar{X}_t}{S_t}
	\sqrt{\frac{p_i}{1 - p_i}}
\end{equation}

onde:


\noindent $ \bar{X}_p $ é a média dos escores dos examinados que responderam ao item corretamente;

\noindent $ \bar{X}_t $ é a média global dos escores;

\noindent $ S_t $ é o desvio padrão do teste;

\noindent $ p_i $  é a proporção de indivíduos que acertaram o item, ou seja, a proporção para qual a variável binária é 1.


O valor do ponto bisserial de um item oscila de -1 a 1. Quando esse valor é inferior a 0,15, requer uma avaliação pedagógica. Com base nessa análise, o item pode ser submetido a ajustes no gabarito ou considerado para descarte \cite{andrade2010uso}.


\subsection{Alpha de Cronbach}

A avaliação da consistência interna, ou seja, a medida em que as respostas dos participantes em um conjunto de itens, que teoricamente avaliam a mesma habilidade, traço ou construto. 
Para avaliar essa consistência, é comum recorrer a métodos que examinam a correlação entre os itens \cite{souza2017}. Nesse contexto, destacam-se o coeficiente \textit{alpha} de Cronbach, introduzido por Cronbach (1951), este método é umas das ferramentas estatísticas para a avaliação da confiabilidade de instrumentos de medida. Ele auxilia  a determinar se os itens realmente estão medindo o mesmo construto, já que uma alta consistência entre eles sugere que estão alinhados na mensuração da habilidade desejada.

O coeficiente \textit{Alpha} de Cronbach ($\alpha$) pode ser medido por meio da seguinte equação:

\begin{equation}
	\alpha = \frac{k}{k-1}(1 - \frac{\sum_{i=1}^{k}{s^2_i}}{s_T^2})
\end{equation}

 Em que $k$ é o numero de itens do teste, ${s_i^2}$ a variância do item, e
${s_T^2}$ a variância total do teste.

O coeficiente calcula consistência no intervalo de 0 a 1, sendo que quanto mais próximo de 1 maior a consistência, para \citeonline{pasquali2003}, valores entre 0,7 e 0,9 são considerados bons, acima de 0,9 indica itens repetitivos.

\section{Teoria de Resposta ao Item}

A TRI é um conjunto de modelos matemáticos que procuram representar a probabilidade de um indivíduo dar uma certa resposta a um item, com função do parâmetros desses itens e do traço latente, chamada de Theta ($\theta$), do respondente \cite{de2000teoria}. A TRI considera o dado conjunto de respostas ($\textbf{U}$) de um determinado teste como um elemento capaz de fornecer estimativas para a habilidade $\theta$ avaliada. \cite{baker2001}

A escolha do modelo vai depender da natureza dos itens, do questionário, de qual e quantas variáveis latentes que se pretende medir. Na educação, na maioria das vezes, a variável latente medida é o conhecimento do aluno em determinada área, para isso utiliza-se testes com itens de múltipla escolha, que posteriormente será dicotomizado em certo e errado, por isso, será usado modelos para itens dicotômicos. 





\section{Modelos Unidimensionais para Itens Dicotômicos}

\subsection{Modelo Logístico de 1 parâmetro}

O modelo de \citeonline{rasch1960}, também conhecido como Modelo Logístico de 1 Parâmetro (1PL), é um dos modelos mais simples dentro da TRI. Ele pressupõe que a probabilidade de um indivíduo acertar um item depende apenas da diferença entre a habilidade do indivíduo ($\theta$) e a dificuldade do item ($b_i$). Esse modelo não considera parâmetros de discriminação ou acerto casual, assumindo que todos os itens possuem o mesmo poder de discriminação.


A equação que representa a probabilidade de um indivíduo $ j $ com habilidade $ \theta_j $ dar uma resposta correta para o item $ i $ é representada por:

\[
	 P({U_i}_j = 1|{\theta}_j) =
	\frac{1}{1+e^{-D(\theta_j - b_i)}}
\]
 


$\newline$
com $i = 1, 2, ..., I $ e $ j = 1,2, ... , n $, onde:
\newline

\noindent ${U_i}_j$  é uma variável dicotômica que representa a resposta do indivíduo $ j $ para o item $ i $, assumindo o valor 1 quando para resposta correta e 0 para a resposta incorreta.

\noindent ${\theta}_j$  representa o traço latente, ou habilidade, do $ j $-ésimo indivíduo.

\noindent $ b_i $ representa o parâmetro de dificuldade do $i$-ésimo item.

\noindent $ D $ é uma constante fixada em 1,702, introduzida para que a função forneça resultados parecidos ao da curva normal.

A \autoref{fig:rasch} representa um exemplo de curva característica do Item para o modelo de 1 parâmetro, neste exemplo, o valor da dificuldade ($b_i$) é igual a 2.

\begin{figure}[H]
	\centering
	\includegraphics[width=12cm]{rasch.png}
	\caption{Exemplo de curva característica do item para modelo 1PL.}
	\label{fig:rasch}
\end{figure}

\subsection{Modelo Logístico de 2 parâmetros}

O Modelo Logístico de 2 Parâmeros, também conhecido como o Modelo de Birnbaum é uma extensão do Modelo de Rasch.
Desenvolvido por Allan \citeonline{birnbaum1968}, o modelo de 2PL conta com 2 Parâmetros Logísticos: dificuldade e discriminação, a segunda refere-se ao quanto o item diferencia respondentes de habilidades diferentes.

A fórmula associada ao modelo de 2 parâmetros é a seguinte:

\[
	\label{eq:2PL}
	P({U_i}_j = 1|{\theta}_j) =
	\frac{1}{1+e^{-Da_i(\theta_j- b_i)}}
\]


$\newline$
com $i = 1, 2, ..., I $ e $ j = 1,2, ... , n $, onde:
\newline

\noindent $ a_i $ representa o parâmetro de discriminação do $i$-ésimo item.\\

 A diferença entre o Modelo 1PL e o Modelo 2PL está no fato de que o Modelo de 1PL pressupõe que todos os itens diferenciam igualmente, ou seja, que $a_i = 1$.
 
A \autoref{fig:2PL} é um exemplo de uma curva característica do item onde o parâmetro $a_i$ pode variar, neste caso, vale 1,8.


\begin{figure}[H]
	\centering
	\includegraphics[width=12cm]{2PL.png}
	\caption{Exemplo de curva característica do item para modelo 2PL.}
	\label{fig:2PL}
\end{figure}


\subsection{Modelo Logístico de 3 parâmetros}

Desenvolvido por \citeonline{lord1980}, o Modelo Logístico de 3 Parâmetros, ou Modelo de 3PL, é uma extensão do Modelo de 2 Parâmetros. O Modelo de 3PL adiciona um terceiro Parâmetro a função: o de acerto casual (ou "chute"), que representa a probabilidade de um respondente responder corretamente a um item, mesmo que ele não tenha a habilidade necessários para fazê-lo.

A equação do Modelo de 3PL é dada por:

\begin{equation}\label{eq:3PL}
	P({U_i}_j = 1|{\theta}_j) =
	c_i(1-c_i)+\frac{1}{1+e^{-Da_i(\theta_j- b_i)}}
\end{equation}
$\newline$
com $i = 1, 2, ..., I $ e $ j = 1,2, ... , n $, onde:
\newline

\noindent $c_i$ representa o acerto casual do $i$-ésimo item.\\


A figura \ref{fig:3PL} apresenta os mesmos parâmetros $a$ e $b$ mostrados na figura \ref{fig:2PL}, acrescidos do parâmetro $c$, cujo valor é igual a $0,2$.

\begin{figure}[H]
	\centering
	\includegraphics[width=12cm]{3PL.png}
	\caption{Exemplo de curva característica do item para modelo 3PL.}
	\label{fig:3PL}
\end{figure}

O Modelo de 3PL é especialmente útil em testes de múltipla escolha, nos quais os respondentes têm a opção de adivinhar a resposta correta. Essa é uma das razões pelas quais o Modelo de 3PL é usado em avaliações de grande escala, como o Exame Nacional do Ensino Médio (ENEM) \cite{inep2021} e outras avaliações padronizadas. 

\subsection{Função de Informação do Item}

A função de informação do item descreve o poder informativo de um item. Essa função fornece informações sobre a precisão com que um item pode estimar a habilidade latente de um respondente. \cite{de2000teoria} 
A função de informação do item é dada por:

\begin{equation}\label{eq:info_item}
		I_i(\theta) = \dfrac{[\frac{d}{d\theta}P_i(\theta)]^2}{P_i(\theta)Q_i(\theta)}
\end{equation}
onde:

\noindent $I_i(\theta) $ representa a informação fornecida pelo item $i$ no nível de habilidade $\theta$;\\

\noindent $P_i(\theta) = P(U_{ij} = 1| \theta) $ e $ Q_i(\theta) = 1 - P_i(\theta) $ \\

A figura \ref{fig:fii} ilustra a função de informação de um item que informa bem para valores de $\theta$ em torno de $-2$.

\begin{figure}[H]
	\centering
	\includegraphics[width=12cm]{fii_pb.png}
	\caption{Função de Informação do Item.}
	\label{fig:fii}
\end{figure}

\subsection{Função de Informação do Teste}

A informação do teste é obtida pela soma das informações fornecidas pelos itens que compõem a prova \cite{de2000teoria}. Ou seja:

\begin{equation}\label{eq:info_teste}
I(\theta) = \sum_{i=1}^{I}I_i(\theta)
\end{equation}

 O erro-padrão de estimação é calculado como o inverso da raiz quadrada da informação (I) sobre a variável latente ($\theta$), conforme a seguinte fórmula:

\[
EP(\theta) = \dfrac{1}{\sqrt{I(\theta)}}
\]


O erro padrão da medida é inversamente proporcional à informação sobre a variável latente, o que significa que quanto menor a informação, maior será o erro associado. A relação entre a curva de informação do teste e o erro padrão da estimativa pode ser observada na Figura \ref{fig:fft}.

A função de informação do teste é uma medida útil pois permite analisar o quanto o instrumento trás de informação sobre a habilidade e em quais regiões o instrumento estima melhor.

\begin{figure}[H]
	\centering
	\includegraphics[width=12cm]{fft.png}
	\caption{Informação do Teste e Erro Padrão.}
	\label{fig:fft}
\end{figure}



\subsection{Pressupostos dos Modelos Unidimensionais}

Os 3 modelos citados anteriormente são unidimensionais e possuem dois pressupostos principais, na qual os itens de um teste ou simulado devem ser elaborados de forma que atendam esses requisitos. 

\begin{itemize}
	
\item  Unidimensionalidade

Pressupõe que os itens que compõe um teste estão medindo apenas uma traço latente, ou seja, deve haver apenas uma aptidão capaz de realizar os itens do teste. Isso implica que a probabilidade de resposta correta depende apenas da habilidade latente $\theta$ e dos parâmetros do item, sem considerar outras habilidades latentes \cite{pasquali1996}.

\item  Independência Local

Este postulado implica que os itens são respondidos de forma independente. Em outras palavras, as respostas a um item não são afetadas pelas respostas a outros itens após controlar a habilidade latente do indivíduo, assim como diferentes indivíduos do teste são independentes entre si. Isso implica que a probabilidade para respondentes com uma habilidade dada, a probabilidade de resposta a um conjunto de itens é igual aos produtos das probabilidades das respostas do respondente em cada item \cite{pasquali1996}. Matematicamente, isso pode ser expresso como:


\[
 P(U_{1j} = 1, U_{2j} = 1, \cdots, U_{Ij} = 1 |\theta_j) = 
 P(U_{1j} = 1|\theta_j)  P(U_{2j} = 1|\theta_j) \cdots P(U_{Ij} = 1|\theta_j) 
\] 
\[
 = \prod_{i=1}^{I}P(U_{ij} = 1|\theta_j) 
\]

\end{itemize}
\section{Estimação dos Parâmetros}



Um dos desafios do TRI é a estimação dos parâmetros, no geral, temos 3 situações:  (i) Habilidade é conhecida e deseja-se estimar os parâmetros dos itens; (ii) Os parâmetros dos itens são desconhecidos e deseja-se estimar a habilidade; (iii) Ambos os parâmetros são desconhecidos \cite{de2000teoria}.  Nesse contexto, as notações utilizadas para representar os parâmetros a serem estimados são as seguintes:\\


\noindent $ \boldsymbol{\theta} = (\theta_1, \cdots, \theta_n) $  representa o vetor de habilidades dos $n$ indivíduos e

\noindent $ \boldsymbol{\zeta} = (\boldsymbol{\zeta}_1, \cdots, \boldsymbol{\zeta}_I) $ o conjunto de parâmetros dos itens.

\[ \textbf{U}_{n\times I} =  
\begin{bmatrix}
	u_{11} & u_{12} & \cdots & u_{1I} \\
	u_{21} & u_{22} & \cdots & u_{2I} \\
	\vdots & \vdots & \ddots & \vdots\\
	u_{n1} & u_{n2} & \cdots & u_{nI}
\end{bmatrix}
\]

\noindent $\textbf{U}_{n\times I}$ representa a matriz dicotômica de respostas de $ n $ respondentes e $ I $ Itens. A variável $U_{ji}$ é uma variável dicotômica com distribuição Bernoulli, sendo:

\[U_{ji} =    \begin{cases}
	
	  1, & \mbox{resposta correta;}  \\
	
	  0, & \mbox{resposta incorreta.}
	
\end{cases}
\]

Portanto,  

\begin{equation} \label{eq:bern}
P(U_{ji} = u_{ji}|\theta_j, \zeta_i) = P(U_{ji} = 1|\theta_j, \zeta_i)^{u_{ji}}
P(U_{ji} = 0|\theta_j, \zeta_i)^{1 - u_{ji}} = P_{ji}^{u_{ji}}Q_{ji}^{1-u_{ji}}
\end{equation}

\subsection{Estimador de Máxima Verossimilhança (EMV)}

Considerando a situação (i), onde $ \boldsymbol{\theta} $ é conhecido, dados os pressupostos de independência e unidimensionalidade da TRI e pela equação (\ref{eq:bern}). A função de verossimilhança de  $ \boldsymbol{\zeta} $ pode ser escrita como:
\[
L(\boldsymbol{\zeta}) =  \prod_{j=1}^{n}\prod_{i=1}^{I}P(U_{ij} = u_{ji}|\theta_j, \zeta_i) = \prod_{j=1}^{n}\prod_{i=1}^{I}P_{ji}^{u_{ji}}Q_{ji}^{1-u_{ji}}
\]

Os Estimadores de Máxima Verossimilhança (EMV) para $ \boldsymbol{\zeta}_i = (a_i, b_i , c_i )$, dados por:\\

\noindent $
	a_i: D(1 - c_i)\sum_{j=1}^{n}(u_{ji} - P_{ji})(\theta_j - b_i)W_{ij} = 0
$\\

\noindent $
	b_i: -Da_i(1 - c_i)\sum_{j=1}^{n}(u_{ji} - P_{ji})W_{ij} = 0
$\\

\noindent $ 
	c_i: \sum_{j=1}^{n}(u_{ji} - P_{ji})\frac{W_{ij}}{P^*_{ij}} = 0
$\\

\noindent onde:\\

\noindent $P_{ji} = P(U_{ij}|\theta_j,\zeta_i)$ representada pelo modelo 3PL dado na equação \ref{eq:3PL};\\

\noindent $P^*_{ij} = \{1 + e^{-Da_i(\theta_j - b_j)}\}^{-1} $ representa a probabilidade do $j$-ésimo indivíduo acertar o $i$-ésimo item, outra forma da equação \ref{eq:2PL};\\

\noindent $Q^*_{ij} = 1 - P^*_{ij} $ representa a probabilidade do $j$-ésimo indivíduo não acertar o $i$-ésimo item;\\
\\
\noindent $W_{ij} = \dfrac{P^*_{ij}Q^*_{ij}}{P_{ij}Q_{ij}} $ ;\\

Na situação (ii), onde $\boldsymbol{\zeta}$ é conhecido, o EMV para $ \theta_j $,  é dado por:\\


\noindent $ \theta_j : D\sum_{i=1}^{I}{a_i(1-c_i)(u_{ji}-P_{ji})W_{ji}} = 0 $\\

O desenvolvimento detalhado pode ser encontrado pelo leitor no livro de \citeonline{de2000teoria}.

Os Estimadores de Máxima Verossimilhança (EMV) de $ \boldsymbol{\zeta} $ e $ \boldsymbol{\theta} $ requerem um método iterativo para sua estimação, uma vez que não possuem uma solução direta. Em geral, a estimação dos parâmetros é através do método de Newton-Raphoson, mas também podem ser utilizados os métodos de \textit{Scoring} de Fisher ou o Algoritmo EM para essa finalidade \citeonline{de2000teoria}. Sendo assim, considerando $\boldsymbol{\hat{\zeta}}^{(t)}_{i}$
uma estimativa de $\boldsymbol{\hat{\zeta}}$ na iteração $t$, o procedimento de
 Newton-Raphson é dado por:\\

\noindent $ \boldsymbol{\hat{\zeta}}^{(t+1)}_{i} = \boldsymbol{\hat{\zeta}}^{(t)}_{i} + [\bold{H}(\boldsymbol{\hat{\zeta}}^{(t)})]^{-1}\bold{h}({\boldsymbol{\hat{\zeta}}^{(t)}})
$, onde $\bold{H}$ representa a matriz Hessiana.\\

Na situação (iii), na qual tanto $ \boldsymbol{\hat{\zeta}} $ quanto $ \boldsymbol{\theta} $ são desconhecidos, é a situação mais comum, e requer uma estimação conjunta de ambos.
Para tal finalidade, Birbaum (1968), propôs um algoritmo onde os parâmetros são estimados individualmente, usando um do métodos iterativos citados anteriormente.
O processo é dividido em dois estágios. Inicia-se com um chute inicial, uma estimativa grosseira de $ \boldsymbol{\theta} $ considerando que $ \boldsymbol{\zeta} $ é conhecido, após estimada a habilidade, a segunda parte consiste em estimar $ \boldsymbol{\zeta} $ considerando $ \boldsymbol{\theta} $ (estimado na primeira etapa) como conhecido. Esse processo \textit{vai} e \textit{volta} é repetido até a convergência de ambos os parâmetros.



\begin{comment}
Além do método de Estimação de Máxima Verossimilhança (EMV), existem outros métodos de estimação dos parâmetros, que são frequentemente utilizados em contextos de avaliação educacional, estes incluem:
\end{comment}

\subsection{Estimação de Máxima Verossimilhança Marginal (EMVM)}

 Proposto por \citeonline{bock1970} este método estima os parâmetros dos itens e as habilidades dos respondentes em duas etapas, levando em consideração as margens das distribuições das habilidades dos respondentes. É útil quando o número de indivíduos é grande, apresentando vantagens computacionais.
 
\subsection{Estimação Bayesiana}


A abordagem bayesiana, proposta por \citeonline{mislevy1986} é usada para estimar os parâmetros da TRI com base na distribuição \textit{a priori} dos parâmetros. Ela incorpora informações prévias sobre os parâmetros e atualiza essas informações com base nas respostas dos participantes. O ENEM e utiliza o método EAP (\textit{Expected a Posteriori}) para estimar as habilidades dos participantes \cite{inep2021}.


\section{Escala}


Na Teoria de Resposta ao Item (TRI), os resultados da habilidade ($\theta$) e dos parâmetros dos itens são geralmente reportados na escala normal padrão, $\theta_z \sim N(0, 1)$. No entanto, para algumas aplicações, como no Exame Nacional do Ensino Médio (ENEM), esses valores são transformados para uma escala com média 500 e desvio padrão 100, $\theta_{\text{ENEM}} \sim N(500, 100)$ \cite{inep2021procedimento}.

As transformações da habilidade de uma escala padrão para a escala ENEM é dada por:


\[
	\theta_{\text{ENEM}} = 500 + 100 \times \theta_z
\]



%Onde $ \theta_{\text{ENEM}} $ é o valor de $\theta$ na escala do enem e $ \theta_z $ na escala normal padrão.
\begin{comment}
\begin{itemize}
	
	\item Habilidade	
	
\item Dificuldade


\[
b_{\text{ENEM}} = 500 + 100 \times b_z 
\]

\item Discriminação

\[
a_{\text{ENEM}} = \frac{a_z}{100}
\]

\item Chute

\[
c_{\text{ENEM}} = c_z
\]

\end{itemize}
\end{comment}




A escala utilizada para os parâmetros de habilidade ($\theta$) e dos itens é flexível, pois o que realmente importa é a ordem relativa entre esses parâmetros. Essa ordenação determina as relações entre as habilidades dos indivíduos e as características dos itens, e se mantém consistente independentemente da escala adotada. Ao transformar a escala padrão normal  para a escala utilizada pelo ENEM, a hierarquia dos indivíduos em termos de habilidade e a dificuldade dos itens permanecem as mesmas, facilitando a interpretação sem alterar as conclusões essenciais da análise.

\section{Itens Âncoras}

A escala de habilidade é construída a partir dos dados coletados por meio do desempenho dos participantes nos itens do teste. Ela atribui um valor numérico que representa o nível de proficiência de cada participante no domínio avaliado. Quanto maior o valor na escala, maior é o grau de habilidade ou competência do indivíduo naquele domínio específico. A inclusão de múltiplos níveis de dificuldade ou complexidade em um teste visa avaliar o desempenho dos participantes. Esses níveis âncoras representam diferentes graus de proficiência ou habilidade, permitindo comparações mais precisas entre os participantes. Essa abordagem auxilia na definição de critérios de desempenho e na avaliação da capacidade do teste em diferenciar os participantes com base em seus distintos níveis de habilidade \cite{valle2001}.

O conceito de "item âncora" refere-se a um item específico em um teste ou avaliação que é usado como referência para comparar o desempenho dos participantes em diferentes edições do teste. A ideia é que o item âncora permaneça constante ao longo do tempo e seja usado para garantir a equidade e a comparabilidade dos resultados entre diferentes grupos de participantes ou em diferentes momentos de aplicação do teste. De acordo com \citeonline{de2000teoria}, os 3 requisitos para um item ser considerado "âncora" são:

\begin{enumerate}
	\item Ser respondido corretamente por pelo menos 65\% dos respondentes com aquele nível de habilidade.
	
	\[
	P(U = 1| \theta = Z ) \geq 0,65
	\]
	
	\item Ser respondido corretamente por no máximo 50\% dos respondentes que estão em um nível abaixo.
	
	\[
	P(U = 1| \theta = Y ) < 0,50
	\]
	
	\item A diferença entre a proporção de respondentes de diferentes níveis deve ser de pelo menos 30\%.
	
	\[
	P(U = 1| \theta = Z ) - P(U = 1| \theta = Y ) \geq  0,30
	\]
\end{enumerate}












\chapter{METODOLOGIA}

\section{Dados}

Os dados utilizados são  de uma aplicação realizada na plataforma digital est.s de um simulado de ciências humanas. O simulado é realizado online e gratuitamente pela plataforma da empresa. 
A prova contém um total de 30 itens, sendo que 16 deles são provenientes de instituições de ensino públicas (UEA, UFPR, UNICENTRO, UFMS, IFPR, ESPCEX, UFSCAR, UFAL, UFMS, UPE e UNCISAL) , 9 de instituições de ensino particulares (FAAP, PUC-PR, FGV-RJ, EMESCAM, PUC-RIO, UNIFESO, FASM e FAMECA), e 5 itens do ENEM.


As provas em que os respondentes deixaram de responder uma ou mais questões foram excluídas da análise. A prova teve um total de 1055 respondentes, sendo que desses, foram analisados 664 responderam a prova inteira.

%Para análises, será utilizado o R \cite{r} com auxílio dos %pacotes \textit{mirt} \cite{mirt} e \textit{ltm} \cite{ltm}.


\section{Análise TCT}

Na Teoria Clássica dos Testes (TCT), foram avaliados o alfa de Cronbach para medir a consistência interna do teste, além de analisar a variação desse coeficiente ao excluir cada item, com o objetivo de avaliar a contribuição individual dos itens para a consistência geral. Foram também calculadas a correlação bisserial ($r_{bis}$), que relaciona o desempenho no item com a pontuação total, a discriminação clássica, que mede a capacidade do item de diferenciar entre indivíduos de diferentes níveis de habilidade, e a porcentagem de acertos, que indica a dificuldade do item. As análises foram realizadas utilizando o pacote \textit{ltm} \cite{ltm} do R \cite{r}.




\section{Avaliação do Modelo}

Na avaliação de ajuste de modelos TRI, um aspecto a ser considerado é a adequação do modelo aos dados observados. \citeonline{cai2013limited} destacaram a importância desse processo, ressaltando a necessidade de métodos adequados para avaliar a adequação dos modelos aos dados observados. Recentemente, os testes de adequação de informações limitadas têm recebido maior atenção na literatura de psicometria. Esses testes utilizam tabelas marginais de ordem inferior em vez da tabela de contingência completa, tornando-os mais eficientes computacionalmente e menos sensíveis a problemas de convergência \cite{maydeu2014assessing}.

Para avaliar a qualidade do modelo, são consideradas as hipóteses:
\[
\begin{cases}
H_0: \boldsymbol{\pi} = \boldsymbol{\pi}(\boldsymbol{\theta}) \\

H_1: \boldsymbol{\pi} \neq \boldsymbol{\pi}(\boldsymbol{\theta})
\end{cases}
\]

Ou seja, avalia-se se o vetor de probabilidade populacional $\boldsymbol{\pi}$ surge do modelo paramétrico $\boldsymbol{\pi}(\boldsymbol{\theta})$ contra a alternativa de que o modelo está incorreto \cite{maydeu2006limited}.

Neste trabalho, utilizaremos a estatística $M_2$ proposta por \citeonline{maydeu2005limited} para avaliar o ajuste do modelo. Essa estatística é parte de uma família de estatísticas de informação limitada, denominada $M_r$, desenvolvida para avaliar modelos TRI. A estatística $M_2$ é particularmente útil porque utiliza momentos de ordem 2 em vez da tabela de contingência completa, o que a torna mais adequada para modelos TRI. \citeonline{maydeu2006limited} demonstraram que, especialmente quando $r=2$, a estatística $M_2$ apresenta desempenho superior em comparação com estatísticas de informação completa. 



O RMSEA (índice de raiz quadrada média do erro de aproximação) é um índice de ajuste absoluto que mede a discrepância média entre o modelo especificado e os dados observados. O valor do RMSEA varia de 0 a 1, sendo que quanto mais próximo a 0, melhor o modelo \cite{kline2016principles}. \citeonline{maydeu2014assessing} propôs a estatística de informação limitada RMSEA$_{2}$ para aplicações em modelos TRI, na qual utiliza momentos bivariados e é estimado através do $M_2$. O RMSEA$_2$ pode ser estimado por:

\[
	\hat{\epsilon}_2 = \sqrt{Max\left(\frac{\hat{M_{2}} - df_{2}}
		{N \times df_{2}}, 0 \right) } ,
\]

\noindent onde $df_2$ equivale ao grau de liberdade para o momento $r = 2$.



\begin{comment}
	Outro método método para avaliar o modelo são os índices TLI  (\textit{Tucker–Lewis Index}) e CFI (\textit{Comparative Fit Index}), TLI significa Índice Tucker-Lewis e CFI significa Índice de Ajuste Comparativo. O TLI  compara o modelo estimado com um modelo teórico nulo e visa determinar se todos os indicadores são
	associados a um único fator latente, o CFI é um indicador adicional que serve para comparar modelos alternativos \cite{boruchovitch2017dark}. Ambos os indicadores indicam modelos com bom ajustes quando seu valor próximos de 1 \cite{hair2009multivariada}). 
\end{comment}


Outro método para avaliar o modelo são os índices TLI (\textit{Tucker–Lewis Index}) e CFI (\textit{Comparative Fit Index}). O TLI, ou Índice Tucker-Lewis, compara o modelo estimado com um modelo teórico nulo e visa determinar se todos os indicadores estão associados a um único fator latente. Já o CFI, ou Índice de Ajuste Comparativo, é um indicador adicional utilizado para comparar modelos alternativos. Ambos os índices sugerem um bom ajuste quando seus valores se aproximam de 1 \cite{hair2009multivariada}.

\citeonline{timothy2015} considera o ajuste adequado quando o RMSEA é menor que 0,05. Quanto ao CFI e TLI, os onde valores acima de 0,90 sugerem um ajuste aceitável, e valores acima de 0,95 são considerados indicativos de um excelente ajuste.

\begin{comment}
---------------------------

Para verificar a adequação do melhor modelo, TLI e CFI são dois índices de ajuste usados na teoria de resposta ao item para avaliar o ajuste do modelo aos dados \cite{alvarenga2020item}.

 TLI significa Índice Tucker-Lewis e CFI significa Índice de Ajuste Comparativo. Esses índices de ajuste fornecem informações sobre o quão bem o modelo se ajusta às respostas dos itens observados. O TLI compara o ajuste do modelo especificado com um modelo nulo, enquanto o CFI compara o ajuste do modelo especificado com um modelo de linha de base. Valores mais altos de TLI e CFI indicam melhor ajuste entre o modelo e os dados. Esses índices de ajuste são aplicados no IRT para avaliar a adequação do modelo em representar a relação entre a característica latente e as respostas dos itens observados. 


RMSEA: O índice de raiz quadrada média do erro de aproximação (RMSEA) é um índice de ajuste absoluto que mede a discrepância média entre o modelo especificado e os dados observados. O valor do RMSEA varia de 0 a 1, e valores abaixo de 0,05 indicam um bom ajuste do modelo. Valores entre 0,05 e 0,08 indicam um ajuste razoável, enquanto valores acima de 0,10 indicam um ajuste pobre. O RMSEA é calculado como a diferença entre a discrepância média observada e a discrepância média esperada, dividida pelo número de graus de liberdade do modelo.

SRMR: O Índice Raiz Quadrada Média Residual Padronizada (SRMR) é uma medida de ajuste global que avalia a diferença entre as correlações observadas e as correlações estimadas pelo modelo. Em outras palavras, ele mede a discrepância entre as correlações amostrais e as correlações que o modelo estima. É uma medida padronizada, o que significa que ele é independente da escala de medida das variáveis do modelo. Ele é calculado dividindo o erro médio quadrático (EMQ) pelo erro médio quadrático residual (EMQr), que é a diferença entre o EMQ e o EMQ dos resíduos. O valor do SRMR varia de 0 a 1, sendo que valores menores indicam um melhor ajuste do modelo. Uma regra geral é que um valor de SRMR menor que 0,08 indica um bom ajuste do modelo, enquanto valores acima de 0,1 indicam um ajuste pobre. Valores entre 0,08 e 0,1 indicam um ajuste razoável.

(buscar referencias)
\end{comment}
Além disso, foram = avaliado o Akaike Information Criterion (AIC), criado por Akaike (1974) e o Deviance Information Criterion (DIC). Segundo esses critérios, o modelo com o menor valores de AIC e DIC se ajustam melhor aos dados \cite{raftery2006}.








\chapter{RESULTADOS}

\section{Análise TCT}

Segundo a ótica TCT, que analisa o escore bruto, ou seja, a soma de todos os acertos, o simulado apresentou notas que variam entre 2 e 26 pontos em um total de 30 questões. A média geral foi de 16 pontos, com desvio padrão de 4,6 pontos. A figura \ref{fig:hist_acertos} apresenta distribuição de frequência de acertos no teste. A mediana é de 16, indicando que o número de acertos dos participantes está bastante concentrado em torno dessa faixa. O primeiro quartil para os acertos é de 13 e o terceiro quartil é de 19, o que demonstra que a maioria dos respondentes acertou entre 13 e 19 itens.

\begin{figure}[H]
	\centering
	\caption{Distribuição do total de acertos do simulado.}
	\includegraphics[width=16cm]{hist_acertos.png}
	\label{fig:hist_acertos}
	\parbox{\textwidth}{
	\centering % 
	\makebox[16cm][l]{
		\parbox{16cm}{
			\raggedright
			\small \textbf{Fonte}: Elaborado pelos autores.
		}
	}
}
\end{figure}

\begin{table}[H]
	\centering
		\caption{Índices TCT.}
		\label{tabela-tct}
		\begin{tabular*}{\textwidth}{@{\extracolsep{\fill}}clccccc@{}}
			\toprule
			\textbf{Item} & \textbf{Origem} &
			 \makecell{\textbf{\% Erro} \\\textbf{ $(ID_i)$ }} & 
			 \makecell{\textbf{\% Acerto}\\}&
			  \makecell{\textbf{Discriminação} \\ \textbf{($D_i$)}} & \makecell{\textbf{Ponto} \\ \textbf{Bisserial}} & \makecell{\textbf{Cronbach} \\ \textbf{Excluindo item}} \\ 
\hline \textbf{1 }& FAAP & 32,2\% & 67,8\% & 0,552 & 0,468 & 0,730 \\ 
\hline \textbf{2 }& PUC & 56,0\% & 44,0\% & 0,394 & 0,319 & 0,741 \\ 
\hline \textbf{3 }& FGV-RJ & 49,4\% & 50,6\% & 0,551 & 0,468 & 0,730 \\ 
\hline \textbf{4 }& UEA & 21,5\% & 78,5\% & 0,464 & 0,486 & 0,729 \\ 
\hline \textbf{5 }& UFPR & 61,0\% & 39,0\% & 0,507 & 0,415 & 0,734 \\ 
\hline \textbf{6 }& UNICENTRO & 63,7\% & 36,3\% & 0,367 & 0,317 & 0,740 \\
\hline \textbf{7 }& ENEM & 48,6\% & 51,4\% & 0,523 & 0,431 & 0,733 \\ 
\hline \textbf{8 }& UFMS & 41,4\% & 58,6\% & 0,402 & 0,358 & 0,738 \\ 
\hline \textbf{9 }& UEA & 73,9\% & 26,1\% & 0,307 & \textbf{0,290} & 0,741 \\ 
\hline \textbf{10} & EMESCAM & 23,0\% & 77,0\% & 0,390 & 0,402 & 0,734 \\ 
\hline \textbf{11} & UFMS & 11,9\% & 88,1\% & 0,366 & 0,488 & 0,731 \\ 
\hline \textbf{12} & PUC-RIO & 32,8\% & 67,2\% & 0,564 & 0,507 & 0,727 \\ 
\hline \textbf{13} & ENEM-Digital & 54,1\% & 45,9\% & 0,541 & 0,449 & 0,731 \\ 
\hline \textbf{14} & IFPR & 61,0\% & 39,0\% & 0,358 & \textbf{0,299} & 0,742 \\ 
\hline \textbf{15} & ESPCEX & 86,1\% & 13,9\% & 0,155 & \textbf{0,205} & 0,744 \\ 
\hline \textbf{16} & UNIFESO & 24,4\% & 75,6\% & 0,441 & 0,431 & 0,733 \\ 
\hline \textbf{17} & FASM & 25,0\% & 75,0\% & 0,399 & 0,399 & 0,735 \\ 
\hline \textbf{18} & ENEM & 22,4\% & 77,6\% & 0,423 & 0,439 & 0,732 \\ 
\hline \textbf{19} & UFSCAR & 19,4\% & 80,6\% & 0,391 & 0,431 & 0,733 \\ 
\hline \textbf{20} & ENEM & 20,8\% & 79,2\% & 0,297 & 0,306 & 0,740 \\ 
\hline \textbf{21} & UFSCAR & 8,1\% & 91,9\% & 0,242 & 0,450 & 0,734 \\ 
\hline \textbf{22} & UNICENTRO & 65,7\% & 34,3\% & 0,241 & \textbf{0,235} & 0,745 \\ 
\hline \textbf{23} & UNICENTRO & 49,2\% & 50,8\% & 0,429 & 0,357 & 0,738 \\ 
\hline \textbf{24} & UFAL & 22,1\% & 77,9\% & 0,307 & 0,336 & 0,738 \\ 
\hline \textbf{25} & ENEM & 61,1\% & 38,9\% & 0,399 & 0,350 & 0,738 \\ 
\hline \textbf{26} & FAMECA & 51,1\% & 48,9\% & 0,390 & 0,305 & 0,742 \\ 
\hline \textbf{27} & UFMS & 85,8\% & 14,2\% & 0,000 & \textbf{0,006} & 0,753 \\ 
\hline \textbf{28} & UPE & 77,0\% & 23,0\% & -0,068 & \textbf{-0,061} & 0,760 \\ 
\hline \textbf{29} & FAMECA & 57,5\% & 42,5\% & 0,389 & 0,324 & 0,740 \\ 
\hline \textbf{30} & UNCISAL & 87,5\% & 12,5\% & 0,078 & \textbf{0,119} & 0,747 \\
		\hline  \textbf{Total} &&&&&& 0,744 \\
			\bottomrule
		\end{tabular*}\\
		\vspace*{0.5cm}
		\small{\textbf{Fonte:} Produzido pelos autores.}
\end{table}


A Tabela \ref{tabela-tct} apresenta os principais índices TCT para os itens. Inicialmente, observa-se que o item 28 possui índices de discriminação negativo, indicando que esses item foi mais acertados por indivíduos de menor habilidade.


O alfa de Cronbach obtido foi de 0,744, um valor próximo ao recomendado, sendo considerado adequado para a análise. No entanto, foi observado um aumento no alfa de Cronbach ao excluir os itens 21, 27, 28 e 30, o que sugere que esses itens podem estar impactando negativamente a consistência interna da prova.


Os índices de dificuldade do teste estão distribuídos conforme ilustrado na figura \ref{fig:hist_difi}. O item mais fácil é o 21, com 91,9\% de acertos, enquanto o item mais difícil é o 30, com apenas 12,5\% de acertos. Além disso, o item mais difícil apresenta um valor bisserial abaixo do recomendado. A média do índice de dificuldade (DIFI) foi de 50,6\%, com um desvio padrão de 23,4\%.

\begin{figure}[H]
	\centering
	\caption{Distribuição da dificuldade clássica dos itens.}
	\includegraphics[width=16cm]{hist_difi.png}
	\parbox{\textwidth}{
		\centering % 
		\makebox[16cm][l]{
			\parbox{16cm}{
				\raggedright
				\small \textbf{Fonte}: Produzido pelos autores.
			}
		}
	}
	\label{fig:hist_difi}
\end{figure}


Observa-se pela figura \ref{fig:hist_difi} que a distribuição dos índices de dificuldade dos itens não segue uma distribuição normal, que sugere que uma prova deve ter uma maior concentração de itens de dificuldade média, complementada por uma menor quantidade de itens fáceis e difíceis, de forma a se aproximar de uma distribuição normal. Segundo essa recomendação, o ideal seria que a maioria dos itens tivesse uma dificuldade intermediária. A tabela \ref{tabela-dificuldade-obtida} mostra uma comparação entre o recomendado na tabela \ref{tabela-class-ID} o que indica que faltam itens na faixa de 40\% a 60\%.

\begin{table}[H]
	\centering
		\caption{Distribuição ideal dos itens por ID.}
		\label{tabela-dificuldade-obtida}
		\begin{tabular}{lccc}
			\toprule
				\textbf{Faixa} & \textbf{Total Itens} &\textbf{ Distribuição Esperada} & \textbf{Distribuição Obtida}   \\ 
			\hline
			 \textbf{I} & 3 & 10\% &  10,0\% \\ 
			\hline
			\textbf{II} & 7 & 20\% & 23,3\% \\
			\hline
			\textbf{III} & 8 & 40\% & 26,7\% \\ 
			\hline
			 \textbf{IV}& 9 & 20\% & 30,0\% \\ 
			\hline
			 \textbf{V} & 3 & 10\% & 10,0\% \\ 
			\bottomrule
		\end{tabular}\\
		\vspace*{0.5cm}
		\small{\textbf{Fonte:} Produzido pelos autores.}
\end{table}



\section{Análise TRI}

\subsection{Avaliação do Modelo}

Para avaliar a adequação dos modelos unidimensionais de TRI aos dados do simulado, foram
testados os três modelos distintos: o modelo de um parâmetro logístico (1PL), o modelo de dois parâmetros logísticos (2PL) e o modelo de três parâmetros logísticos (3PL).

\begin{table}[!htb]
	\centering
		\caption{Teste Razão de verossimilhança.}
		\label{tabela-anova}
		\begin{tabular}{lcccc}
			\hline
			\textbf{Modelo} &  \textbf{ log-verossimilhança }& $\boldsymbol{\chi^2}$ & \textbf{df} &\textbf{ p-valor }\\ 
			\hline
			\textbf{1PL} &  -10991,57 &  &  &  \\ 
			\hline
			\textbf{2PL} &  -10768,56 & 446,01 & 29 & 0,000 \\ 
			\hline
			\textbf{3PL} & -10741,91 & 53,29 & 30 & 0,006 \\ 
			\hline
		\end{tabular}\\
		\vspace*{0.5cm}
		\small{\textbf{Fonte:} Produzido pelos autores.}
\end{table}



Os resultados do teste de razão de verossimilhança, apresentados na \ref{tabela-anova}, indicam que a inclusão de parâmetros adicionais melhora significativamente o ajuste do modelo. O modelo de 2PL mostrou uma melhoria significativa em relação ao modelo de 1PL ($p < 0,001$), e o modelo de 3PL também apresentou um ajuste superior ao 2PL ($p = 0,006$). Esses resultados sugerem que, entre os modelos testados, o 3PL é o mais adequado para representar os dados do simulado, capturando de forma mais precisa as variáveis latentes relacionadas ao desempenho dos respondentes.

\begin{table}[!htb]
	 \centering
		\caption{Teste de adequação dos modelos. }
		\label{tabela-m2}
		\begin{tabular}{lcccccccc}
			\hline
			\textbf{Modelo} & \textbf{M}$_\textbf{2}$ & \textbf{df} &\textbf{ p-valor} & \textbf{RMSEA} & \textbf{RMSEA$_\textbf{5}$} & \textbf{RMSEA$_{\textbf{95}}$} & \textbf{TLI} & \textbf{CFI} \\ 
			\hline 
		\textbf{1PL} & 1167 & 434 & 0,0000 & 0,0504 & 0,0469 & 0,0539 & 0,81 & 0,81 \\ 
		\hline
		\textbf{2PL} & 485 & 405 & 0,0036 & 0,0173 & 0,0104 & 0,0228 & 0,98 & 0,98 \\ 
		\hline
		\textbf{3PL} & 371 & 375 & 0,5502 & 0,0000 & 0,0000 & 0,0134 & 1,00 & 1,00 \\ 
			\hline
		\end{tabular}\\
		\vspace*{0.5cm}
		\small{\textbf{Fonte:} Produzido pelos autores.}
\end{table}


A tabela \ref{tabela-m2} apresenta os resultados do teste de adequação dos modelos. O teste M$_2$ mostra que o modelo de 1PL não se ajusta bem, com um valor de p < 0,001 indicando rejeição da hipótese nula de bom ajuste e índices de TLI e CFI de 0,81, abaixo do valor de referência de 0,90. O modelo de 2PL apresenta valores altos para TLI e CFI (0,98), porém, também não passa no teste de adequação,com um p-valor de 0,0036, sugerindo que ele não se ajusta bem aos dados.

O modelo de 3PL demonstra o melhor ajuste entre os três modelos testados. Com um p-valor de 0,5502, que não rejeita a hipótese nula de bom ajuste, os índices TLI e CFI perfeitos (1,00), além de um RMSEA próximo de zero. Esses resultados indicam que o modelo de 3PL tem a melhor representação das relações entre os itens do teste e a habilidade latente dos respondentes.


\subsection{Modelo de 3 Parâmetros}

Os resultados dos parâmetros estimados para o modelo de 3 parâmetros logísticos (3PL) estão detalhados na Tabela \ref{tabela-coef3} os 3 parâmetros: discriminação (a), dificuldade (b) e chute (c), permitindo construir a curva característica de cada item. 

\begin{table}[!htb]
	\centering
		\caption{Parâmetros do modelo 3PL}
		\label{tabela-coef3}

		\begin{tabular*}{.9\textwidth}{@{\extracolsep{\fill}}clccc@{}}
			\toprule
			\textbf{Item}  & \textbf{Origem} & 
		   \makecell{\textbf{Discriminação} \\\textbf{(a)}}& 
			\makecell{\textbf{Dificuldade} \\ \textbf{(b)}} &
			 \makecell{\textbf{Acerto Casual} \\\textbf{(c)}} 
			   \\ 
		\hline \textbf{1 }& FAAP & 1,94 & -0,17 & 0,27 \\ 
		\hline \textbf{2 }& PUC & 3,08 & 1,17 & 0,34 \\ 
		\hline \textbf{3 }& FGV-RJ & 1,47 & 0,30 & 0,14  \\ 
		\hline \textbf{4 }& UEA & 1,50 & -1,16 & 0,02  \\ 
		\hline \textbf{5 }& UFPR & 1,12 & 0,77 & 0,08 \\ 
		\hline \textbf{6 }& UNICENTRO & 0,60 & 1,23 & 0,04  \\ 
		\hline \textbf{7 }& ENEM & 0,92 & 0,06 & 0,04 \\ 
		\hline \textbf{8 }& UFMS & 0,62 & -0,60 & 0,00 \\ 
		\hline \textbf{9 }& UEA & 0,98 & 2,00 & 0,12  \\ 
		\hline \textbf{10} & UMESCAM & 1,44 & -0,44 & 0,40 \\ 
		\hline \textbf{11} & UFMS & 2,16 & -1,52 & 0,00  \\ 
		\hline \textbf{12} & PUC-RIO & 1,34 & -0,70 & 0,00 \\ 
		\hline \textbf{13} & ENEM-Digital & 1,80 & 0,61 & 0,19 \\ 
		\hline \textbf{14} & IFPR & 0,48 & 1,01 & 0,01  \\ 
		\hline \textbf{15} & ESPECEX & 0,54 & 3,63 & 0,01 \\ 
		\hline \textbf{16} & UNIFESO & 1,19 & -0,91 & 0,18  \\ 
		\hline \textbf{17} & FASM & 1,00 & -1,31 & 0,00  \\ 
		\hline \textbf{18} & ENEM & 1,20 & -1,29 & 0,01 \\ 
		\hline \textbf{19} & UFSCAR & 1,36 & -1,38 & 0,00 \\ 
		\hline \textbf{20} & ENEM & 0,68 & -2,14 & 0,01 \\ 
		\hline \textbf{21} & UFSCAR & 2,46 & -1,74 & 0,00  \\ 
		\hline \textbf{22} & UNICENTRO & 0,32 & 2,20 & 0,01 \\ 
		\hline \textbf{23} & UNICENTRO & 0,68 & -0,04 & 0,00  \\ 
		\hline \textbf{24} & UFAL & 0,73 & -1,91 & 0,00 \\ 
		\hline \textbf{25} & ENEM & 1,83 & 1,15 & 0,23  \\ 
		\hline \textbf{26} & FAMECA & 0,51 & 0,10 & 0,00 \\ 
		\hline \textbf{27} & UFMS & \textbf{-1,16} & -3,45 & 0,11  \\ 
		\hline \textbf{28} & UPE & \textbf{-0,46} & -2,76 & 0,00  \\ 
		\hline \textbf{29} & FAMECA & 0,87 & 0,93 & 0,14  \\ 
		\hline \textbf{30} & UNCISAL & 3,21 & 2,22 & 0,10 \\ 
			\bottomrule
		\end{tabular*}\\
		\vspace*{0.5cm}
		\small{\textbf{Fonte:} Produzido pelos autores.}
\end{table}
\clearpage
\begin{figure}[H]
	\centering
	\caption{Curva característica dos itens.}
	\includegraphics[width=16cm]{../TCCfigura01.png}
	\parbox{\textwidth}{
		\centering % 
		\makebox[16cm][l]{
			\parbox{16cm}{
				\raggedright
				\small \textbf{Fonte}: Produzido pelos autores.
			}
		}
	}
	\label{fig:curva_itens}
\end{figure}

A figura \ref{fig:curva_itens} apresenta as curvas características dos itens com os parâmetros listados na Tabela \ref{tabela-coef3}. Observa-se que os itens 27 e 28 possuem inclinações contrárias, indicativas de valores de discriminação negativos, o que é problemático em avaliações. Esses itens sugerem que participantes com maior habilidade têm menor probabilidade de acertá-los, o que não é esperado em um teste bem construído. Conforme apontado por \citeonline{baker2001}, itens com discriminação negativa indicam que há algum problema no item, seja por estar mal formulado ou por gerar desinformação entre os alunos de maior capacidade. \citeonline{ayala2013theory} reforça que um valor negativo de discriminação é um indicativo de que o item deve ser descartado, uma vez que seu comportamento é inconsistente com o modelo. Portanto, a presença de discriminação negativa nesses itens requer atenção imediata, sendo recomendada a revisão ou a exclusão, já que tais inconsistências comprometem a validade da avaliação e dificultam a mensuração precisa da habilidade dos participantes.


\subsubsection{Análise dos itens com discriminação negativa}

Itens com discriminação negativa, como os analisados, não conseguem medir adequadamente a habilidade e podem gerar resultados incoerentes. Portanto, é recomendável revisá-los ou descartá-los para garantir a validade da avaliação.


\begin{figure}[!htb]
	\centering
	\caption{Curva característica dos itens com discriminação negativa.}
	\includegraphics[width=16cm]{../itens_disc_negativa.png}
	\parbox{\textwidth}{
		\centering % 
		\makebox[16cm][l]{
			\parbox{16cm}{
				\raggedright
				\small \textbf{Fonte}: Produzido pelos autores.
			}
		}
	}
	\label{fig:itens_disc_negativa}
\end{figure}


Conforme mostrado na Figura \ref{fig:itens_disc_negativa}, a probabilidade de acerto dos participantes diminui à medida que a habilidade aumenta, ou seja, funciona de maneira contra-intuitiva. No contexto da TRI, o parâmetro b representa o nível de habilidade necessário para que um participante tenha $(c+1)/2$ de probabilidade de acertar o item. Entretanto, a presença de discriminação negativa faz com que o item funcione de maneira inversa: participantes com maior habilidade têm menor probabilidade de acerto.

As figuras \ref{fig:item_27} e \ref{fig:item_28} exibem a proporção de respostas marcadas para cada alternativa em função do total de acertos, com os resultados agrupados em intervalos de 5 acertos.

Analisando o item 27, observa-se que respondentes com mais acertos tenderam a marcar a alternativa D, as alternativas da questão apresentam interpretações que, em alguns casos, são excessivamente simplificadas ou, ao contrário, exageradas, o que pode confundir o estudante. A alternativa D, por exemplo, sugere que o narrador se torna um "reprodutor do sistema opressor", o que está incorreto mas pode ser uma leitura possível se o aluno fizer a interpretação do texto. Isso cria uma ambiguidade que pode desviar o foco da análise sociológica adequada, levando o aluno a acreditar que o item está correto. Por fim, a questão abre espaço para múltiplas interpretações, o que pode ser visto como uma falha em um contexto de avaliação, onde se espera que haja uma resposta claramente correta.
	
No caso do item 28, verificou-se que ele foi corrigido com o gabarito incorreto, quando a alternativa correta deveria ser a letra C. Quando um item é corrigido com o gabarito incorreto, a TRI pode evidenciar que o comportamento esperado para um item de boa qualidade não está ocorrendo. Nesse contexto, a TRI auxilia na identificação de itens que estão mal formulados ou corrigidos erroneamente.

\begin{figure}[H]
	\centering
	\caption{Proporção de alternativas marcadas total de acertos do item 27.}
	\includegraphics[width=15cm]{../alternativas2_item27.png}
	\parbox{\textwidth}{
		\centering % 
		\makebox[15cm][l]{
			\parbox{15cm}{
				\raggedright
				\small \textbf{Fonte}: Produzido pelos autores.
			}
		}
	}
	\label{fig:item_27}
\end{figure}


\begin{figure}[H]
	\centering
	\caption{Proporção de alternativas marcadas total de acertos do item 28.}
	\includegraphics[width=15cm]{../alternativas2_item28.png}
	\parbox{\textwidth}{
	\centering % 
	\makebox[15cm][l]{
		\parbox{15cm}{
			\raggedright
			\small \textbf{Fonte}: Produzido pelos autores.
		}
	}
}	
	\label{fig:item_28}
\end{figure}



%%%%%%%%%%%%%%%%%%%%%%%%%%%%%%%%%%%%%%%%%%%%%%%%%%%%%%%%%%%%%%%%%%%%%%%%%%%%%%%

\subsubsection{Modelo de 3 Parâmetros - 2º Ajuste}

 Com base nessas análises, o modelo foi ajustado novamente, removendo-se os itens problemáticos. Para a avaliação do modelo o resultados do teste M$_2$ indicam um bom ajuste do modelo. O valor de M$_2$ (344,6) com p-valor = $0,54$ sugere que o modelo não é significativamente diferente dos dados. O RMSEA é 0 (intervalo de confiança de 0 a 0,0137), indicando um ajuste excelente. Os índices de ajuste incremental TLI (1) e CFI (1), reforçam que o modelo se ajustou bem aos dados.
 
 
  A tabela \ref{tabela-coef3-excl} apresenta os resultados dos parâmetros para o 2ª ajuste, além disso, foi acrescentado o ponto máximo de informação para cada item, que é o ponto máximo da curva de informação do item  dada pela equação \ref{eq:info_item}.

Na Figura \ref{fig:info_itens}, nota-se que os itens 6, 8, 14, 15, 20, 22, 23, 24 e 26 oferecem pouca ou nenhuma informação relevante para o teste. Conforme enfatizado por \citeonline{baker2001}, itens com baixa discriminação ou com valores muito baixos de informação máxima (ou seja, que não contribuem significativamente para a mensuração em qualquer nível de habilidade) devem ser considerados para revisão ou exclusão, pois não agregam valor à avaliação.



\begin{figure}[!ht]
	\centering
	\caption{Curva de informação dos itens.}
	\includegraphics[width=16cm]{../info_itens.png}
	\parbox{\textwidth}{
		\centering % 
		\makebox[16cm][l]{
			\parbox{16cm}{
				\raggedright
				\small \textbf{Fonte}: Produzido pelos autores.
			}
		}
	}
	\label{fig:info_itens}
\end{figure}

\begin{table}[h]
	\centering
		\caption{Parâmetros do modelo 3PL - 2ª Ajuste}
		\label{tabela-coef3-excl}
		\begin{tabular}{clcccc}
			\hline
			\textbf{Item}  & \textbf{Origem} & 
			\makecell{\textbf{Discriminação} \\\textbf{(a)}}& 
			\makecell{\textbf{Dificuldade} \\\textbf{ (b)}} &
			\makecell{\textbf{Acerto Casual} \\\textbf{(c})} &
			\makecell{\textbf{Máxima} \\ \textbf{Informação}} 
			\\ 
		\hline \textbf{22} & UNICENTRO & 0,32 & 2,12 & 0,01 & 0,03 \\ 
		\hline \textbf{14} & IFPR & 0,49 & 0,98 & 0,00 & 0,06 \\ 
		\hline \textbf{15} & ESPECEX & 0,53 & 3,68 & 0,00 & 0,06 \\ 
		\hline \textbf{26} & FAMECA & 0,51 & 0,10 & 0,00 & 0,07 \\ 
		\hline \textbf{6 }& UNICENTRO & 0,59 & 1,23 & 0,04 & 0,08 \\ 
		\hline \textbf{8 }& UFMS & 0,62 & -0,60 & 0,00 & 0,10 \\ 
		\hline \textbf{23} & UNICENTRO & 0,67 & -0,04 & 0,00 & 0,11 \\ 
		\hline \textbf{20} & ENEM & 0,68 & -2,16 & 0,00 & 0,11 \\ 
		\hline \textbf{24} & UFAL & 0,72 & -1,92 & 0,00 & 0,13 \\ 
		\hline \textbf{29} & FAMECA & 0,90 & 0,93 & 0,14 & 0,15 \\ 
		\hline \textbf{28} & UPE & 0,86 & -1,06 & 0,00 & 0,18 \\ 
		\hline \textbf{7 }& ENEM & 0,96 & 0,14 & 0,07 & 0,20 \\ 
		\hline \textbf{9 }& UEA & 1,02 & 2,01 & 0,13 & 0,20 \\ 
		\hline \textbf{10} & UMESCAM & 1,55 & -0,31 & 0,44 & 0,25 \\ 
		\hline \textbf{17} & FASM & 1,00 & -1,31 & 0,00 & 0,25 \\ 
		\hline \textbf{16} & UNIFESO & 1,25 & -0,82 & 0,21 & 0,26 \\ 
		\hline \textbf{5 }& UFPR & 1,11 & 0,76 & 0,08 & 0,26 \\ 
		\hline \textbf{18} & ENEM & 1,18 & -1,31 & 0,00 & 0,35 \\ 
		\hline \textbf{3 }& FGV-RJ & 1,49 & 0,32 & 0,15 & 0,42 \\ 
		\hline \textbf{12} & PUC-RIO & 1,33 & -0,70 & 0,00 & 0,44 \\ 
		\hline \textbf{19} & UFSCAR & 1,37 & -1,37 & 0,00 & 0,47 \\ 
		\hline \textbf{25} & ENEM & 1,83 & 1,15 & 0,23 & 0,54 \\ 
		\hline \textbf{1 }& FAAP & 1,88 & -0,22 & 0,25 & 0,54 \\ 
		\hline \textbf{13} & ENEM-Digital & 1,77 & 0,59 & 0,18 & 0,55 \\ 
		\hline \textbf{4 }& UEA & 1,50 & -1,18 & 0,00 & 0,56 \\ 
		\hline \textbf{11} & UFMS & 2,13 & -1,53 & 0,00 & 1,13 \\ 
		\hline \textbf{2 }& PUC & 2,96 & 1,18 & 0,34 & 1,14 \\ 
		\hline \textbf{21} & UFSCAR & 2,50 & -1,73 & 0,00 & 1,56 \\ 
		\hline \textbf{30} & UNCISAL & 3,41 & 2,18 & 0,10 & 2,37 \\
		\hline
		\end{tabular}\\
		\vspace*{0.5cm}
		\small{\textbf{Fonte:} Produzido pelos autores.}
\end{table}



\subsection{Informação do Teste}

A figura \ref{fig:info} ilustra a curva de informação do teste do segundo ajuste, conforme a equação \ref{eq:info_teste} e a linha pontilhada representa o erro padrão. A curva de informação do teste atinge seu pico em 5,71 quando $\theta = -1,46$. A região de $\theta$ com informação mais precisa ($I(\theta) > 5$) é entre -1,9 e -0,7, o que indica que o teste é mais preciso para indivíduos com habilidade abaixo da média. 

A maior concentração de informação ocorre no intervalo de aproximadamente -2,5 a 2,5, o que significa que o teste fornece maior precisão para respondentes cujas habilidades estão dentro dessa faixa. Fora desse intervalo, à medida que $\theta$ se afasta em direção a valores muito baixos ou muito altos, a quantidade de informação diminui consideravelmente, o que implica em menor precisão na estimativa de habilidade para esses extremos. Portanto, o teste se mostra eficaz para diferenciar participantes com habilidades intermediárias, mas perde precisão para aqueles com habilidades muito baixas ou muito altas.

\begin{figure}[H]
	\centering
	\caption{Curva de informação e erro padrão do teste.}
	\includegraphics[width=16cm]{../info_modelo2.png}
	\parbox{\textwidth}{
		\centering % 
		\makebox[16cm][l]{
			\parbox{16cm}{
				\raggedright
				\small \textbf{Fonte}: Produzido pelos autores.
			}
		}
	}
	\label{fig:info}
\end{figure}


\subsection{Estimativa das habilidades}

Com base no segundo ajuste do modelo logístico de 3 parâmetros (3PL), as habilidades dos respondentes foram estimadas. Esse modelo, após a correção e exclusão de itens problemáticos. A TRI permite que a habilidade  de cada respondente seja estimada em uma escala contínua, considerando o padrão de respostas e os parâmetros dos itens. A estimativa das habilidades leva em conta tanto a dificuldade dos itens quanto a discriminação e a probabilidade de acerto ao acaso. Com isso, cada respondente recebe uma estimativa de habilidade.



\begin{table}[H]		
	\centering
		\caption{Distribuição da Habilidade estimada e total de acertos}
		\label{summary-habilidade}
		\begin{tabular}{lcccccc}
			\hline
			& \textbf{Mínimo} & $\textbf{Q}_1$ & \textbf{Mediana} & \textbf{Média} & $\textbf{Q}_3$ & \textbf{Máximo} \\ 
			\hline
			$\boldsymbol{\hat{\theta}_{(0,1)}}$ & -2,79 & -0,56 & 0,00 & 0,00 & 0,62 & 2,30 \\ 
			\hline
			$\boldsymbol{\hat{\theta}_{(500,100)}}$ & -220,5 & -443,7 & 501,4 & 500,0 & 562,3 & 730,3 \\
			\hline
		\end{tabular}\\
		\vspace*{0.5cm}
		\small{\textbf{Fonte:} Produzido pelos autores.}
\end{table}

A Tabela \ref{summary-habilidade} apresenta a distribuição das habilidades estimadas ($\hat{\theta}$) para a escala normal padrão e para a escala  de $\mu = 500$ e $\sigma = 100$. A habilidade mínima estimada é de -2,75, associada a um total de acertos de 2 itens, enquanto a habilidade máxima estimada é de 2,26, correspondente a 26 acertos. Não houveram respondentes que acertaram ou erraram todos os itens. Na simulação de um respondente que errou toda a prova, a habilidade estimada foi de -2,81, enquanto para um participante que acertou todos os itens, a habilidade máxima foi de 2,72. 

\clearpage

\begin{figure}[H]
	\centering
	\caption{Distribuição da habilidade e curva de informação do teste.}
	\includegraphics[width=16cm]{../habilidade_info.png}
		\parbox{\textwidth}{
		\centering % 
		\makebox[16cm][l]{
			\parbox{16cm}{
				\raggedright
				\small \textbf{Fonte}: Produzido pelos autores.
			}
		}
	}
	\label{fig:info_habilidade}
\end{figure}

Podemos observar pela figura \ref{fig:info_habilidade} que a distribuição das habilidades estimadas dos participantes está concentrada majoritariamente (75\%) na região central, entre $\theta = -1$ e $\theta = 1$, o que reflete uma maior frequência de respondentes com habilidades intermediárias. No entanto, a curva de informação do teste atinge seu pico em valores de $\theta$ um pouco mais baixos, sugerindo que o teste está melhor ajustado para discriminar habilidades abaixo da média, especialmente entre -1,9 e -0,7.

Este descompasso entre a concentração das habilidades estimadas e a área de maior informação indica uma lacuna no teste. A curva de informação esteja fornecendo melhor precisão para participantes com habilidades mais baixas, há um decréscimo visível na informação na região central do gráfico, onde está a maior parte dos respondentes. 

Para melhorar a precisão do teste, seria recomendável a adição de itens que aumentem a informação nessa faixa central. Dessa forma, o teste poderá discriminar melhor entre os participantes que têm habilidades próximas à média, melhorando a precisão das estimativas nessa região. Além disso, adotar itens mais discriminativos ao teste ajudaria a melhorar a capacidade de diferenciar entre níveis de habilidade.

\begin{figure}[!hbt]
	\centering
		\caption{Relação entre o número de acertos e a habilidade estimada pela TRI}
	\includegraphics[width=16cm]{../acertos_habilidade.png}
	\parbox{\textwidth}{
		\centering % 
		\makebox[16cm][l]{
			\parbox{16cm}{
				\raggedright
				\small \textbf{Fonte}: Produzido pelos autores.
			}
		}
	}
	\label{fig:acertos_habilidade}
\end{figure}

O gráfico da figura\ref{fig:acertos_habilidade} mostra a relação entre o número total de acertos e a habilidade estimada pelo modelo TRI. Observa-se a diferença no mesmo número de acertos, onde, para um mesmo total de acertos, há uma dispersão de valores da habilidade estimada, podendo um indivíduo que acertou mais itens receber uma nota menor do que um que acertou menos itens. A tabela \ref{exemplo-10acertos} exemplifica diferentes habilidades estimadas para respondentes com 10 acertos na prova, com o vetor de respostas ordenado do item com menor para o maior valor do parâmetro de dificuldade (b).


\begin{table}[!hbt]
	\centering
		\caption{Vetor de resposta e habilidade estimada para respondentes com 10 acertos}
		\label{exemplo-10acertos}
		\begin{tabular*}{0.7\textwidth}{@{\extracolsep{\fill}}lccc@{}}
			\hline
			 & \textbf{Vetor de Respostas} & $\boldsymbol{\hat{\theta}_{(0,1)}}$  & $\boldsymbol{\hat{\theta}_{(500,100)}}$ \\ 
\hline \textbf{1 }& 00000000010110010110001011010 & -2,18 & 282,40 \\ 
\hline \textbf{2 }& 10100000011010000010001001110 & -1,60 & 340,00 \\ 
\hline \textbf{3 }& 01001001100011001001001100000 & -1,57 & 342,90 \\ 
\hline \textbf{4 }& 11100000010010100010100001100 & -1,53 & 346,80 \\ 
\hline \textbf{5 }& 00101001100010100001100001100 & -1,24 & 376,10 \\ 
\hline \textbf{6 }& 10011001110110010000000000100 & -1,19 & 381,20 \\ 
\hline \textbf{7 }& 11110100100010000010011000000 & -1,19 & 381,00 \\ 
\hline \textbf{8 }& 11110001100000010001000100100 & -1,13 & 387,00 \\ 
\hline \textbf{9 }& 10110011000101000000000101100 & -1,12 & 388,10 \\ 
\hline \textbf{10} & 11101011010000010000100000100 & -1,09 & 390,90 \\ 
\hline \textbf{11} & 01111100000000001100100001100 & -1,09 & 391,10 \\ 
\hline \textbf{12} & 10101001100101001100010000000 & -1,08 & 392,10 \\ 
\hline \textbf{13} & 00101001101010001011010000000 & -1,04 & 396,40 \\ 
\hline \textbf{14} & 00111100001110000011000100000 & -1,01 & 398,90 \\ 
\hline \textbf{15} & 10111000101010101000010000000 & -0,92 & 408,10 \\ 
\hline \textbf{16} & 11110101110000100000100000000 & -0,91 & 408,60 \\ 
\hline \textbf{17} & 01111100001110110000000000000 & -0,91 & 408,70 \\ 
\hline \textbf{18} & 11111110110000010000000000000 & -0,86 & 413,80 \\ 
\hline \textbf{19} & 10111110110000001000000010000 & -0,85 & 414,90 \\ 
\hline \textbf{20} & 10111101110000100000000010000 & -0,83 & 417,40 \\ 
\hline \textbf{21} & 10111111010110000000000000000 & -0,75 & 425,20 \\ 
\hline
		\end{tabular*}\\
		\vspace*{0.5cm}
		\small{\textbf{Fonte:} Produzido pelos autores.}
\end{table}


A tabela \ref{exemplo-10acertos} demonstra que, com o mesmo número de acertos (10), há uma diferença nas habilidades estimadas. Indivíduos com maior coerência nas resposta, ou seja, aqueles que acertam itens de dificuldade progressiva, recebem uma pontuação maior, enquanto aqueles com menos coerência, que acertam itens mais difíceis e erram os mais fáceis, tendem a receber uma nota inferior. Esse comportamento pode ser interpretado como uma indicação de chute, uma vez que o participante acerta itens que exigem maior habilidade, mas erra itens de menor dificuldade, o que foge do esperado.



\chapter{CONCLUSÃO}
	

A análise dos dados do simulado utilizando tanto a TCT permitiu uma visão geral sobre a prova, a TRI permitiu uma compreensão mais aprofundada sobre o desempenho dos participantes e a qualidade dos itens. Na TCT, os resultados indicaram uma distribuição de acertos concentrada em torno da mediana. No entanto, a TCT, ao basear-se apenas no escore bruto, tem limitações para captar nuances importantes da prova, como a dificuldade e discriminação de cada item, bem como a coerência nas respostas dos indivíduos. Nesse sentido, a TRI apresentou-se como uma ferramenta mais robusta, permitindo não apenas a estimativa das habilidades dos participantes em uma escala contínua, mas também oferecendo uma análise detalhada de cada item em termos de dificuldade, discriminação e probabilidade de acerto ao acaso.

O modelo 3PL apresentou o melhor ajuste aos dados, indicando ser o mais apropriado para este simulado, conforme demonstrado pelos testes de adequação (M2, TLI, CFI e RMSEA). Esse modelo permitiu estimar com maior precisão a habilidade dos participantes, diferenciando não apenas o total de acertos, mas a coerência no padrão de respostas.

Os itens com bons índices de discriminação e dificuldade adequada, como os itens 3, 5, 7 e 12, mostraram-se eficazes na diferenciação de participantes com diferentes níveis de habilidade e, portanto, devem ser preservados no banco de itens para futuras aplicações. Esses itens fornecem boas informações e contribuem para a precisão das estimativas de habilidade. Por outro lado, os itens 6, 8, 14, 15, 20, 22, 23, 24 e 26 que apresentaram discriminação muito baixa e pouca informação para o teste, além de coeficientes bisserial problemáticos, precisam ser revisados e não é recomendado para incluir em futuros simulados, pois possuem pouco informam sobre a variável latente de interesse, no caso, a habilidade do respondente.

Foram identificados dois itens problemáticos, o item 27 e o item 28, ambos apresentando discriminação negativa. No caso do item 28, verificou-se que o gabarito estava incorreto, sendo necessária a correção do item. Já o item 27 foi mal formulado, com alternativas  que induzem interpretações equivocadas, dificultando a identificação da resposta correta.

A análise da curva de informação do teste indicou que a prova foi mais eficaz para discriminar participantes com habilidades abaixo da média, especialmente na faixa entre -1,9 e -0,7. No entanto, identificou-se uma lacuna no teste em relação aos participantes com habilidades próximas à média, onde a quantidade de informação foi menor. Para melhorar a precisão da avaliação nesse grupo, seria recomendável a inclusão de itens que ofereçam mais informação nessa faixa central de habilidades. A adição desses itens poderá aumentar a capacidade do teste de discriminar com maior precisão os participantes com habilidades intermediárias, melhorando a qualidade da mensuração.

Em síntese, a análise realizada demonstra a importância da TRI como ferramenta complementar à TCT, fornecendo uma estimativa mais acurada das habilidades dos participantes e permitindo uma análise detalhada dos itens. A partir dos resultados, conclui-se que a revisão de itens com baixa discriminação e a adequação da curva de informação podem contribuir para aprimorar a qualidade do teste e aumentar sua capacidade de mensurar habilidades de forma precisa e justa em toda a gama de participantes.

%\chapter{CRONOGRAMA}


\begin{table}[!htb]\centering 
	\caption{Cronograma de atividades a serem realizadas na monografia.}
	\begin{tabular}{c|c|c|c|c|c|c|c|c|c|c|c}\hline 
		& \multicolumn{6}{c|}{2023}      & \multicolumn{5}{c}{2024} \\ \cline{2-12} 
	\multirow[c]{-2}{*}{Atividades}	& Jul & Ago & Set & Out & Nov & Dez & Jan & Fev & Mar & Mai  & Abr \\ 
	\hline 	Escolha do tema	& X &   &   &   &   &   &   &   &   &    &  \\ 
	\hline  Planejamento & X &   &   &   &   &   &   &   &   &    &  \\ 
	\hline  Revisão da Literatura & X & X  &  X &   &   &   &   &   &   &    &  \\ 
	\hline  Elaboração do projeto &  &  X  & X  & X  &   &   &   &   &   &    &  \\ 
	\hline  Entrega do TCC1 &  &   &   & X  &   &   &   &   &   &    &  \\ 
	\hline  Apresentação do TCC1 &  &   &   &  X &   &   &   &   &   &    &  \\ 
	\hline  Extração e organização dos dados &  &   &   &   & X  &   &   &   &   &    &  \\ 
	\hline  Análise dos dados &  &   &   &   &   &  X & X  & X  &  X &    &  \\ 
	\hline  Interpretação e Discussões &  &   &   &   &   &   &  X &  X &  X &  X  &  \\ 
	\hline  Resumo &  &   &   &   &   &   &   &   &   &  X  &  \\ 
	\hline  Conclusões &  &   &   &   &   &   &   &   &   &  X  &  \\ 
	\hline  Defesa &  &   &   &   &   &   &   &   &   &    & X  \\ 
	\hline 
	\end{tabular} 
	\label{tab:Crono} 
\end{table} 



%\include{referencias}
%\section{Equações e Fórmulas}
Use o ambiente \texttt{equation} para escrever Equações e Fórmulas numeradas:

\begin{equation}
	\forall x \in X, \quad \exists \: y \leq \epsilon
\end{equation}

Escreva expressões matemáticas entre \$ e \$, como em $ \lim_{x \to \infty}
\exp(-x) = 0 $, para que fiquem na mesma linha.

Também é possível usar colchetes para indicar o início de uma expressão matemática que não precisa estar numerada.

\[
\left|\sum_{i=1}^n a_ib_i\right|
\le
\left(\sum_{i=1}^n a_i^2\right)^{1/2}
\left(\sum_{i=1}^n b_i^2\right)^{1/2}
\]




\section{Figuras}
As Figuras podem ser criadas diretamente em \LaTeX, como o exemplo da \autoref{fig_circulo}.

\begin{figure}[!htb]
	\caption{A delimitação do espaço}\label{fig_circulo}
	\begin{center}
		\setlength{\unitlength}{5cm}
		\begin{picture}(1,1)
			\put(0,0){\line(0,1){1}}
			\put(0,0){\line(1,0){1}}
			\put(0,0){\line(1,1){1}}
			\put(0,0){\line(1,2){.5}}
			\put(0,0){\line(1,3){.3333}}
			\put(0,0){\line(1,4){.25}}
			\put(0,0){\line(1,5){.2}}
			\put(0,0){\line(1,6){.1667}}
			\put(0,0){\line(2,1){1}}
			\put(0,0){\line(2,3){.6667}}
			\put(0,0){\line(2,5){.4}}
			\put(0,0){\line(3,1){1}}
			\put(0,0){\line(3,2){1}}
			\put(0,0){\line(3,4){.75}}
			\put(0,0){\line(3,5){.6}}
			\put(0,0){\line(4,1){1}}
			\put(0,0){\line(4,3){1}}
			\put(0,0){\line(4,5){.8}}
			\put(0,0){\line(5,1){1}}
			\put(0,0){\line(5,2){1}}
			\put(0,0){\line(5,3){1}}
			\put(0,0){\line(5,4){1}}
			\put(0,0){\line(5,6){.8333}}
			\put(0,0){\line(6,1){1}}
			\put(0,0){\line(6,5){1}}
		\end{picture}
	\end{center}
	\legend{Fonte: os autores}
\end{figure}

Ou então figuras podem ser incorporadas de arquivos externos, como é o caso da
\autoref{fig_grafico}.

\begin{figure}[!htb]
	\caption{Exemplo}\label{fig_grafico}
	\centering
	\includegraphics[width=5cm]{grafico.pdf}
	\legend{Fonte: \citeonline[p. 24]{araujo2012}}
\end{figure}


\section{Tabelas}
As tabelas devem seguir o padrão do \citeonline{ibge1993}, na  \autoref{tabela-ibge} é apresentado um exemplo .

\begin{table}[!htb]
	\IBGEtab{%
		\caption{Um Exemplo de tabela alinhada que pode ser longa   ou curta, conforme parão do IBGE.}
		\label{tabela-ibge}
	}{%
		\begin{tabular}{ccc}
			\toprule
			Nome & Nascimento & Documento \\
			\midrule \midrule
			Maria da Silva & 11/11/1111 & 111.111.111-11 \\
			\midrule
			João Souza & 11/11/2111 & 211.111.111-11 \\
			\midrule
			Laura Vicua & 05/04/1891 & 3111.111.111-11 \\
			\bottomrule
		\end{tabular}
	}{%
		\fonte{Produzido pelos autores.}
		\nota{Esta é uma nota, que diz que os dados são baseados na  regressão linear.}
		\nota[Anotações]{Uma anotação adicional, que pode ser seguida de várias  outras.}
	}
\end{table}



\section{Citações diretas}

Utilize o ambiente \texttt{citação} para incluir citações diretas com mais de três linhas:

\begin{citacao}
	As citaçõs diretas, no texto, com mais de três linhas, devem ser
	destacadas com recuo de 4 cm da margem esquerda, com letra menor que a do texto
	utilizado e sem as aspas. No caso de documentos datilografados, deve-se
	observar apenas o recuo \cite[5.3]{NBR10520:2002}.
\end{citacao}


O ambiente \texttt{citação} pode receber como parâmetro opcional um nome de idioma previamente carregado nas opções da classe. Nesse caso, o texto da citação é automaticamente escrito em itálico e a hifenização ajustada para o idioma selecionado na opção do ambiente. Por exemplo:
\begin{citacao}[english]
	Text in English language in italic with correct hyphenation.
\end{citacao}


Citação simples, com até três linhas, devem ser
incluídas com aspas. Observe que em \LaTeX  as aspas iniciais são diferentes das finais: ``Amor é fogo que arde sem se ver''.


\section{Notas de rodapé}


As notas de rodapé são detalhadas pela NBR 14724:2011 na seção 5.2.1
\footnote{As notas devem ser digitadas ou datilografadas dentro das margens, ficando
	separadas do texto por um espaço simples de entre as linhas e por filete de 5
	cm, a partir da margem esquerda. Devem ser alinhadas, a partir da segunda linha
	da mesma nota, abaixo da primeira letra da primeira palavra, de forma a destacar
	o expoente, sem espaço entre elas e com fonte menor
	
	\citeonline[5.2.1]{NBR14724:2011}.}\footnote{Caso uma séries de notas sejam
	criadas sequencialmente, o \abnTeX  instrui o \LaTeX\ para que uma vígula seja
	colocada após cada número do expoente que indica a nota de rodapé no corpo do
	texto.}\footnote{Verifique se os números do expoente possuem uma vírgula para
	dividi-los no corpo do texto.}.

% ----------------------------------------------------------
% ELEMENTOS POS-TEXTUAIS
% ----------------------------------------------------------
\postextual
\addtocontents{toc}{\protect\vspace{-22pt}}

\renewcommand{\bibname}{REFERÊNCIAS}
\bibliography{referencias}
%Apendices
%\renewcommand{\apendicesname}{APÊNDICE}
%
\begin{apendicesenv}
%\chapter{}
%\chapter{}
\end{apendicesenv} 

%

%

%\renewcommand{\anexosname}{ANEXOS}
%
\chapter*{ANEXO A - SIMULADO DE CIÊNCIAS HUMANAS}
\addcontentsline{toc}{chapter}{\hspace{4.94em}ANEXO A - SIMULADO DE CIÊNCIAS HUMANAS}
\includepdf[pages=-]{itenssimulado.pdf}

\chapter*{ANEXO B - AUTORIZAÇÃO DE USO DOS DADOS}
\addcontentsline{toc}{chapter}{\hspace{4.94em}ANEXO B - AUTORIZAÇÃO DE USO DOS DADOS}


\newpage
\thispagestyle{empty} % Remove numeração da página
\hspace{-4cm}% Ajusta o deslocamento para a esquerda
\includegraphics[width=\paperwidth, height=\paperheight, keepaspectratio]{../Termo_de_autorizacao_de_uso_de_dados_assinado.jpg}
%\includepdf[pages=-]{../Termo_de_autorizacao_de_uso_de_dados_assinado.pdf}

\begin{comment}
\begin{anexosenv}
	
	\chapter{}
	
	\addcontentsline{toc}{chapter}{\hspace{4.94em}ANEXO A - Itens do simulado}
	\includepdf[pages=-]{itenssimulado.pdf}
	%\chapter{}
	
	
\end{anexosenv}
\end{comment}


\end{document} 