\documentclass{beamer}
\usepackage[utf8]{inputenc}
\usepackage{comment}
\usetheme{Darmstadt}
\usepackage[brazil]{babel} 
\usecolortheme{default}
\usepackage{multirow}
\usepackage{makecell}
%------------------------------------------------------------
%This block of code defines the information to appear in the
%Title page
\title[TRI] %optional
{m}

\subtitle{Trabalho de Conclusão do Curso}

\author[Priscila] % (optional)
{Priscila Dalepiane}

\institute[UFMT] % (optional)
{
	
	Bacharelado em Estatística\\
	Universidade Federal de Mato Grosso
	
}

\date[2024] % (optional)
{Novembro 2024}

%\logo{\includegraphics[height=1cm]{overleaf-logo}}

%End of title page configuration block
%------------------------------------------------------------



%------------------------------------------------------------
%The next block of commands puts the table of contents at the 
%beginning of each section and highlights the current section:

\AtBeginSection[]
{
	\begin{frame}
		\frametitle{Sumário}
		\tableofcontents[currentsection]
	\end{frame}
}
%------------------------------------------------------------


\begin{document}
	
	%The next statement creates the title page.
	\frame{\titlepage}

	
	\section{Introdução}
	
		\begin{frame}
		
		\frametitle{Introdução}
		\begin{columns}
			
			\column{0.6\textwidth}
			
			\begin{center}
				\textbf{Variável Latente}
			\end{center}
			
			Em muitas situações de medição na psicometria, existe uma variável de interesse que não pode ser medida diretamente. Esta variável é denominada \textbf{latente}. \newline  Por exemplo, como medir a inteligência?
			
	

			
			\column{0.4\textwidth}
			
			\begin{figure}
			%	\includegraphics[width=5cm,height=5cm]{medida.jpg}
			\end{figure}
			
			
		\end{columns}

		
		
\end{frame}
	%---------------------------------------------------------
	\begin{frame}
		
		\frametitle{Introdução}
		\begin{columns}
		
		\column{0.6\textwidth}
		
		Na educação, em geral, precisamos avaliar o conhecimento do alunos, por isso a variável latente pode ser chamada de \textbf{habilidade} ou \textbf{proeficiência}.
		A habilidade é uma representação abstrata e não observável do conhecimento.	\newline\newline
		Pode ser estimada por meio de avaliações, chamadas também de instrumentos.

		\column{0.4\textwidth}
		
		\begin{figure}
		%	\includegraphics[width=5cm,height=5cm]{medida.jpg}
		\end{figure}	
		
		\end{columns}
			
\end{frame}
	%---------------------------------------------------------
\begin{frame}
	
	\frametitle{Introdução}
	\begin{columns}
		
		\column{0.6\textwidth}
		
		Um \textbf{instrumento} se refere ao teste ou conjunto de itens que são usados para medir uma habilidade ou traço específico dos respondentes. Cada \textbf{item} nesse instrumento é projetado para medir algo específico e contribui para a avaliação geral da proficiência do indivíduo nesse traço ou habilidade. \newline
		
		A estimação precisa da habilidade depende da qualidade do instrumento.
		
		\column{0.4\textwidth}
		
		\begin{figure}
			%\includegraphics[width=4.4cm,height=7cm]{inst.jpg}
		\end{figure}	
		
	\end{columns}
	
\end{frame}


	
	%---------------------------------------------------------
\begin{frame}
		\frametitle{Introdução}
		\begin{center}
		\textbf{Objetivo Geral}	
		\end{center}
		
			O objetivo geral é analisar a qualidade dos itens que compõe um simulado de Ciências Humanas (instrumento) e estimar a habilidade dos respondentes (traço latente).
			

	\end{frame}
	%---------------------------------------------------------
	\begin{frame}
		\frametitle{Objetivos Específicos}
		\begin{center}
			\textbf{Objetivos Específicos}	
		\end{center}
		
		\begin{itemize}
			\item<1-> Selecionar o melhor modelo TRI para estimação dos parâmetros dos itens e da habilidade.
			\item<2-> Analisar a qualidade dos Itens e da prova utilizando a TCT e a TRI. 
			\item<3-> Determinar se os itens que compõem a prova são adequados para estimar a habilidade.
		\end{itemize}
	\end{frame}
	
	%--------------------------------------------------
	
	\section{Referencial Teórico}
	
	% --------------- TCT ------------------------------------------

	\begin{frame}
		
		\frametitle{Teoria Clássica dos Testes}
		
		A \textbf{Teoria clássica dos testes} é a teoria que antecede o TRI, 
		com cálculos mais simples, onde é o foco é o teste como um todo.
		
	\end{frame}		
	
	\begin{frame}
		
		\frametitle{Teoria Clássica dos Testes}
		
		\begin{center}
			\textbf{Dificuldade Clássica}
		\end{center}
		
		É calculada pela proporção de itens incorretos.
		
		\[
		ID_i =\dfrac{A_i}{n} 
		\]
		onde $A_i$ é o total de respostas incorretas no item e $n$ é o total de respostas ao item. 
		
	\end{frame}	
	
	\begin{frame}
		
		\frametitle{Teoria Clássica dos Testes}
		\begin{center}
			\textbf{Coeficiente Ponto Bisserial}
		\end{center}
		O coeficiente ponto bisserial representa uma métrica que avalia a relação entre o desempenho em um item e o desempenho geral na prova.
		Auxiliando na identificação de questões que podem apresentar problemas, como respostas incorretas no gabarito.
		
		\[
		r_{bis} = \frac{\bar{X}_p - \bar{X}_t}{S_t}
		\sqrt{\frac{p_i}{1 - p_i}}
		\]
		
		
		onde:
		
		
		$ \bar{X}_p :$ média dos escores dos examinados que responderam ao item corretamente;
		
		$ \bar{X}_t :$  média global dos escores;
		
		$ S_t :$  desvio padrão do teste;
	\end{frame}
	
	
	\begin{frame}
		
		\frametitle{Teoria Clássica dos Testes}
		\begin{center}
			\textbf{\textit{Alpha} de Cronbach}
		\end{center}
		Para avaliar se um conjunto de itens avaliam a mesma habilidade.
		
		
		\[
		\alpha = \frac{k}{k-1}(1 - \frac{\sum_{i=1}^{k}{s^2_i}}{s_T^2})
		\]
		
		
		$k$ é o numero de itens do teste;
		
		${s_i^2}$ a variância do item;
		
		${s_T^2}$ a variância total do teste.
		
		
	\end{frame}
	
		
	\begin{frame}
		
		\frametitle{Teoria Clássica dos Testes}
		
		\begin{center}
			\textbf{\textit{Alpha} de Cronbach}
		\end{center}
		
		
		
		O coeficiente calcula consistência no intervalo de 0 a 1, sendo quanto mais próximo de 1 maior a consistência, para Pasquali (2003), valores entre \textbf{0,7 e 0,9} são considerados bons, acima de 0,9	 indica itens repetitivos
		
	\end{frame}
	
	
    % --------------- TRI ------------------------------------------
	\begin{frame}
		
		\frametitle{Teoria de Resposta ao Item}
		
		\begin{center}
			\textbf{Teoria de Resposta ao Item (TRI)}
		\end{center}
		
		\begin{block}{Definição}
			A TRI é um \textbf{conjunto de modelos} matemáticos que procuram representar a probabilidade de um indivíduo dar uma certa resposta a um item, como função dos parâmetros dos itens.
			(ANDRADE, 2000)
		\end{block}
		
	\end{frame}
	
	\begin{frame}
		
		\frametitle{Teoria de Resposta ao Item}
		
		A escolha do modelo depende: \newline
		
	
		\begin{itemize}
			\item<1-> Quantas variáveis latentes se pretende medir;
			\item<1-> Formato dos Itens (Dicotômicos, Múltipla escolha);
			\item<1-> Número de parâmetros a serem estimados.
		    
		\end{itemize}
	\end{frame}

	
	\begin{frame}
		
		\frametitle{Modelos Unidimensionais para Itens dicotômicos}
		
		Modelos unidimensionais assumem que a probabilidade de um indivíduo responder corretamente a um item depende de um único traço latente (habilidade) que se deseja medir. Serão apresentados 3 modelos para casos de itens dicotômicos. 
		
		

	\end{frame}
	
	\begin{frame}
		
		\frametitle{Modelos Unidimensionais para Itens dicotômicos }
		
		\begin{center}
			\textbf{Denotações:}
		\end{center}
		
		$ \boldsymbol{\theta} = (\theta_1, \cdots, \theta_n) \rightarrow $  vetor de habilidades dos $n$ indivíduos; \newline
		
		$ \boldsymbol{\zeta} = (\boldsymbol{\zeta}_1, \cdots, \boldsymbol{\zeta}_I) \rightarrow $ conjunto de parâmetros dos itens.\newline
		
		$ \textbf{U}_{n\times I} =  
		\begin{bmatrix}
			u_{11} & u_{12} & \cdots & u_{1I} \\
			u_{21} & u_{22} & \cdots & u_{2I} \\
			\vdots & \vdots & \ddots & \vdots\\
			u_{n1} & u_{n2} & \cdots & u_{nI}
		\end{bmatrix} \rightarrow
		$ Matriz de respostas
		
		
		
	\end{frame}
	
	% Modelo 1PL
	\begin{frame}
		
		\frametitle{Modelo de 1 Parâmetro Logístico}
		
		\[
				P({U_i}_j = 1|{\theta}_j, b_i) = 
				\frac{1}{1+e^{-D(\theta_j - b_i)}}
		\]
		\begin{figure}
		\includegraphics[width=10cm,height=6cm]{rasch.png}
		\end{figure}

	\end{frame}
	
	% Modelo 2PL
	
	\begin{frame}
		
		\frametitle{Modelo de 2 Parâmetros Logísticos}
		
		\[
			P({U_i}_j = 1|{\theta}_j, a_i, b_i) =
			\frac{1}{1+e^{-Da_i(\theta_j- b_i)}}
		\]
		
		\begin{figure}
			\includegraphics[width=10cm,height=6cm]{../2PL.png}
		\end{figure}

		
	\end{frame}
	
	% Modelo 3PL
	
	\begin{frame}
		
		\frametitle{Modelo de 3 Parâmetros Logísticos}
		\[
			P({U_i}_j = 1|{\theta}_j, a_i, b_i, c_i) =
			c_i+(1-c_i)\frac{1}{1+e^{-Da_i(\theta_j- b_i)}}
		\]
		\begin{figure}
			\includegraphics[width=10cm,height=6cm]{../3PLcci.png}
		\end{figure}

		
	\end{frame}

	% Função de Informação do Item	
	\begin{frame}
		
		\frametitle{Função de Informação do Item}
		
		\begin{center}
			\textbf{Função de Informação do Item}
		\end{center}
		
		$I_i(\theta)$ representa a quantidade de informação que o item $i$ trás sobre a
		habilidade $\theta$.
		
		\[
		I_i(\theta) =  D^2 a_i^2\frac{Q_i(\theta)}{P_i(\theta)} \left[\frac{P_i(\theta) - c_i}{1 - c_i}\right]^2
		\] \\ \pause	
				A Informação é maior quando:
		
		\begin{itemize}
			\item<1-> Maior o valor de $a_i$;
			\item<1-> Menor o valor de $c_i$;
			\item<1-> Quando $b_i$ se aproxima de $\theta_j$.
			
		\end{itemize}
		%$P_i(\theta) = P(U_{ij} = 1| \theta) $ e\\ $ Q_i(\theta) = 1 - P_i(\theta) $

	\end{frame}
	
	
	% função de informação do item - gráfico
	\begin{frame}
		
		\frametitle{Função de Informação do Item}

		\begin{figure}
			\includegraphics[width=10cm,height=8cm]{../info_com_cci.png}
		\end{figure}
		
	\end{frame}
	
	% Função de Informação do Teste	
	\begin{frame}
			
			\frametitle{Função de Informação do Teste}
		
		Obtida pela soma das informações fornecidas pelos itens que compõem a prova.	
			\[
			I(\theta) = \sum_{i=1}^{I}I_i(\theta)
			\]
	
		o erro padrão de estimação é dado por:
			
			\[
			EP(\theta) = \dfrac{1}{\sqrt{I(\theta)}}
			\]
			
	\end{frame}	
		
	% Função de Informação do Teste	- gráfico
	\begin{frame}
		
		\frametitle{Função de Informação do Teste}

		\begin{figure}
			\includegraphics[width=10cm,height=8cm]{fft.png}
		\end{figure}
		
	
			
		
	\end{frame}
	
	\begin{frame}
		
		\frametitle{Suposições do modelo TRI}
	
		\begin{itemize}
			
			\item<1-> Apenas os traços latentes e os parâmetros são necessários para modelar a probabilidade dos indivíduos.
						
			\item<1-> Em modelos unidimensionais, apenas uma dimensão do traço latente e necessária, ou seja, o teste mede a mesma variável latente. (\textbf{Unidimensionalidade})
			
			\item<1-> Respostas entre itens e indivíduos são independentes \textbf{Independência}.
	
		\end{itemize}
		
	\end{frame}
	
	

	\begin{frame}
		
		\frametitle{Estimação dos Parâmetros }
		
		A variável $U_{ji}$ é uma variável dicotômica com distribuição Bernoulli, sendo:
		
		\[U_{ji} =    \begin{cases}
			
			1, & \mbox{resposta correta;}  \\
			
			0, & \mbox{resposta incorreta.}
			
		\end{cases}
		\]
		
		portanto,  
		
		$ \label{eq:bern}
			P(U_{ji} = u_{ji}|\theta_j, \zeta_i) = P(U_{ji} = 1|\theta_j, \zeta_i)^{u_{ji}}
			P(U_{ji} = 0|\theta_j, \zeta_i)^{1 - u_{ji}} = P_{ji}^{u_{ji}}Q_{ji}^{1-u_{ji}}
		$
	
	\end{frame}
	
	\begin{frame}
		
		\frametitle{Estimação dos Parâmetros }
		
		Em geral, temos as respostas do instrumento $U_{ij}$ e desejamos estimar tanto $\boldsymbol{\zeta}$ quanto $\boldsymbol{\theta}$.
		
		Para estimação conjunta, Birnbaum (1968), propôs um processo vai e volta.\newline
		
		\begin{enumerate}
			\item Inicia com uma estimativa grosseira de $\boldsymbol{\theta}$ considerando que $\boldsymbol{\zeta}$ é conhecido.
			\item Estima-se $\boldsymbol{\zeta}$ com $\boldsymbol{\theta}$ conhecido (estimado na primeira etapa).\newline 
		\end{enumerate} 
		
		O Processo é repetido até a convergência dos parâmetros.
		
	\end{frame}
	
	\begin{frame}
		
		\frametitle{Estimação dos Parâmetros }
		
		\begin{center}
			\textbf{Estimador de Máxima Verossimilhança (EMV)}
		\end{center}
		
		Considerando  $ \boldsymbol{\theta} $ conhecido, dados as suposições de independência e unidimensionalidade. A verossimilhança de  $ \boldsymbol{\zeta} $ pode ser escrita como:
		\[
		L(\boldsymbol{\zeta}) =  \prod_{j=1}^{n}\prod_{i=1}^{I}P(U_{ij} = u_{ji}|\theta_j) = \prod_{j=1}^{n}\prod_{i=1}^{I}P_{ji}^{u_{ji}}Q_{ji}^{1-u_{ji}}
		\]
			
   \end{frame}
	
	\begin{frame}
		
		\frametitle{Estimação dos Parâmetros }
		
		Após calcular os valores que maximizam a verossimilhança, ou seja, a solução de: $\dfrac{\partial~log~ L(\boldsymbol{\zeta})}{\partial \boldsymbol{\zeta_i}} = 0$ , são encontrados os EMV para $ \boldsymbol{\zeta}_i = (a_i, b_i , c_i )$:\newline
		
		$
		a_i: D(1 - c_i)\sum_{j=1}^{n}(u_{ji} - P_{ji})(\theta_j - b_i)W_{ij} = 0
		$\newline
		
		$
		b_i: -Da_i(1 - c_i)\sum_{j=1}^{n}(u_{ji} - P_{ji})W_{ij} = 0
		$\newline
		
		$ 
		c_i: \sum_{j=1}^{n}(u_{ji} - P_{ji})\frac{W_{ij}}{P^*_{ij}} = 0
		$  \newline
		
		\begin{center} onde: \space
			$ W_{ji} = \dfrac{P_{ji}^*Q_{ji}^*}{P_{ji}Q_{ji}} ~$ e $ ~ 
			P^*_{ij} = \{1 + e^{-Da_i(\theta_j - b_j)}\}^{-1} $
		\end{center}

			
\end{frame}
	
	\begin{frame}
		
		\frametitle{Estimação dos Parâmetros }
		
		Considerando $\boldsymbol{\zeta}$, o EMV de $\theta_j$, é equivalente a solução da equação
		$\dfrac{\partial~log~ L(\boldsymbol{\theta})}{\partial \theta_j} = 0$ .
		A equação de estimação é dada por:\newline \newline
		
		
		$ \theta_j : D\sum_{i=1}^{I}{a_i(1-c_i)(u_{ji}-P_{ji})W_{ji}} = 0 $ \newline \newline
		
		Os EMV para $ \boldsymbol{\zeta}_i$ e $\theta_j$ não possuem soluções explicitas, por isso é necessário o uso de métodos interativos de integração numérica.
		
	\end{frame}
	
	\begin{frame}
		
		\frametitle{Estimação dos Parâmetros }
		
		\begin{center}
			\textbf{Estimador de Máxima Verossimilhança Marginal (EMVM)}
		\end{center}
		
					
		Para isso, assume-se que os respondentes são uma amostra de uma população cuja a habilidade segue uma determinada função de densidade $g(\theta|\boldsymbol{\eta})$, onde $\boldsymbol{\eta}$ é o vetor de parâmetros da distribuição da habilidade, no caso de uma distribuição normal padrão $\boldsymbol{\eta} = (\mu = 0, \sigma = 1) $. Usando a independência entre indivíduos a probabilidade associada ao vetor de respostas $\boldsymbol{U}$, pode ser escrito como:
		
		\[
		P(\boldsymbol{u}_{..}|\boldsymbol{\zeta}) = \prod_{j=1}^{n}P(\boldsymbol{u}_{j.}|\boldsymbol{\zeta}, \boldsymbol{\eta})
		\]
		
		
		
		
	\end{frame}
	\begin{comment}
		\begin{frame}
			
			\frametitle{Estimação dos Parâmetros }
			
			
			\begin{block}{Estimação Bayesiana}
				
				A abordagem bayesiana é usada para estimar os parâmetros da TRI com base na distribuição \textit{a priori} dos parâmetros. Ela incorpora informações prévias sobre os parâmetros e atualiza essas informações com base nas respostas dos participantes.\\ 
				%O ENEM e utiliza o método EAP (\textit{Expected a Posteriori}) para estimar as habilidades dos participantes (INEP, 2021).
				
			\end{block}
			
		\end{frame}
	\end{comment}
	
\begin{frame}
		
		\frametitle{Avaliação do Modelo}
		
	   \begin{center}
	   	\textbf{Teste Razão de verossimilhança} 
	   	\hfill
	   \end{center}
		
		A razão de verossimilhança é calculada como a diferença entre os logaritmos das verossimilhanças dos dois modelos
		
		\[
		\lambda = -2 (\text{log} L_0 - \text{log} L_1)
		\]
		
		onde $L_0$ representa a Verossimilhança do modelo restrito e
		$L_1$ verossimilhança do modelo completo, Este teste segue uma distribuição assintoticamente qui-quadrado $(\chi^2$).
		
	\end{frame}	

\begin{frame}
	
	\frametitle{Avaliação do Modelo}

	\begin{center}
		\textbf{Teste Razão de verossimilhança} 
		\hfill
	\end{center}
	
	Hipóteses:\\
	\hfill 

	$H_0$: o modelo restrito é suficiente para explicar os dados.\\
	$H_1$: o modelo completo, com mais parâmetros, proporciona um ajuste significativamente melhor.\\
	\hfill
	
	Se o valor de $\lambda$ for grande o suficiente, rejeita-se a hipótese nula em favor do modelo completo. 
	
\end{frame}
	
\begin{frame}
		
		\frametitle{Avaliação do Modelo}
		
		\begin{center}
			\textbf{Estatística M$_2$} 
			\hfill
		\end{center}
		
	A estatística (M$_2$) é parte de uma família de estatísticas de informação limitada, denominada M$_r$, desenvolvida para avaliar modelos TRI.
	A estatística M$_2$ é particularmente útil porque utiliza momentos de ordem 2 em vez da tabela de contingência completa, o que a torna mais adequada para modelos TRI.
	\cite{maydeu2006limited} demonstraram que, especialmente quando $r=2$, a estatística M$_2$ apresenta desempenho superior em comparação com estatísticas de informação completa. 
		
\end{frame}		

\begin{frame}
	
	\frametitle{Avaliação do Modelo}
	
	\begin{center}
		\textbf{Estatística M$_2$} 
		\hfill
	\end{center}
	
		Para avaliar a qualidade do modelo, são consideradas as hipóteses:
		\[
		\begin{cases}
			H_0: \boldsymbol{\pi} = \boldsymbol{\pi}(\boldsymbol{\theta}) \\
			
			H_1: \boldsymbol{\pi} \neq \boldsymbol{\pi}(\boldsymbol{\theta})
		\end{cases}
		\]

	Ou seja, avalia-se se o vetor de probabilidade populacional $\boldsymbol{\pi}$ surge do modelo paramétrico $\boldsymbol{\pi}(\boldsymbol{\theta})$ contra a alternativa de que o modelo está incorreto \cite{maydeu2006limited}.
	
\end{frame}	

\begin{frame}
	
	\frametitle{Avaliação do Modelo}
	
	\begin{center}
		\textbf{RMSEA} 
		\hfill
	\end{center}
	
	O RMSEA (índice de raiz quadrada média do erro de aproximação) é um índice de ajuste absoluto que mede a discrepância média entre o modelo especificado e os dados observados. O valor do RMSEA varia de 0 a 1, sendo que quanto mais próximo a 0, melhor o modelo \cite{kline2016principles}.
	
\end{frame}	

\begin{frame}
	
	\frametitle{Avaliação do Modelo}
	
	\begin{center}
		\textbf{RMSEA} 
		\hfill
	\end{center}
		 \cite{maydeu2014assessing} propôs a estatística de informação limitada RMSEA$_{2}$ para aplicações em modelos TRI, na qual utiliza momentos bivariados e é estimado através do M$_2$. O RMSEA$_2$ pode ser estimado por:
		
		\[
		\hat{\epsilon}_2 = \sqrt{Max\left(\frac{\hat{M_{2}} - df_{2}}
			{N \times df_{2}}, 0 \right) } ,
		\]
	
\end{frame}	

\begin{frame}
	
	\frametitle{Avaliação do Modelo}
	
	\begin{center}
		\textbf{TLI e CFI} 
		\hfill
	\end{center}
	
	Outro método para avaliar o modelo são os índices TLI (\textit{Tucker–Lewis Index}) e CFI (\textit{Comparative Fit Index}).
	
	 O TLI, ou Índice Tucker-Lewis, compara o modelo estimado com um modelo teórico nulo e visa determinar se todos os indicadores estão associados a um único fator latente.
	 Já o CFI, ou Índice de Ajuste Comparativo, é um indicador adicional utilizado para comparar modelos alternativos. 
	 
	 Ambos os índices sugerem um bom ajuste quando seus valores se aproximam de 1 \cite{hair2009multivariada}.
\end{frame}

	%---------------------------------------------------------
	\section{Metodologia}
	
	\begin{frame}
		
		\frametitle{Metodologia}
		
		\begin{center}	Dados:	\end{center}
		
		Os dados utilizados foram de um simulado online de Ciências Humanas aplicado pela empresa de tecnologia educacional estuda.com
		30 itens abertos de provas variadas.\\
		\pause
		Os dados foram dicotomizados em 1 (certo) e 0 (errado).zz
		\pause
		O simulado teve um total de 1055 respondentes, sendo que desses, foram analisados 664	responderam a prova inteira.

	\end{frame}
	
	\begin{frame}
		
		\frametitle{Metodologia}
		\begin{center}	Programas e pacotes: \end{center}
		Software R (R Core Team, 2022) com auxílio dos pacotes: \\
		- \textbf{mirt} (CHALMERS, 2012)\\
		- \textbf{ltm} (RIZOPOULOS, 2006).\\ \pause
		
		\begin{center}	Critério de avaliação:	\end{center}
		
		O teste de razão de verossimilhança foi usado para verificar se a adição de parâmetros melhora o modelo. Para avaliar o ajuste do modelo foi utilizada a estatística M$_2$ proposta por \cite{maydeu2005limited}. 
		Além disso foram avaliados os índices RMSEA, TLI e CFI.

	\end{frame}
	%---------------------------------------------------------
	\section{Resultados}
	
	\begin{frame}
		
	\frametitle{Teoria Clássica dos Testes}
		
	\begin{figure}[H]
	%	\caption{Distribuição do total de acertos do simulado.}
		\includegraphics[width=10cm,height=6cm]{hist_acertos.png}
	\end{figure}
		
	\end{frame}
	
	\begin{frame}
		\frametitle{Índices TCT - Parte 1}
		
		\begin{table}[H]
			\centering
			\scriptsize % reduz o tamanho da fonte
			\caption{Índices TCT (Itens 1 a 15)}
			\begin{tabular*}{\textwidth}{@{\extracolsep{\fill}}cccccc@{}}
				\hline
				\textbf{Item} & \makecell{\textbf{\% Erro}} & \makecell{\textbf{\% Acerto}} & \makecell{\textbf{Discriminação} \\ \textbf{($D_i$)}} & \makecell{\textbf{Ponto} \\ \textbf{Bisserial}} & \makecell{\textbf{Cronbach} \\ \textbf{Excluindo item}} \\ 
				\hline
				1  & 32,2\% & 67,8\% & 0,552 & 0,468 & 0,730 \\ 
				2  & 56,0\% & 44,0\% & 0,394 & 0,319 & 0,741 \\ 
				3  & 49,4\% & 50,6\% & 0,551 & 0,468 & 0,730 \\ 
				4  & 21,5\% & 78,5\% & 0,464 & 0,486 & 0,729 \\ 
				5  & 61,0\% & 39,0\% & 0,507 & 0,415 & 0,734 \\ 
				6  & 63,7\% & 36,3\% & 0,367 & 0,317 & 0,740 \\ 
				7  & 48,6\% & 51,4\% & 0,523 & 0,431 & 0,733 \\ 
				8  & 41,4\% & 58,6\% & 0,402 & 0,358 & 0,738 \\ 
				9  & 73,9\% & 26,1\% & 0,307 & \textbf{0,290} & 0,741 \\ 
				10 & 23,0\% & 77,0\% & 0,390 & 0,402 & 0,734 \\ 
				11 & 11,9\% & 88,1\% & 0,366 & 0,488 & 0,731 \\ 
				12 & 32,8\% & 67,2\% & 0,564 & 0,507 & 0,727 \\ 
				13 & 54,1\% & 45,9\% & 0,541 & 0,449 & 0,731 \\ 
				14 & 61,0\% & 39,0\% & 0,358 & \textbf{0,299} & 0,742 \\ 
				15 & 86,1\% & 13,9\% & \textbf{0,155} & \textbf{0,205} & 0,744 \\ 
				\hline
			\end{tabular*}
		\end{table}
	\end{frame}
	
\begin{frame}
		\frametitle{Índices TCT - Parte 2}
		
		\begin{table}[H]
			\centering
			\scriptsize % reduz o tamanho da fonte
			\caption{Índices TCT (Itens 16 a 30)}
			\begin{tabular*}{\textwidth}{@{\extracolsep{\fill}}cccccc@{}}
				\hline
				\textbf{Item} & \makecell{\textbf{\% Erro}} & \makecell{\textbf{\% Acerto}} & \makecell{\textbf{Discriminação} \\ \textbf{($D_i$)}} & \makecell{\textbf{Ponto} \\ \textbf{Bisserial}} & \makecell{\textbf{Cronbach} \\ \textbf{Excluindo item}} \\ 
				\hline
				16 & 24,4\% & 75,6\% & 0,441 & 0,431 & 0,733 \\ 
				17 & 25,0\% & 75,0\% & 0,399 & 0,399 & 0,735 \\ 
				18 & 22,4\% & 77,6\% & 0,423 & 0,439 & 0,732 \\ 
				19 & 19,4\% & 80,6\% & 0,391 & 0,431 & 0,733 \\ 
				20 & 20,8\% & 79,2\% & \textbf{0,297} & 0,306 & 0,740 \\ 
				21 & 8,1\%  & 91,9\% & \textbf{0,242} & 0,450 & \textbf{0,734}\\ 
				22 & 65,7\% & 34,3\% & \textbf{0,241} & \textbf{0,235} & \textbf{0,745} \\ 
				23 & 49,2\% & 50,8\% & 0,429 & 0,357 & 0,738 \\ 
				24 & 22,1\% & 77,9\% & 0,307 & 0,336 & 0,738 \\ 
				25 & 61,1\% & 38,9\% & 0,399 & 0,350 & 0,738 \\ 
				26 & 51,1\% & 48,9\% & 0,390 & 0,305 & 0,742 \\ 
				27 & 85,8\% & 14,2\% & \textbf{0,000} & \textbf{0,006} &\textbf{0,753} \\ 
				28 & 77,0\% & 23,0\% & \textcolor{red}{\textbf{-0,068}} & \textcolor{red}{\textbf{-0,061}} & \textbf{0,760} \\ 
				29 & 57,5\% & 42,5\% & 0,389 & 0,324 & 0,740 \\ 
				30 & 87,5\% & 12,5\% & \textbf{0,078} &\textbf{0,119} & \textbf{0,747} \\ 
				\hline
				\textbf{Total} &&&&& 0,744 \\
				\hline
			\end{tabular*}
		\end{table}
	\end{frame}
		
	
	
	\begin{frame}
		
		\frametitle{Teoria Clássica dos Testes}
		
			
			Observado um aumento no alfa de Cronbach ao excluir os itens:\\
			21, 22, 27, 28 e 30.
			
			Itens com Ponto bisserial abaixo de 0,3:\\
			28, 27, 30, 15, 22, 9 e 14
			
			Itens com Discriminação Clássica abaixo de 0,3:\\
			28, 27, 30, 15, 22, 21 e 20
%		\begin{table}[ht]
%	\centering
%	\caption{Classificação do item de acordo com a discriminação clássica.}			

%			 \resizebox{\textwidth}{!}{%
%			\begin{tabular}{lc}
%				\hline
%				\textbf{Classificação}  & \textbf{Itens}  \\ 
%				\hline
%				Item bom  & 1, 3, 4, 5, 7, 8, 12, 13, 16, 18 e 23.  \\ 
%				\hline
%				Item bom, mas sujeito a aprimoramento & 
%				2, 6, 9, 10, 11, 14, 17, 19, 23, 24, 25, 26 e 29\\ 
%				\hline
%				Item marginal, sujeito a reelaboração & 20, 21 e 22\\ 
%				\hline
%				Item deficiente, que deve ser rejeitado &  15, 27, 28 e 30\\ 
%				\hline
%			\end{tabular}%
%		}
%		\end{table}

		
	\end{frame}
	
	\begin{frame}
		
		\frametitle{Teoria Clássica dos Testes}
		
	\begin{table}[H]
		\centering
		\caption{Distribuição ideal dos itens por ID.}
		\resizebox{\textwidth}{!}{%
		\begin{tabular}{lcccc}
			\hline
			\textbf{Faixa} & 
			\makecell{\textbf{Total} \\ \textbf{Itens}} & 
			\makecell{\textbf{Distribuição} \\ \textbf{Esperada}} & 
			\makecell{\textbf{Distribuição} \\ \textbf{Obtida}}  & \textbf{Itens} \\ 
			\hline
			\textbf{I} & 3 & 10\% &  10,0\% & 11, 19 e 21\\ 
			\hline
			\textbf{II} & 9 & 20\% &  30,0\% & 1, 4, 10, 12, 16, 17, 18, 20 e 24 \\
			\hline
			\textbf{III} & 8 & 40\% & 26,7\% & 2, 3, 7, 8, 13, 23, 26 e 29 \\ 
			\hline
			\textbf{IV}& 7 & 20\% & 23,3\%  & 5, 6, 9, 14, 22, 25 e 28\\ 
			\hline
			\textbf{V} & 3 & 10\% & 10,0\% & 15, 27 e 30\\ 
			\hline
		\end{tabular}%
	}
	
	
	\end{table}
	\end{frame}
	
	\begin{frame}
	
	\begin{figure}
		\caption{Distribuição da dificuldade clássica dos itens.}
		\includegraphics[width=10cm]{../dificuldade_tct.png}
	\end{figure}

		
	\end{frame}
	
	
	
	
	
	\begin{frame}
		
		\frametitle{Análise TRI - Avaliação do Modelo}
		
		\begin{center}
			\textbf{Avaliação do Modelo}
		\end{center}
		
		\begin{table}
			\centering
			\caption{Teste Razão de verossimilhança.}
			\begin{tabular}{lcccc}
				\hline
				\textbf{Modelo} &  \textbf{ log-verossimilhança }& $\boldsymbol{\chi^2}$ & \textbf{df} & \textbf{p-valor }\\ 
				\hline
				\textbf{1PL} \textbf{(1,b,0)} &  -10991,57 &  &  &  \\ 
				\hline
				\textbf{2PL} \textbf{(a,b,0)} & -10768,56 & 446,01 & 29 & 0,000 \\ 
				\hline
				\textbf{3PL} \textbf{(a,b,c)} & -10741,91 & 53,29 & 30 & 0,006 \\ 
				\hline
			\end{tabular}\\
		\end{table}
		
	O modelo bidimensional também foi testado, porém não obteve convergência dos estimadores.
		
	\end{frame}
	

\begin{frame}
	
	\frametitle{Análise TRI - Avaliação do Modelo}
	
	\begin{center}
		\textbf{Avaliação do Modelo}
	\end{center}
	
	
	\begin{table}
		\centering
		\caption{Teste de adequação do modelo.}
		\begin{tabular}{lcccccc}
			\hline
			\textbf{Modelo} & \textbf{M2}& \textbf{df} & \textbf{p-valor} & \textbf{RMSEA} & \textbf{TLI} & \textbf{CFI} \\ 
			\hline 
			\textbf{1PL} & 1167 & 434 & 0,0000 & 0,0504 & 0,81 & 0,81 \\ 
		\hline	\textbf{2PL} & 485 & 405 & 0,0036 & 0,0173 & 0,98 & 0,98 \\ 
		\hline	\textbf{3PL} & 371 & 375 & 0,5502 & 0,0000 & 1,00 & 1,00 \\ 
			\hline
		\end{tabular}
	\end{table}
\end{frame}


\begin{frame}
	\frametitle{Análise TRI - Modelo 3PL}
	\scriptsize % reduz o tamanho da fonte para caber no slide
	\begin{table}[H]
		\centering
		\caption{Parâmetros do modelo 3PL}
		\begin{tabular*}{0.95\textwidth}{@{\extracolsep{\fill}}cccc|cccc}
			\hline
			\textbf{Item} & \textbf{a$_i$} & \textbf{b$_i$} & \textbf{c$_i$} & \textbf{Item} &  \textbf{a$_i$} & \textbf{b$_i$} & \textbf{c$_i$} \\ 
			\hline
			1 & 1,94 & -0,17 & 0,27 & 16 & 1,19 & -0,91 & 0,18 \\ 
			2 & 3,08 & 1,17 & 0,34 & 17 & 1,00 & -1,31 & 0,00 \\ 
			3 & 1,47 & 0,30 & 0,14 & 18 & 1,20 & -1,29 & 0,01 \\ 
			4 & 1,50 & -1,16 & 0,02 & 19 & 1,36 & -1,38 & 0,00 \\ 
			5 & 1,12 & 0,77 & 0,08 & 20 & 0,68 & -2,14 & 0,01 \\ 
			6 & 0,60 & 1,23 & 0,04 & 21 & 2,46 & -1,74 & 0,00 \\ 
			7 & 0,92 & 0,06 & 0,04 & 22 & 0,32 & 2,20 & 0,01 \\ 
			8 & 0,62 & -0,60 & 0,00 & 23 & 0,68 & -0,04 & 0,00 \\ 
			9 & 0,98 & 2,00 & 0,12 & 24 & 0,73 & -1,91 & 0,00 \\ 
			10 & 1,44 & -0,44 & 0,40 & 25 & 1,83 & 1,15 & 0,23 \\ 
			11 & 2,16 & -1,52 & 0,00 & 26 & 0,51 & 0,10 & 0,00 \\ 
			12 & 1,34 & -0,70 & 0,00 & \textbf{\textcolor{red}{27}} & \textbf{\textcolor{red}{-1,16}} & -3,45 & 0,11 \\ 
			13 & 1,80 & 0,61 & 0,19 & \textbf{\textcolor{red}{28}} & \textbf{\textcolor{red}{-0,46}} & -2,76 & 0,00 \\ 
			14 & 0,48 & 1,01 & 0,01 & 29 & 0,87 & 0,93 & 0,14 \\ 
			15 & 0,54 & 3,63 & 0,01 & 30 & 3,21 & 2,22 & 0,10 \\ 
			\hline
		\end{tabular*}
	\end{table}
\end{frame}


	
\begin{frame}
		
		\frametitle{Análise TRI - Modelo 3PL}
		
		\begin{figure}
			\includegraphics[width=10cm]{../TCCfigura01.png}
		\end{figure}
		
	\end{frame}
	
	\begin{frame}
		
		\frametitle{Análise TRI - Modelo 3PL}
		
		\begin{figure}
			\includegraphics[width=12cm]{../item27.png}
		\end{figure}
		
	\end{frame}
	
	\begin{frame}
		
	\frametitle{Análise TRI - Modelo 3PL}
		
		\begin{figure}
				\includegraphics[width=10cm]{../alternativas2_item27.png}
		\end{figure}
		
	\end{frame}
	

	\begin{frame}
	
	\frametitle{Análise TRI - Modelo 3PL}
	
	\begin{figure}
		\includegraphics[width=10cm]{../alternativas2_item28.png}
	\end{figure}
	
	\end{frame}
	%--------------------------------------------------------------
	\begin{frame}
		
	 \frametitle{Teoria de Resposta ao Item}
		
		m22
		
	\end{frame}	
	
	
% Slide 1
\begin{frame}
	\frametitle{Análise TRI - Modelo 3PL}
	
	\begin{table}[ht]
		\centering
			\caption{Parâmetros do modelo 3PL - 2º Ajuste}
		\scriptsize % reduz o tamanho da fonte para caber melhor
		\begin{tabular}{ccccc}
			\hline
			\textbf{Item} & \textbf{a} & \textbf{b} & \textbf{c} & \textbf{Informação Máxima} \\ 
			\hline 
			22 & 0,32 & 2,12 & 0,01 & 0,03 \\ 
			14 & 0,49 & 0,98 & 0,00 & 0,06 \\ 
			15 & 0,53 & 3,68 & 0,00 & 0,06 \\ 
			26 & 0,51 & 0,10 & 0,00 & 0,07 \\ 
			6  & 0,59 & 1,23 & 0,04 & 0,08 \\ 
			8  & 0,62 & -0,60 & 0,00 & 0,10 \\ 
			23 & 0,67 & -0,04 & 0,00 & 0,11 \\ 
			20 & 0,68 & -2,16 & 0,00 & 0,11 \\ 
			24 & 0,72 & -1,92 & 0,00 & 0,13 \\ 
			29 & 0,90 & 0,93 & 0,14 & 0,15 \\ 
			\hline
		\end{tabular}%
	\end{table}
	
\end{frame}

% Slide 2
\begin{frame}
	\frametitle{Análise TRI - Modelo 3PL}
	
	\begin{table}[ht]
		\centering
			\caption{Parâmetros do modelo 3PL - 2º Ajuste}
		\scriptsize % reduz o tamanho da fonte para caber melhor
		\begin{tabular}{ccccc}
			\hline
			\textbf{Item} & \textbf{a} & \textbf{b} & \textbf{c} & \textbf{Informação Máxima} \\ 
			\hline 
			28 & 0,86 & -1,06 & 0,00 & 0,18 \\ 
			7  & 0,96 & 0,14 & 0,07 & 0,20 \\ 
			9  & 1,02 & 2,01 & 0,13 & 0,20 \\ 
			3  & 1,49 & 0,32 & 0,15 & 0,42 \\ 
			12 & 1,33 & -0,70 & 0,00 & 0,44 \\ 
			19 & 1,37 & -1,37 & 0,00 & 0,47 \\ 
			25 & 1,83 & 1,15 & 0,23 & 0,54 \\ 
			1  & 1,88 & -0,22 & 0,25 & 0,54 \\ 
			13 & 1,77 & 0,59 & 0,18 & 0,55 \\ 
			4  & 1,50 & -1,18 & 0,00 & 0,56 \\ 
			11 & 2,13 & -1,53 & 0,00 & 1,13 \\ 
			2  & 2,96 & 1,18 & 0,34 & 1,14 \\ 
			21 & 2,50 & -1,73 & 0,00 & 1,56 \\ 
			30 & 3,41 & 2,18 & 0,10 & 2,37 \\ 
			\hline
		\end{tabular}%
	\end{table}
	
\end{frame}

	
	\begin{frame}
		
		\frametitle{Análise TRI - Modelo 3PL}
		\begin{figure}
				\caption{Curva de informação dos itens.}
				\includegraphics[width=10cm]{../info_itens.png}
		\end{figure}
	
	\end{frame}	
	
	\begin{frame}
		
		\frametitle{Análise TRI - Modelo 3PL - Informação do teste}
		\begin{figure}
			\caption{Curva de informação e erro padrão do teste.}
	  	    \includegraphics[width=10cm]{../info_modelo2.png}
		\end{figure}
		
	\end{frame}	
	
	\begin{frame}
		
		\frametitle{Análise TRI - Habilidade}
		\begin{figure}
			\caption{Distribuição da habilidade e curva de informação do teste.}
		    \includegraphics[width=10cm]{../habilidade_info.png}
		\end{figure}
		
	\end{frame}	
	
	\begin{frame}
		
	\frametitle{Comparação TCT e TRI}
		\begin{figure}
			\caption{Relação entre o número de acertos e a habilidade estimada pela TRI}
			\includegraphics[width=10cm]{../acertos_habilidade.png}
		\end{figure}
		
	\end{frame}	
	
\begin{frame}
	\frametitle{Comparação TCT e TRI}
	
	\begin{table}[!hbt]
		\centering
		\scriptsize % diminui o tamanho da fonte
		\label{exemplo-10acertos}
		\begin{tabular*}{0.85\textwidth}{@{\extracolsep{\fill}}lccc@{}}
			\hline
			& \textbf{Vetor de Respostas} & $\boldsymbol{\hat{\theta}}$ & \textbf{Acertos} \\ 
			\hline
			1 & 00000000010110010110001011010 & -2,18 & 10 \\ 
			2 & 10100000011010000010001001110 & -1,60 & 10 \\ 
			3 & 01001001100011001001001100000 & -1,57 & 10 \\ 
			4 & 11100000010010100010100001100 & -1,53 & 10 \\ 
			5 & 00101001100010100001100001100 & -1,24 & 10 \\ 
			6 & 10011001110110010000000000100 & -1,19 & 10 \\ 
			7 & 11110100100010000010011000000 & -1,19 & 10 \\ 
			8 & 11110001100000010001000100100 & -1,13 & 10 \\ 
			9 & 10110011000101000000000101100 & -1,12 & 10 \\ 
			10 & 11101011010000010000100000100 & -1,09 & 10 \\ 
			11 & 01111100000000001100100001100 & -1,09 & 10 \\ 
			12 & 10101001100101001100010000000 & -1,08 & 10 \\ 
			13 & 00101001101010001011010000000 & -1,04 & 10 \\ 
			14 & 00111100001110000011000100000 & -1,01 & 10 \\ 
			15 & 10111000101010101000010000000 & -0,92 & 10 \\ 
			16 & 11110101110000100000100000000 & -0,91 & 10 \\ 
			17 & 01111100001110110000000000000 & -0,91 & 10 \\ 
			18 & 11111110110000010000000000000 & -0,86 & 10 \\ 
			19 & 10111110110000001000000010000 & -0,85 & 10 \\ 
			20 & 10111101110000100000000010000 & -0,83 & 10 \\ 
			21 & 10111111010110000000000000000 & -0,75 & 10 \\ 
			\hline
		\end{tabular*}
	\end{table}
\end{frame}


\begin{frame}
	
	\frametitle{Comparação TCT e TRI}
	\begin{figure}
		\includegraphics[width=07cm]{../item30.png}
	\end{figure}
	
\end{frame}	


\begin{frame}
	
	\frametitle{Comparação TCT e TRI}
	\begin{figure}
		\includegraphics[width=10cm]{../tct_item30.png}
	\end{figure}
	
\end{frame}	

\begin{frame}
	
	\frametitle{Comparação TCT e TRI}
	\begin{figure}
		\includegraphics[width=10cm]{../tri_item30.png}
	\end{figure}
	
\end{frame}	

	
	\section{Conclusão}
	
	\section{Referências}
	
	
	%---------------------------------------------------------
	\begin{frame}
	\frametitle{Referências}
	ANDRADE, D. F. de; TAVARES, H. R.; VALLE, R. da C. Teoria da resposta ao item: conceitos e	aplicações. ABE, Sao Paulo, 2000. \newline	
	
		
	PASQUALI, L. Teoria e Método de Medida em Ciências do Comportamento. Brasília,DF: INEP, 1996.\newline
	
	PASQUALI, L. Psicometria: teoria dos testes na psicologia e na educação. Petropolis, RJ: Editora Vozes, 2003.\newline
	
	PASQUALI, L. TRI, Procedimentos e aplicaçães. Curitiba, PR: Appris editora, 2018.


	\end{frame}
	
	
	%--------------------------------------------------
	

	
\end{document}