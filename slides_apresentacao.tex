\documentclass{beamer}
\usepackage[utf8]{inputenc}
\usepackage{comment}
\usetheme{Darmstadt}
\usepackage[brazil]{babel} 
\usecolortheme{default}
\usepackage{multirow}
\usepackage{makecell}
\usepackage[brazilian,hyperpageref]{backref}	 % Paginas com as citações na bibl
\usepackage[alf]{abntex2cite}
%------------------------------------------------------------
%This block of code defines the information to appear in the
%Title page
\title[TRI] %optional
{Aplicação da Teoria de Resposta ao Item em um simulado de Ciências Humanas}

\subtitle{Trabalho de Conclusão do Curso}

\author[Priscila] % (optional)
{Priscila Dalepiane}

\institute[UFMT] % (optional)
{
	
	Bacharelado em Estatística\\
	Universidade Federal de Mato Grosso
	
}

\date[2024] % (optional)
{Novembro 2024}

%\logo{\includegraphics[height=1cm]{overleaf-logo}}

%End of title page configuration block
%------------------------------------------------------------



%------------------------------------------------------------
%The next block of commands puts the table of contents at the 
%beginning of each section and highlights the current section:

\AtBeginSection[]
{
	\begin{frame}
		\frametitle{Sumário}
		\tableofcontents[currentsection]
	\end{frame}
}
%------------------------------------------------------------


\begin{document}
	
	%The next statement creates the title page.
	\frame{\titlepage}

	
\section{Introdução}
	
	\begin{frame}
		
		\frametitle{Introdução}

			\begin{center}
				\textbf{Introdução}
			\end{center}
			
		
			Em situações da psicometria existe uma variável de interesse que não pode ser medida diretamente. Esta variável é denominada \textbf{latente}.\newline \newline
			Pode ser chamada de: traço latente, variável hipotética, fator,  \textbf{habilidade}, aptidão, tendência, \textbf{proficiência}, satisfação, entre outros.
			
			 		
			
			\begin{figure}
			%	\includegraphics[width=5cm,height=5cm]{medida.jpg}
			\end{figure}

		
 \end{frame}

	%---------------------------------------------------------

\begin{frame}
		
		\frametitle{Introdução}

		
		Na educação, precisamos avaliar o conhecimento do alunos, por isso a variável latente pode ser chamada de \textbf{habilidade} ou \textbf{proficiência}. \newline
		A habilidade é uma representação abstrata do conhecimento.	\newline\newline
		Pode ser estimada por meio de avaliações, chamadas também de instrumentos.


		
		\begin{figure}
		%	\includegraphics[width=5cm,height=5cm]{medida.jpg}
		\end{figure}	
		
			
\end{frame}
	%---------------------------------------------------------
	
		%\begin{columns}
		
		%\column{0.6\textwidth}
\begin{frame}
	
	\frametitle{Introdução}

		
		Um \textbf{instrumento} se refere ao teste, prova ou questionário que possui um conjunto de itens que são usados para medir uma habilidade ou traço específico dos respondentes. \newline

		
		 Cada \textbf{item} nesse instrumento deve contribuir para a avaliação geral da proficiência do indivíduo nesse traço ou habilidade. \newline 
		
		A estimação precisa do traço latente depende da qualidade do instrumento como um todo e dos itens que o compõem.

\end{frame}

\begin{frame}
	\frametitle{Introdução}
	2 abordagens para se trabalhar com testes e questionários:
	
	
	\begin{itemize}
		\item Teoria Clássica dos testes (TCT)
		
		\item Teoria de Resposta ao Item (TRI)
	\end{itemize}	
	
	
	
	
\end{frame}

\begin{frame}
	
	\frametitle{Introdução}
	
	\begin{center}
			\textbf{Teoria de Resposta ao Item (TRI)}
	\end{center}
	
	 Atualmente, todas as provas no INEP Instituto Nacional de Estudos e Pesquisas Educacionais utilizam TRI, como:\\
	 
	 
	 - \textbf{SAEB}: Sistema de Avaliação da Educação Básica; \\
	 
	 - \textbf{ENCCEJA}: Exame Nacional para Certificação de Competências de Jovens e Adultos;\\
	 
	 - \textbf{SARESP}: Sistema de Avaliação de Rendimento Escolar do Estado de São Paulo; \\
	 
	 - \textbf{ENEM}: Exame Nacional do Ensino Médio.
		
	
\end{frame}

\begin{frame}
    O estado de Mato Grosso faz avaliação da educação com TRI pelo CAED/UFJF (Centro de Políticas Públicas e Avaliação da Educação), para avaliar e aumentar os índices do IDEB (Índice de Desenvolvimento da Educação Básica) no estado.
\end{frame}
	
	%---------------------------------------------------------
\begin{frame}
		\frametitle{Introdução}
		\begin{center}
		\textbf{Objetivo Geral}	
		\end{center}

		
			O objetivo geral é analisar a qualidade dos itens que compõe um simulado de Ciências Humanas (instrumento) e estimar a habilidade dos respondentes (traço latente).
			

\end{frame}
	%---------------------------------------------------------
\begin{frame}
		\frametitle{Objetivos Específicos}
		\begin{center}
			\textbf{Objetivos Específicos}	
		\end{center}
		
		\begin{itemize}
			\item<1-> Selecionar o melhor modelo TRI para estimação dos parâmetros dos itens e da habilidade.
			\item<2-> Analisar a qualidade dos Itens e da prova utilizando a TCT e a TRI. 
			\item<3-> Determinar se os itens que compõem a prova são adequados para estimar a habilidade.
		\end{itemize}
\end{frame}
	
	%--------------------------------------------------
	
	\section{Referencial Teórico}
	
	% --------------- TCT ------------------------------------------

	\begin{frame}
		
		\frametitle{Teoria Clássica dos Testes}
		
		\begin{center}
			\textbf{Teoria clássica dos testes (TCT)}
		\end{center}
		
		A TCT é a teoria que antecede o TRI, com cálculos mais simples, com a desvantagem da dependência dos respondentes. \newline
		Na TCT a habilidade do respondente é definido pela soma das questões certas, sendo que todos os itens tem o mesmo peso na nota.
	
		
	\end{frame}		
	
	\begin{frame}
		
		\frametitle{Teoria Clássica dos Testes}
		
		\begin{center}
			\textbf{Dificuldade Clássica}
		\end{center}
		
		É calculada pela proporção de itens incorretos.
		
		\[
		ID_i =\dfrac{A_i}{n} 
		\]
		onde $A_i$ é o total de respostas incorretas no item e $n$ é o total de respostas ao item. 
		
	\end{frame}	
	
	\begin{frame}
		
		\frametitle{Teoria Clássica dos Testes}
		\begin{center}
			\textbf{Coeficiente Ponto Bisserial}
		\end{center}
		O coeficiente ponto bisserial representa uma métrica que avalia a relação entre o desempenho em um item e o desempenho geral na prova.
	
		
		\[
		r_{bis} = \frac{\bar{X}_p - \bar{X}_t}{S_t}
		\sqrt{\frac{p_i}{1 - p_i}}
		\]
		
		
		onde:
		
		
		$ \bar{X}_p :$ média dos escores dos examinados que responderam ao item corretamente;
		
		$ \bar{X}_t :$  média global dos escores;
		
		$ S_t :$  desvio padrão do teste;
	\end{frame}
	
	\begin{frame}
		
			\frametitle{Teoria Clássica dos Testes}
		\begin{center}
			\textbf{Índice de Discriminação Clássico ($D_i$)}
		\end{center}
		
		\[D_i =  ACIM_i - ABAI_i\]  \\ 
		
		\hspace{0.5in}
		
		 $ACIM_i:$ proporção de acertos do item no grupo superior, no qual é formado por 27\% dos escores superiores. \\ 
		
		\hspace{0.5in}
		 
		 $ABAI_i:$ proporção de acertos do grupo inferior, no qual é formado pelos 27\% menores escores. 
		
	\end{frame}
	
	
	\begin{frame}
		
		\frametitle{Teoria Clássica dos Testes}
		\begin{center}
			\textbf{Alpha de Cronbach}
		\end{center}
		Para avaliar a consistência interna do teste.
		
		
		\[
		\alpha = \frac{k}{k-1}(1 - \frac{\sum_{i=1}^{k}{s^2_i}}{s_T^2})
		\]
		
		
		$k$ é o numero de itens do teste;
		
		${s_i^2}$ a variância do item;
		
		${s_T^2}$ a variância total do teste.
		
		
	\end{frame}
	
		
	\begin{frame}
		
		\frametitle{Teoria Clássica dos Testes}
		
		\begin{center}
			\textbf{\textit{Alpha} de Cronbach}
		\end{center}
		
		
		
		O coeficiente calcula consistência no intervalo de 0 a 1, sendo quanto mais próximo de 1 maior a consistência, para Pasquali (2003), valores entre \textbf{0,7 e 0,9} são considerados bons, acima de 0,9	 indica itens repetitivos
		
	\end{frame}
	
	
    % --------------- TRI ------------------------------------------
	\begin{frame}
		
		\frametitle{Teoria de Resposta ao Item}
		
		\begin{center}
			\textbf{Teoria de Resposta ao Item (TRI)}
		\end{center}
		
		\begin{block}{Definição}
			A TRI é um \textbf{conjunto de modelos} matemáticos que procuram representar a probabilidade de um indivíduo dar uma certa resposta a um item, como função dos parâmetros dos itens.
			(ANDRADE, 2000)
		\end{block}
		
	\end{frame}
	
	\begin{frame}
		
		\frametitle{Teoria de Resposta ao Item}
		
		A escolha do modelo depende: \newline
		
	
		\begin{itemize}
			\item<1-> Quantas variáveis latentes são necessárias;
			\item<1-> Formato dos Itens (Dicotômicos, Politômicos);

		    
		\end{itemize}
	\end{frame}

	
	\begin{frame}
		
		\frametitle{Modelos Unidimensionais para Itens dicotômicos}
		
		\begin{center}
			\textbf{Modelos Unidimensionais para Itens dicotômico}
		\end{center}
		
		Serão apresentados 3 modelos unidimensionais para casos de itens dicotômicos. 
		
		Modelos unidimensionais assumem que a probabilidade de um indivíduo responder corretamente a um item depende de um único traço latente (habilidade) que se deseja medir. 
		

	\end{frame}
	
	
	% Modelo 1PL
	\begin{frame}
		
		\frametitle{Modelo de 1 Parâmetro Logístico}
		
		\[
				P({U_i}_j = 1|{\theta}_j, b_i) = 
				\frac{1}{1+e^{-D(\theta_j - b_i)}}
		\]
		\begin{figure}
		\includegraphics[width=10cm,height=6cm]{rasch.png}
		\end{figure}

	\end{frame}
	
	% Modelo 2PL
	
	\begin{frame}
		
		\frametitle{Modelo de 2 Parâmetros Logísticos}
		
		\[
			P({U_i}_j = 1|{\theta}_j, a_i, b_i) =
			\frac{1}{1+e^{-Da_i(\theta_j- b_i)}}
		\]
		
		\begin{figure}
			\includegraphics[width=10cm,height=6cm]{../2PL.png}
		\end{figure}

		
	\end{frame}
	
	% Modelo 3PL
	
	\begin{frame}
		
		\frametitle{Modelo de 3 Parâmetros Logísticos}
		\[
			P({U_i}_j = 1|{\theta}_j, a_i, b_i, c_i) =
			c_i+(1-c_i)\frac{1}{1+e^{-Da_i(\theta_j- b_i)}}
		\]
		\begin{figure}
			\includegraphics[width=10cm,height=6cm]{../3PLcci.png}
		\end{figure}

		
	\end{frame}

	% Função de Informação do Item	
	\begin{frame}
		
		\frametitle{Função de Informação do Item}
		
		\begin{center}
			\textbf{Função de Informação do Item}
		\end{center}
		
		$I_i(\theta)$ representa a quantidade de informação que o item $i$ trás sobre a
		habilidade $\theta$.
		
		\[
		I_i(\theta) =  D^2 a_i^2\frac{Q_i(\theta)}{P_i(\theta)} \left[\frac{P_i(\theta) - c_i}{1 - c_i}\right]^2
		\] \\ \pause	
				A Informação é maior quando:
		
		\begin{itemize}
			\item<1-> Maior o valor de $a_i$;
			\item<1-> Menor o valor de $c_i$;
			\item<1-> Quando $\theta_j$  se aproxima de $b_i$.
			
		\end{itemize}
		%$P_i(\theta) = P(U_{ij} = 1| \theta) $ e\\ $ Q_i(\theta) = 1 - P_i(\theta) $

	\end{frame}
	
	
	% função de informação do item - gráfico
	\begin{frame}
		
		\frametitle{Função de Informação do Item}

		\begin{figure}
			\includegraphics[width=10cm,height=8cm]{../info_com_cci.png}
		\end{figure}
		
	\end{frame}
	
	% Função de Informação do Teste	
	\begin{frame}
			
			\frametitle{Função de Informação do Teste}
			
		\begin{center}
				\textbf{Função de Informação do Teste}
		\end{center}
		
		Obtida pela soma das informações fornecidas pelos itens que compõem a prova.	
			\[
			I(\theta) = \sum_{i=1}^{I}I_i(\theta)
			\]
	
		o erro padrão de estimação é dado por:
			
			\[
			EP(\theta) = \dfrac{1}{\sqrt{I(\theta)}}
			\]
			
	\end{frame}	
		
	% Função de Informação do Teste	- gráfico
	\begin{frame}
		
		\frametitle{Função de Informação do Teste}

		\begin{figure}
			\includegraphics[width=10cm,height=8cm]{fft.png}
		\end{figure}
		
	
			
		
	\end{frame}
	
	% Suposições
	\begin{frame}
		
		\frametitle{Suposições do modelo TRI}
	
		\begin{itemize}
			
			\item<1-> Apenas os traços latentes e os parâmetros são necessários para modelar a probabilidade.
						
			\item<1-> Em modelos unidimensionais, apenas uma dimensão do traço latente e necessária, ou seja, o teste mede a mesma variável latente (\textbf{Unidimensionalidade}).
			
			\item<1-> Respostas entre itens e indivíduos são independentes (\textbf{Independência}).
	
		\end{itemize}
		
	\end{frame}
	
	
    % ESTIMAÇÃO DE PARÂMETROS
    
    \begin{frame}
    	
    	\frametitle{Estimação dos parâmetros }
    	
    	\begin{center}
    		\textbf{Denotações:}
    	\end{center}
    	
    	$ \boldsymbol{\theta} = (\theta_1, \cdots, \theta_n) \rightarrow $  vetor de habilidades dos $n$ indivíduos; \newline
    	
    	$ \boldsymbol{\zeta} = (\boldsymbol{\zeta}_1, \cdots, \boldsymbol{\zeta}_I) \rightarrow $ conjunto de parâmetros dos itens.\newline
    	
    	$ \textbf{U}_{n\times I} =  
    	\begin{bmatrix}
    		u_{11} & u_{12} & \cdots & u_{1I} \\
    		u_{21} & u_{22} & \cdots & u_{2I} \\
    		\vdots & \vdots & \ddots & \vdots\\
    		u_{n1} & u_{n2} & \cdots & u_{nI}
    	\end{bmatrix} \rightarrow
    	$ Matriz de respostas
    	
    	
    	
    \end{frame}
    
    
	\begin{frame}
		
		
		\frametitle{Estimação dos Parâmetros}
		
		A variável $U_{ji}$ é uma variável dicotômica com distribuição Bernoulli, sendo:
		
		\[U_{ji} =    \begin{cases}
			
			1, & \mbox{resposta correta;}  \\
			
			0, & \mbox{resposta incorreta.}
			
		\end{cases}
		\]
		
		portanto,  
		
		$ \label{eq:bern}
			P(U_{ji} = u_{ji}|\theta_j, \zeta_i) = P(U_{ji} = 1|\theta_j, \zeta_i)^{u_{ji}}
			P(U_{ji} = 0|\theta_j, \zeta_i)^{1 - u_{ji}} = P_{ji}^{u_{ji}}Q_{ji}^{1-u_{ji}}
		$
	
	\end{frame}
	
	
	\begin{frame}
		
		\frametitle{Estimação dos Parâmetros }
		
		\begin{center}
			\textbf{Estimador de Máxima Verossimilhança (EMV)}
		\end{center}
		
		Considerando  $ \boldsymbol{\theta} $ conhecido, dados as suposições de independência e unidimensionalidade. A verossimilhança de  $ \boldsymbol{\zeta} $ pode ser escrita como:
		\[
		L(\boldsymbol{\zeta}) =  \prod_{j=1}^{n}\prod_{i=1}^{I}P(U_{ij} = u_{ji}|\theta_j) = \prod_{j=1}^{n}\prod_{i=1}^{I}P_{ji}^{u_{ji}}Q_{ji}^{1-u_{ji}}
		\]
			
   \end{frame}
	
		
	
	\begin{frame}
		
		\frametitle{Estimação dos Parâmetros}
		
		Após calcular os valores que maximizam a verossimilhança, ou seja, a solução de: $\dfrac{\partial~log~ L(\boldsymbol{\zeta})}{\partial \boldsymbol{\zeta_i}} = 0$ , são encontrados os EMV para $ \boldsymbol{\zeta}_i = (a_i, b_i , c_i )$:\newline
		
		$
		\hat{a_i}: D(1 - c_i)\sum_{j=1}^{n}(u_{ji} - P_{ji})(\theta_j - b_i)W_{ij} = 0
		$\newline
		
		$
		\hat{b_i}: -Da_i(1 - c_i)\sum_{j=1}^{n}(u_{ji} - P_{ji})W_{ij} = 0
		$\newline
		
		$ 
		\hat{c_i}: \sum_{j=1}^{n}(u_{ji} - P_{ji})\frac{W_{ij}}{P^*_{ij}} = 0
		$  \newline
		
		\begin{center} onde: \space
			$ W_{ji} = \dfrac{P_{ji}^*Q_{ji}^*}{P_{ji}Q_{ji}} ~$ e $ ~ 
			P^*_{ij} = \{1 + e^{-Da_i(\theta_j - b_j)}\}^{-1} $
		\end{center}

			
\end{frame}
	

	\begin{frame}
		
		\frametitle{Estimação dos Parâmetros }
		
		Considerando $\boldsymbol{\zeta}$, o EMV de $\theta_j$, é equivalente a solução da equação
		$\dfrac{\partial~log~ L(\boldsymbol{\theta})}{\partial \theta_j} = 0$ .
		A equação de estimação é dada por:\newline \newline
		
		
		$ \hat{\theta_j} : D\sum_{i=1}^{I}{a_i(1-c_i)(u_{ji}-P_{ji})W_{ji}} = 0 $ \newline \newline
		
		Os EMV para $ \boldsymbol{\zeta}_i$ e $\theta_j$ não possuem soluções explicitas, por isso é necessário o uso de métodos interativos.
		
	\end{frame}
	
	\begin{frame}
		
		\frametitle{Estimação dos Parâmetros }
		
		Em geral, temos as respostas do instrumento $U_{ij}$ e desejamos estimar tanto $\boldsymbol{\zeta}$ quanto $\boldsymbol{\theta}$.
		
		Para estimação conjunta, Birnbaum (1968), propôs um processo vai e volta.\newline
		
		\begin{enumerate}
			\item Inicia com uma estimativa grosseira de $\boldsymbol{\theta}$ considerando que $\boldsymbol{\zeta}$ é conhecido.
			\item Estima-se $\boldsymbol{\zeta}$ com $\boldsymbol{\theta}$ conhecido (estimado na primeira etapa).\newline 
		\end{enumerate} 
		
		O Processo é repetido até a convergência dos parâmetros.
		
	\end{frame}	
	
	
	
	% EMV Marginal
	\begin{frame}
		\frametitle{Estimação dos Parâmetros}
			
		\begin{center}
			\textbf{Outros Métodos de estimação}
		\end{center}
			
		
		\begin{itemize}
			\item Maxima Verossimilhança Marginal: 
			\item Bayesiana
		\end{itemize}
		
		
	\end{frame}

	

%  \begin{frame}
		
%	\frametitle{Estimação dos Parâmetros - EMVM}
%					
%	A equações de estimação usando EMVM para $ \boldsymbol{\zeta}_i$, são:\\
%	
%
%\[
%\begin{aligned}
%	\hat{a}_i & : D(1 - c_i) \sum_{j=1}^{s} r_j
%	\int_{\mathbb{R}} \left[(u_{ji} - P_i)(\theta - b_i)W_i\right] g_j^*(\theta) \, d\theta = 0, \\[1em]	
%	\hat{b}_i & : -D a_i (1 - c_i) \sum_{j=1}^{s} r_j
%	\int_{\mathbb{R}} \left[(u_{ji} - P_i)W_i \right] g_j^*(\theta) \, d\theta = 0, \\[1em]	
%	\hat{c}_i & : \sum_{j=1}^{s} r_j \int_{\mathbb{R}}
%	\left[(u_{ji} - P_i)\frac{W_i}{P^*_i}\right]
%	g_j^*(\theta) \, d\theta = 0
%\end{aligned}
%\]
%		
%\end{frame}
	
	
\begin{comment}
		\begin{frame}
			
			\frametitle{Estimação dos Parâmetros }
			
			
			\begin{block}{Estimação Bayesiana}
				
				A abordagem bayesiana é usada para estimar os parâmetros da TRI com base na distribuição \textit{a priori} dos parâmetros. Ela incorpora informações prévias sobre os parâmetros e atualiza essas informações com base nas respostas dos participantes.\\ 
				%O ENEM e utiliza o método EAP (\textit{Expected a Posteriori}) para estimar as habilidades dos participantes (INEP, 2021).
				
			\end{block}
			
		\end{frame}
	\end{comment}
	
\begin{frame}
		
		\frametitle{Avaliação do Modelo}
		
	   \begin{center}
	   	\textbf{Teste Razão de verossimilhança} 
	   	\hfill
	   \end{center}
		
		A razão de verossimilhança é calculada como a diferença entre os logaritmos das verossimilhanças dos dois modelos
		
		\[
		\lambda = -2 (\text{log} L_0 - \text{log} L_1)
		\]
		
		onde $L_0$ representa a Verossimilhança do modelo restrito e
		$L_1$ verossimilhança do modelo completo, Este teste segue uma distribuição assintoticamente qui-quadrado $(\chi^2$).
		
	\end{frame}	

\begin{frame}
	
	\frametitle{Avaliação do Modelo}

	\begin{center}
		\textbf{Teste Razão de verossimilhança} 
		\hfill
	\end{center}
	
	Hipóteses:\\
	\hfill 

	$H_0$: o modelo restrito é suficiente para explicar os dados.\\
	$H_1$: o modelo completo, com mais parâmetros, proporciona um ajuste significativamente melhor.\\
	\hfill
	
	Se o valor de $\lambda$ for grande o suficiente, rejeita-se a hipótese nula em favor do modelo completo. 
	
\end{frame}
	
\begin{frame}
		
		\frametitle{Avaliação do Modelo}
		
		\begin{center}
			\textbf{Estatística M$_2$} 
		\end{center}
		
	A estatística (M$_2$) é parte de uma família de estatísticas de informação limitada, denominada M$_r$, desenvolvida para avaliar modelos TRI. \newline 
	A estatística M$_2$ é particularmente útil porque utiliza momentos de ordem 2 em vez da tabela de contingência completa, o que a torna mais adequada para modelos TRI. \newline 
	Maydeu e Joe (2006)  demonstraram que, especialmente quando $r=2$, a estatística M$_2$ apresenta desempenho superior em comparação com estatísticas de informação completa. 
		
\end{frame}		

\begin{frame}
	
	\frametitle{Avaliação do Modelo}
	
	\begin{center}
		\textbf{Estatística M$_2$} 
	\end{center}
	
		Para avaliar a qualidade do modelo, são consideradas as hipóteses:
		\[
		\begin{cases}
			H_0: \boldsymbol{\pi} = \boldsymbol{\pi}(\boldsymbol{\theta}) \\
			
			H_1: \boldsymbol{\pi} \neq \boldsymbol{\pi}(\boldsymbol{\theta})
		\end{cases}
		\]

	Ou seja, avalia-se se o vetor de probabilidade populacional $\boldsymbol{\pi}$ surge do modelo paramétrico $\boldsymbol{\pi}(\boldsymbol{\theta})$ contra a alternativa de que o modelo está incorreto.
	
\end{frame}	

\begin{frame}
	
	\frametitle{Avaliação do Modelo}

	\begin{itemize}
		\item \textbf{RMSEA:} (índice de raiz quadrada média do erro de aproximação) é um índice de ajuste absoluto que mede a discrepância média entre o modelo especificado e os dados observados. Varia de 0 a 1, sendo que quanto mais próximo a 0, melhor o modelo.
		
		\item \textbf{CFI:} (\textit{Tucker–Lewis Index})  compara o modelo estimado com um modelo teórico nulo e visa determinar se todos os indicadores estão associados a um único fator latente. 
		
		\item \textbf{CFI:} (\textit{Comparative Fit Index}), ou Índice de Ajuste Comparativo, é um indicador adicional utilizado para comparar modelos alternativos.
		
		
	\end{itemize}
	
\end{frame}	




	%---------------------------------------------------------
	\section{Metodologia}
	
	\begin{frame}
		
		\frametitle{Metodologia}
		
		\begin{center}	Dados:	\end{center}
		
		\begin{itemize}
			\item Os dados utilizados foram de um simulado online de Ciências Humanas aplicado pela empresa de tecnologia educacional estuda.com. \pause
			\item 30 itens de domínio público provenientes de instituições de ensino públicas e privadas. \pause
			\item Os dados foram dicotomizados em 1 (certo) e 0 (errado). \pause
			\item O simulado teve um total de 1055 respondentes, sendo que desses, foram analisados 664	responderam a prova inteira. \pause
		\end{itemize}
		
		

	\end{frame}
	
	\begin{frame}
		
		\frametitle{Metodologia}
		\begin{center}	Programas e pacotes: \end{center}
		Software R (R Core Team, 2022) com auxílio dos pacotes: \\ \pause
		- \textbf{ltm} (RIZOPOULOS, 2006) para análise TCT.\\ \pause 
		- \textbf{mirt} (CHALMERS, 2012) para análise TRI.\\
		

	\end{frame}
	
	\begin{frame}
		
		\frametitle{Metodologia}
		\begin{center}	Estimação: \end{center}
		
		- Parâmetros: Máxima verossimilhança marginal Algoritmo EM delineado por Bock e Aitkin (1981).\\ \newline
		
		- As habilidades: o método EAP (\textit{Expected A posteriori}). 
		
	\end{frame}
	
	\begin{frame}
		
		
		\begin{center}	Critério de avaliação:	\end{center}
		
		Para análise exploratória: \\
		
		O teste de razão de verossimilhança. \newline
		
		Para análise confirmatória:\\
		
		Teste M$_2$, RMSEA, TLI e CFI. \newline
		
	\end{frame}
	
	
	%---------------------------------------------------------
	\section{Resultados}
	
	\begin{frame}
		
	\frametitle{Teoria Clássica dos Testes}
		
	\begin{figure}[H]
	%	\caption{Distribuição do total de acertos do simulado.}
		\includegraphics[width=10cm,height=6cm]{hist_acertos.png}
	\end{figure}
		
	\end{frame}
	
	\begin{frame}
		\frametitle{Índices TCT - Parte 1}
		
		\begin{table}[H]
			\centering
			\scriptsize % reduz o tamanho da fonte
			\caption{Índices TCT (Itens 1 a 15)}
			\begin{tabular*}{\textwidth}{@{\extracolsep{\fill}}cccccc@{}}
				\hline
				\textbf{Item} & \makecell{\textbf{\% Erro}} & \makecell{\textbf{\% Acerto}} & \makecell{\textbf{Discriminação} \\ \textbf{($D_i$)}} & \makecell{\textbf{Ponto} \\ \textbf{Bisserial}} & \makecell{\textbf{Cronbach} \\ \textbf{Excluindo item}} \\ 
				\hline
				1  & 32,2\% & 67,8\% & 0,552 & 0,468 & 0,730 \\ 
				2  & 56,0\% & 44,0\% & 0,394 & 0,319 & 0,741 \\ 
				3  & 49,4\% & 50,6\% & 0,551 & 0,468 & 0,730 \\ 
				4  & 21,5\% & 78,5\% & 0,464 & 0,486 & 0,729 \\ 
				5  & 61,0\% & 39,0\% & 0,507 & 0,415 & 0,734 \\ 
				6  & 63,7\% & 36,3\% & 0,367 & 0,317 & 0,740 \\ 
				7  & 48,6\% & 51,4\% & 0,523 & 0,431 & 0,733 \\ 
				8  & 41,4\% & 58,6\% & 0,402 & 0,358 & 0,738 \\ 
				9  & 73,9\% & 26,1\% & 0,307 & \textbf{0,290} & 0,741 \\ 
				10 & 23,0\% & 77,0\% & 0,390 & 0,402 & 0,734 \\ 
				11 & 11,9\% & 88,1\% & 0,366 & 0,488 & 0,731 \\ 
				12 & 32,8\% & 67,2\% & 0,564 & 0,507 & 0,727 \\ 
				13 & 54,1\% & 45,9\% & 0,541 & 0,449 & 0,731 \\ 
				14 & 61,0\% & 39,0\% & 0,358 & \textbf{0,299} & 0,742 \\ 
				15 & 86,1\% & 13,9\% & \textbf{0,155} & \textbf{0,205} & 0,744 \\ 
				\hline
			\end{tabular*}
		\end{table}
	\end{frame}
	
\begin{frame}
		\frametitle{Índices TCT - Parte 2}
		
		\begin{table}[H]
			\centering
			\scriptsize % reduz o tamanho da fonte
			\caption{Índices TCT (Itens 16 a 30)}
			\begin{tabular*}{\textwidth}{@{\extracolsep{\fill}}cccccc@{}}
				\hline
				\textbf{Item} & \makecell{\textbf{\% Erro}} & \makecell{\textbf{\% Acerto}} & \makecell{\textbf{Discriminação} \\ \textbf{($D_i$)}} & \makecell{\textbf{Ponto} \\ \textbf{Bisserial}} & \makecell{\textbf{Cronbach} \\ \textbf{Excluindo item}} \\ 
				\hline
				16 & 24,4\% & 75,6\% & 0,441 & 0,431 & 0,733 \\ 
				17 & 25,0\% & 75,0\% & 0,399 & 0,399 & 0,735 \\ 
				18 & 22,4\% & 77,6\% & 0,423 & 0,439 & 0,732 \\ 
				19 & 19,4\% & 80,6\% & 0,391 & 0,431 & 0,733 \\ 
				20 & 20,8\% & 79,2\% & \textbf{0,297} & 0,306 & 0,740 \\ 
				21 & 8,1\%  & 91,9\% & \textbf{0,242} & 0,450 & \textbf{0,734}\\ 
				22 & 65,7\% & 34,3\% & \textbf{0,241} & \textbf{0,235} & \textbf{0,745} \\ 
				23 & 49,2\% & 50,8\% & 0,429 & 0,357 & 0,738 \\ 
				24 & 22,1\% & 77,9\% & 0,307 & 0,336 & 0,738 \\ 
				25 & 61,1\% & 38,9\% & 0,399 & 0,350 & 0,738 \\ 
				26 & 51,1\% & 48,9\% & 0,390 & 0,305 & 0,742 \\ 
				27 & 85,8\% & 14,2\% & \textbf{0,000} & \textbf{0,006} &\textbf{0,753} \\ 
				28 & 77,0\% & 23,0\% & \textcolor{red}{\textbf{-0,068}} & \textcolor{red}{\textbf{-0,061}} & \textbf{0,760} \\ 
				29 & 57,5\% & 42,5\% & 0,389 & 0,324 & 0,740 \\ 
				30 & 87,5\% & 12,5\% & \textbf{0,078} &\textbf{0,119} & \textbf{0,747} \\ 
				\hline
				\textbf{Total} &&&&& 0,744 \\
				\hline
			\end{tabular*}
		\end{table}
	\end{frame}
		
	
	

			

%		\begin{table}[ht]
%	\centering
%	\caption{Classificação do item de acordo com a discriminação clássica.}			

%			 \resizebox{\textwidth}{!}{%
%			\begin{tabular}{lc}
%				\hline
%				\textbf{Classificação}  & \textbf{Itens}  \\ 
%				\hline
%				Item bom  & 1, 3, 4, 5, 7, 8, 12, 13, 16, 18 e 23.  \\ 
%				\hline
%				Item bom, mas sujeito a aprimoramento & 
%				2, 6, 9, 10, 11, 14, 17, 19, 23, 24, 25, 26 e 29\\ 
%				\hline
%				Item marginal, sujeito a reelaboração & 20, 21 e 22\\ 
%				\hline
%				Item deficiente, que deve ser rejeitado &  15, 27, 28 e 30\\ 
%				\hline
%			\end{tabular}%
%		}
%		\end{table}

		

	
	\begin{frame}
		
		\frametitle{Teoria Clássica dos Testes}
		
	\begin{table}[H]
		\centering
		\caption{Distribuição ideal dos itens por ID.}
		\resizebox{\textwidth}{!}{%
		\begin{tabular}{lcccc}
			\hline
			\textbf{Faixa} & 
			\makecell{\textbf{ID} \\ \textbf{Itens}} & 
			\makecell{\textbf{Distribuição} \\ \textbf{Esperada}} & 
			\makecell{\textbf{Distribuição} \\ \textbf{Obtida}}  & \textbf{Itens} \\ 
			\hline
			\textbf{I} & 0 a 0,2 & 10\% &  10,0\% & 11, 19 e 21\\ 
			\hline
			\textbf{II} & 0,2 a 0,4 & 20\% &  30,0\% & 1, 4, 10, 12, 16, 17, 18, 20 e 24 \\
			\hline
			\textbf{III} & 0,4 a 0,6 & 40\% & 26,7\% & 2, 3, 7, 8, 13, 23, 26 e 29 \\ 
			\hline
			\textbf{IV}& 0,6 a 0,8 & 20\% & 23,3\%  & 5, 6, 9, 14, 22, 25 e 28\\ 
			\hline
			\textbf{V} & 0,8 a 1 & 10\% & 10,0\% & 15, 27 e 30\\ 
			\hline
		\end{tabular}%
	}
	
	
	\end{table}
	\end{frame}
	
	\begin{frame}
	
	\begin{figure}
		\caption{Distribuição da dificuldade clássica dos itens.}
		\includegraphics[width=10cm]{../dificuldade_tct.png}
	\end{figure}

		
	\end{frame}
	

	
	\begin{frame}
		
		\frametitle{Análise TRI - Avaliação do Modelo}
		
		\begin{center}
			\textbf{Avaliação do Modelo}
		\end{center}
		
		\begin{table}
			\centering
			\caption{Teste Razão de verossimilhança.}
			\begin{tabular}{lcccc}
				\hline
				\textbf{Modelo} &  \textbf{ log-verossimilhança }& $\boldsymbol{\chi^2}$ & \textbf{df} & \textbf{p-valor }\\ 
				\hline
				\textbf{1PL} \textbf{(1,b,0)} &  -10991,57 &  &  &  \\ 
				\hline
				\textbf{2PL} \textbf{(a,b,0)} & -10768,56 & 446,01 & 29 & 0,000 \\ 
				\hline
				\textbf{3PL} \textbf{(a,b,c)} & -10741,91 & 53,29 & 30 & 0,006 \\ 
				\hline
			\end{tabular}\\
		\end{table}
		
	O modelo bidimensional também foi testado, porém não obteve convergência dos estimadores.
		
	\end{frame}
	

\begin{frame}
	
	\frametitle{Análise TRI - Avaliação do Modelo}
	
	\begin{center}
		\textbf{Avaliação do Modelo}
	\end{center}
	
	
	\begin{table}
		\centering
		\caption{Teste de adequação do modelo.}
		\begin{tabular}{lcccccc}
			\hline
			\textbf{Modelo} & \textbf{M2}& \textbf{df} & \textbf{p-valor} & \textbf{RMSEA} & \textbf{TLI} & \textbf{CFI} \\ 
			\hline 
			\textbf{1PL} & 1167 & 434 & 0,0000 & 0,0504 & 0,81 & 0,81 \\ 
		\hline	\textbf{2PL} & 485 & 405 & 0,0036 & 0,0173 & 0,98 & 0,98 \\ 
		\hline	\textbf{3PL} & 371 & 375 & 0,5502 & 0,0000 & 1,00 & 1,00 \\ 
			\hline
		\end{tabular}
	\end{table}
\end{frame}


\begin{frame}
	\frametitle{Análise TRI - Modelo 3PL}
	\scriptsize % reduz o tamanho da fonte para caber no slide
	\begin{table}[H]
		\centering
		\caption{Parâmetros do modelo 3PL}
		\begin{tabular*}{0.95\textwidth}{@{\extracolsep{\fill}}cccc|cccc}
			\hline
			\textbf{Item} & \textbf{a$_i$} & \textbf{b$_i$} & \textbf{c$_i$} & \textbf{Item} &  \textbf{a$_i$} & \textbf{b$_i$} & \textbf{c$_i$} \\ 
			\hline
			1 & 1,94 & -0,17 & 0,27 & 16 & 1,19 & -0,91 & 0,18 \\ 
			2 & 3,08 & 1,17 & 0,34 & 17 & 1,00 & -1,31 & 0,00 \\ 
			3 & 1,47 & 0,30 & 0,14 & 18 & 1,20 & -1,29 & 0,01 \\ 
			4 & 1,50 & -1,16 & 0,02 & 19 & 1,36 & -1,38 & 0,00 \\ 
			5 & 1,12 & 0,77 & 0,08 & 20 & 0,68 & -2,14 & 0,01 \\ 
			6 & 0,60 & 1,23 & 0,04 & 21 & 2,46 & -1,74 & 0,00 \\ 
			7 & 0,92 & 0,06 & 0,04 & 22 & 0,32 & 2,20 & 0,01 \\ 
			8 & 0,62 & -0,60 & 0,00 & 23 & 0,68 & -0,04 & 0,00 \\ 
			9 & 0,98 & 2,00 & 0,12 & 24 & 0,73 & -1,91 & 0,00 \\ 
			10 & 1,44 & -0,44 & 0,40 & 25 & 1,83 & 1,15 & 0,23 \\ 
			11 & 2,16 & -1,52 & 0,00 & 26 & 0,51 & 0,10 & 0,00 \\ 
			12 & 1,34 & -0,70 & 0,00 & \textbf{\textcolor{red}{27}} & \textbf{\textcolor{red}{-1,16}} & -3,45 & 0,11 \\ 
			13 & 1,80 & 0,61 & 0,19 & \textbf{\textcolor{red}{28}} & \textbf{\textcolor{red}{-0,46}} & -2,76 & 0,00 \\ 
			14 & 0,48 & 1,01 & 0,01 & 29 & 0,87 & 0,93 & 0,14 \\ 
			15 & 0,54 & 3,63 & 0,01 & 30 & 3,21 & 2,22 & 0,10 \\ 
			\hline
		\end{tabular*}
	\end{table}
\end{frame}


	
\begin{frame}
		
		\frametitle{Análise TRI - Modelo 3PL}
		
		\begin{figure}
			\includegraphics[width=10cm]{../TCCfigura01.png}
		\end{figure}
		
	\end{frame}
	
	
	\begin{frame}
		
	\frametitle{Análise TRI - Modelo 3PL}
		
		\begin{figure}
				\includegraphics[width=10cm]{../alternativas2_item27.png}
		\end{figure}
		
	\end{frame}
	
	\begin{frame}
		
		\frametitle{Análise TRI - Modelo 3PL}
		
		\begin{figure}
			\includegraphics[width=12cm]{../item27.png}
		\end{figure}
		
	\end{frame}
	

	\begin{frame}
	
		\frametitle{Análise TRI - Modelo 3PL}
		
		\begin{figure}
			\includegraphics[width=10cm]{../alternativas2_item28.png}
		\end{figure}
		
	\end{frame}
	
	%--------------------------------------------------------------
%	\begin{frame}
%		
%	 \frametitle{Teoria de Resposta ao Item}
%		
%		\begin{table}
%			\centering
%			\caption{Teste de adequação do 2 Ajuste.}
%			\begin{tabular}{lcccccc}
%				\hline
%				\textbf{Modelo} &
%				 \textbf{M}$_\textbf{2}$ &
%				  \textbf{df} & 
%				  \textbf{p-valor} & 
%				  \textbf{RMSEA}  & 
%				  \textbf{TLI} & 
%				  \textbf{CFI} \\ 
%				\hline 
%				\textbf{3PL - 2 Ajuste} & 344 & 348 & 0,542 & 0,0000  & 1,00 & 1,00 \\ 
%				\hline
%			\end{tabular}
%		\end{table}
%		
%	\end{frame}	
	
	
% Slide 1
\begin{frame}
	\frametitle{Análise TRI - Modelo 3PL}
	
	\begin{table}[ht]
		\centering
			\caption{Parâmetros do modelo 3PL - 2º Ajuste}
		\scriptsize % reduz o tamanho da fonte para caber melhor
		\begin{tabular}{ccccc}
			\hline
			\textbf{Item} & \textbf{a} & \textbf{b} & \textbf{c} & \textbf{Informação Máxima} \\ 
			\hline 
			22 & 0,32 & 2,12 & 0,01 & 0,03 \\ 
			14 & 0,49 & 0,98 & 0,00 & 0,06 \\ 
			15 & 0,53 & 3,68 & 0,00 & 0,06 \\ 
			26 & 0,51 & 0,10 & 0,00 & 0,07 \\ 
			6  & 0,59 & 1,23 & 0,04 & 0,08 \\ 
			8  & 0,62 & -0,60 & 0,00 & 0,10 \\ 
			23 & 0,67 & -0,04 & 0,00 & 0,11 \\ 
			20 & 0,68 & -2,16 & 0,00 & 0,11 \\ 
			24 & 0,72 & -1,92 & 0,00 & 0,13 \\ 
			29 & 0,90 & 0,93 & 0,14 & 0,15 \\ 
			28 & 0,86 & -1,06 & 0,00 & 0,18 \\ 
			7  & 0,96 & 0,14 & 0,07 & 0,20 \\ 
			\hline
		\end{tabular}%
	\end{table}
	
\end{frame}

% Slide 2
\begin{frame}
	\frametitle{Análise TRI - Modelo 3PL}
	
	\begin{table}[ht]
		\centering
			\caption{Parâmetros do modelo 3PL - 2º Ajuste}
		\scriptsize % reduz o tamanho da fonte para caber melhor
		\begin{tabular}{ccccc}
			\hline
			\textbf{Item} & \textbf{a} & \textbf{b} & \textbf{c} & \textbf{Informação Máxima} \\ 
			\hline 
			9  & 1,02 & 2,01 & 0,13 & 0,20 \\ 
			3  & 1,49 & 0,32 & 0,15 & 0,42 \\ 
			12 & 1,33 & -0,70 & 0,00 & 0,44 \\ 
			19 & 1,37 & -1,37 & 0,00 & 0,47 \\ 
			25 & 1,83 & 1,15 & 0,23 & 0,54 \\ 
			1  & 1,88 & -0,22 & 0,25 & 0,54 \\ 
			13 & 1,77 & 0,59 & 0,18 & 0,55 \\ 
			4  & 1,50 & -1,18 & 0,00 & 0,56 \\ 
			11 & 2,13 & -1,53 & 0,00 & 1,13 \\ 
			2  & 2,96 & 1,18 & 0,34 & 1,14 \\ 
			21 & 2,50 & -1,73 & 0,00 & 1,56 \\ 
			30 & 3,41 & 2,18 & 0,10 & 2,37 \\ 
			\hline
		\end{tabular}%
	\end{table}
	
\end{frame}

	
	\begin{frame}
		
		\frametitle{Análise TRI - Modelo 3PL}
		\begin{figure}
				\caption{Curva de informação dos itens.}
				\includegraphics[width=10cm]{../info_itens.png}
		\end{figure}
	
	\end{frame}	
	
	\begin{frame}
		
		\frametitle{Análise TRI - Modelo 3PL - Informação do teste}
		\begin{figure}
			\caption{Curva de informação e erro padrão do teste.}
	  	    \includegraphics[width=10cm]{../info_modelo2.png}
		\end{figure}
		
	\end{frame}	
	
	\begin{frame}
		
		\frametitle{Análise TRI - Habilidade}
		\begin{figure}
			\caption{Distribuição da habilidade e curva de informação do teste.}
		    \includegraphics[width=10cm]{../habilidade_info.png}
		\end{figure}
		
	\end{frame}	
	
	\begin{frame}
		
	\frametitle{Comparação TCT e TRI}
		\begin{figure}
			\caption{Relação entre o número de acertos e a habilidade estimada pela TRI}
			\includegraphics[width=10cm]{../acertos_habilidade.png}
		\end{figure}
		
	\end{frame}	
	
\begin{frame}
	\frametitle{Comparação TCT e TRI}
	
	\begin{table}[!hbt]
		\centering
		\scriptsize % diminui o tamanho da fonte
		\label{exemplo-10acertos}
		\begin{tabular*}{0.85\textwidth}{@{\extracolsep{\fill}}lccc@{}}
			\hline
			& \textbf{Vetor de Respostas} & $\boldsymbol{\hat{\theta}}$ & \textbf{Acertos} \\ 
			\hline
			1 & 00000000010110010110001011010 & -2,18 & 10 \\ 
			2 & 10100000011010000010001001110 & -1,60 & 10 \\ 
			3 & 01001001100011001001001100000 & -1,57 & 10 \\ 
			4 & 11100000010010100010100001100 & -1,53 & 10 \\ 
			5 & 00101001100010100001100001100 & -1,24 & 10 \\ 
			6 & 10011001110110010000000000100 & -1,19 & 10 \\ 
			7 & 11110100100010000010011000000 & -1,19 & 10 \\ 
			8 & 11110001100000010001000100100 & -1,13 & 10 \\ 
			9 & 10110011000101000000000101100 & -1,12 & 10 \\ 
			10 & 11101011010000010000100000100 & -1,09 & 10 \\ 
			11 & 01111100000000001100100001100 & -1,09 & 10 \\ 
			12 & 10101001100101001100010000000 & -1,08 & 10 \\ 
			13 & 00101001101010001011010000000 & -1,04 & 10 \\ 
			14 & 00111100001110000011000100000 & -1,01 & 10 \\ 
			15 & 10111000101010101000010000000 & -0,92 & 10 \\ 
			16 & 11110101110000100000100000000 & -0,91 & 10 \\ 
			17 & 01111100001110110000000000000 & -0,91 & 10 \\ 
			18 & 11111110110000010000000000000 & -0,86 & 10 \\ 
			19 & 10111110110000001000000010000 & -0,85 & 10 \\ 
			20 & 10111101110000100000000010000 & -0,83 & 10 \\ 
			21 & 10111111010110000000000000000 & -0,75 & 10 \\ 
			\hline
		\end{tabular*}
	\end{table}
\end{frame}





\begin{frame}
	
	\frametitle{Comparação TCT e TRI}
	\begin{figure}
		\includegraphics[width=10cm]{../tct_item30.png}
	\end{figure}
	
\end{frame}	

\begin{frame}
	
	\frametitle{Comparação TCT e TRI}
	\begin{figure}
		\includegraphics[width=10cm]{../tri_item30.png}
	\end{figure}
	
\end{frame}	

\begin{frame}
	
	\frametitle{Comparação TCT e TRI}
	\begin{figure}
		\includegraphics[width=07cm]{../item30.png}
	\end{figure}
	
\end{frame}	

	
\section{Conclusão}
	
	\begin{frame}
		\frametitle{Conclusão}
		
		\begin{itemize}
		\item A prova mostrou-se deficiente tanto pela TCT quanto pela TRI.
		\item O modelo de 3 parâmetros (3PL) mostrou-se o melhor ajuste, tanto pelo teste de razão de verossimilhança quanto pelos índices M2, TLI, CFI e RMSEA.
		\end{itemize}
		
		
	\end{frame}
	
		\begin{frame}
		\frametitle{Conclusão}
		
		\begin{itemize}
		
			\item Itens com boa discriminação (ex.: itens 3, 5, 7, 12) recomendados para o banco de itens. Itens com discriminação baixa (ex.: itens 6, 8, 14...) não são recomendados sem revisão.
			\item Itens problemáticos:\\
			Item 28: gabarito incorreto, necessita correção.\\
			Item 27: formulação confusa, com alternativas que levam a interpretações erradas.\\
			\item  Melhor discriminação para habilidades abaixo da média (-1,9 a -0,7).
			Lacuna de informação para habilidades medianas, recomendando-se a inclusão de itens nesse intervalo.\\		
			
		\end{itemize}
		
		
	\end{frame}
	
	
	\section{Referências}
	
	
	
	%---------------------------------------------------------
	\begin{frame}
	\frametitle{Referências}
	
	ANDRADE, D. F. de; TAVARES, H. R.; VALLE, R. da C. Teoria da resposta ao item: conceitos e	aplicações. ABE, Sao Paulo, 2000. \newline	
	
		
	PASQUALI, L. Teoria e Método de Medida em Ciências do Comportamento. Brasília,DF: INEP, 1996.\newline
	
	PASQUALI, L. Psicometria: teoria dos testes na psicologia e na educação. Petropolis, RJ: Editora Vozes, 2003.\newline
	
	PASQUALI, L. TRI, Procedimentos e aplicaçães. Curitiba, PR: Appris editora, 2018.


	\end{frame}
	
	
	%--------------------------------------------------
	

	
\end{document}