\chapter{CONCLUSÃO}


	

A análise dos dados do simulado utilizando tanto a Teoria Clássica dos Testes (TCT) quanto a Teoria de Resposta ao Item (TRI) permitiu uma compreensão mais aprofundada sobre o desempenho dos participantes e a qualidade dos itens. Na TCT, os resultados indicaram uma distribuição de acertos concentrada em torno da mediana, com uma variação moderada entre os participantes. No entanto, a TCT, ao basear-se apenas no escore bruto, tem limitações para captar nuances importantes da prova, como a dificuldade e discriminação de cada item, bem como a coerência nas respostas dos indivíduos. Nesse sentido, a TRI apresentou-se como uma ferramenta mais robusta, permitindo não apenas a estimativa das habilidades dos participantes em uma escala contínua, mas também oferecendo uma análise detalhada de cada item em termos de dificuldade, discriminação e probabilidade de acerto ao acaso.

Os modelos testados na TRI, especialmente o modelo de três parâmetros logísticos (3PL), mostraram-se adequados para capturar as relações entre o desempenho dos participantes e as características dos itens. O modelo 3PL, que inclui o parâmetro de chute, apresentou o melhor ajuste aos dados, indicando ser o mais apropriado para este simulado, conforme demonstrado pelos testes de adequação (M2, TLI, CFI e RMSEA). Esse modelo permitiu estimar com maior precisão a habilidade dos participantes, diferenciando não apenas o total de acertos, mas a coerência no padrão de respostas, algo que a TCT não contempla.

Os itens com bons índices de discriminação e dificuldade adequada, como os itens 3, 5, 7 e 12, mostraram-se eficazes na diferenciação de participantes com diferentes níveis de habilidade e, portanto, devem ser preservados no banco de itens para futuras aplicações. Esses itens fornecem informações valiosas e contribuem para a precisão das estimativas de habilidade. Por outro lado, os itens 21, 27, 28 e 30, que apresentaram discriminação negativa ou baixa, além de coeficientes bisserial problemáticos, precisam ser revisados ou descartados, pois comprometem a validade e a confiabilidade da avaliação.

A análise da curva de informação do teste indicou que a prova foi mais eficaz para discriminar participantes com habilidades abaixo da média, especialmente na faixa entre -2 e -1. No entanto, identificou-se uma lacuna no teste em relação aos participantes com habilidades próximas à média, onde a quantidade de informação foi menor do que o ideal. Para melhorar a precisão da avaliação nesse grupo, seria recomendável a inclusão de itens que ofereçam mais informação nessa faixa central de habilidades. A adição desses itens poderá aumentar a capacidade do teste de discriminar com maior precisão os participantes com habilidades intermediárias, melhorando a qualidade da mensuração global.

Em síntese, a análise realizada demonstra a importância da TRI como ferramenta complementar à TCT, fornecendo uma estimativa mais acurada das habilidades dos participantes e permitindo uma análise detalhada dos itens. A partir dos resultados, conclui-se que a revisão de itens com baixa discriminação e a adequação da curva de informação podem contribuir para aprimorar a qualidade do teste e aumentar sua capacidade de mensurar habilidades de forma precisa e justa em toda a gama de participantes.
