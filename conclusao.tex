\chapter{CONCLUSÃO}


A análise dos dados do simulado utilizando tanto a TCT permitiu uma visão geral sobre a prova, a TRI permitiu uma compreensão mais aprofundada sobre o desempenho dos participantes e a qualidade dos itens. Na TCT, os resultados indicaram uma distribuição de acertos concentrada em torno da mediana. No entanto, a TCT, ao basear-se apenas no escore bruto, tem limitações para captar nuances importantes da prova, como a dificuldade e discriminação de cada item, bem como a coerência nas respostas dos indivíduos. Nesse sentido, a TRI apresentou-se como uma ferramenta mais robusta, permitindo não apenas a estimativa das habilidades dos participantes em uma escala contínua, mas também oferecendo uma análise detalhada de cada item em termos de dificuldade, discriminação e probabilidade de acerto ao acaso.

O modelo 3PL apresentou o melhor ajuste aos dados, indicando ser o mais apropriado para este simulado, conforme demonstrado pelos testes de adequação (M$_2$, TLI, CFI e RMSEA$_2$). Esse modelo permitiu estimar com maior precisão a habilidade dos participantes, diferenciando não apenas o total de acertos, mas a coerência no padrão de respostas.

Os itens com bons índices de discriminação e dificuldade adequada, como os itens 3, 5, 7 e 12, mostraram-se eficazes na diferenciação de participantes com diferentes níveis de habilidade e, portanto, devem ser preservados no banco de itens para futuras aplicações. Esses itens fornecem boas informações e contribuem para a precisão das estimativas de habilidade. Por outro lado, os itens 6, 8, 14, 15, 20, 22, 23, 24 e 26 que apresentaram discriminação muito baixa e pouca informação para o teste, além de coeficientes bisserial problemáticos, precisam ser revisados e não é recomendado para incluir em futuros simulados, pois possuem pouco informam sobre a variável latente de interesse, no caso, a habilidade do respondente.

Foram identificados dois itens problemáticos, o item 27 e o item 28, ambos apresentando discriminação negativa. No caso do item 28, verificou-se que o gabarito estava incorreto, sendo necessária a correção do item. Já o item 27 foi mal formulado, com alternativas  que induzem interpretações equivocadas, dificultando a identificação da resposta correta.

A análise da curva de informação do teste indicou que a prova foi mais eficaz para discriminar participantes com habilidades abaixo da média, especialmente na faixa entre -1,9 e -0,7. No entanto, identificou-se uma lacuna no teste em relação aos participantes com habilidades próximas à média, onde a quantidade de informação foi menor. Para melhorar a precisão da avaliação nesse grupo, seria recomendável a inclusão de itens que ofereçam mais informação nessa faixa central de habilidades. A adição desses itens poderá aumentar a capacidade do teste de discriminar com maior precisão os participantes com habilidades intermediárias, melhorando a qualidade da mensuração.

Em síntese, a análise realizada demonstra a importância da TRI como ferramenta complementar à TCT, fornecendo uma estimativa mais acurada das habilidades dos participantes e permitindo uma análise detalhada dos itens. A partir dos resultados, conclui-se que a revisão de itens com baixa discriminação e a adequação da curva de informação podem contribuir para aprimorar a qualidade do teste e aumentar sua capacidade de mensurar habilidades de forma precisa e justa em toda a gama de participantes.
