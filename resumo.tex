\begin{resumo}
	
Este trabalho avalia a qualidade de itens e da prova de um simulado de Ciências Humanas por meio da Teoria Clássica dos Testes (TCT) e da Teoria de Resposta ao Item (TRI). A prova foi composta por 30 itens provenientes de instituições públicas e privadas, respondida por 664 pessoas. Foram analisadas apenas respostas de indivíduos que responderam todos os itens. Na análise pela TCT, foi obtido um coeficiente alfa de Cronbach igual de 0,744, indicando boa consistência interna do teste, a análise também apontou 7 itens com correlação ponto bisserial abaixo de 0,30.  Na TRI, foram ajustados modelos logísticos de 1, 2 e 3 parâmetros (1PL, 2PL e 3PL) e o modelo unidimensional de 3 parâmetros apresentou melhor ajuste. A comparação entre os modelos utilizando o teste de razão de verossimilhança. A qualidade do ajuste foi avaliada com o índice M$_2$, que demonstrou um bom ajuste,  enquanto os índices RMSEA, TLI e CFI também indicaram boa adequação do modelo. Na análise dos parâmetros TRI dois itens apresentaram discriminação negativa, indicando problemas. Na investigação, um item havia sido corrigido errado no gabarito e o outro estava mal elaborado, levando o leitor desatento a marcar uma alternativa incorreta. A contribuição de vários itens para a informação da habilidade medida foi praticamente nula, pois para muitos a informação máxima foi próxima de zero. A prova apresentou maior informação para valores baixos de habilidade, sendo que, a habilidade estimada dos examinados foram em sua maioria, medianas e altas,ou seja, a prova foi fácil para esses examinados.

\textbf{Palavras-chaves}: 
\end{resumo}

\begin{resumo}[Abstract]
 \begin{otherlanguage*}{english}
   This is the english abstract.

   \vspace{\onelineskip}

   \noindent
   \textbf{Key-words}: Text, editoration.
 \end{otherlanguage*}
\end{resumo}