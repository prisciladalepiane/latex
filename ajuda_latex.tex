\section{Equações e Fórmulas}
Use o ambiente \texttt{equation} para escrever Equações e Fórmulas numeradas:

\begin{equation}
	\forall x \in X, \quad \exists \: y \leq \epsilon
\end{equation}

Escreva expressões matemáticas entre \$ e \$, como em $ \lim_{x \to \infty}
\exp(-x) = 0 $, para que fiquem na mesma linha.

Também é possível usar colchetes para indicar o início de uma expressão matemática que não precisa estar numerada.

\[
\left|\sum_{i=1}^n a_ib_i\right|
\le
\left(\sum_{i=1}^n a_i^2\right)^{1/2}
\left(\sum_{i=1}^n b_i^2\right)^{1/2}
\]




\section{Figuras}
As Figuras podem ser criadas diretamente em \LaTeX, como o exemplo da \autoref{fig_circulo}.

\begin{figure}[!htb]
	\caption{A delimitação do espaço}\label{fig_circulo}
	\begin{center}
		\setlength{\unitlength}{5cm}
		\begin{picture}(1,1)
			\put(0,0){\line(0,1){1}}
			\put(0,0){\line(1,0){1}}
			\put(0,0){\line(1,1){1}}
			\put(0,0){\line(1,2){.5}}
			\put(0,0){\line(1,3){.3333}}
			\put(0,0){\line(1,4){.25}}
			\put(0,0){\line(1,5){.2}}
			\put(0,0){\line(1,6){.1667}}
			\put(0,0){\line(2,1){1}}
			\put(0,0){\line(2,3){.6667}}
			\put(0,0){\line(2,5){.4}}
			\put(0,0){\line(3,1){1}}
			\put(0,0){\line(3,2){1}}
			\put(0,0){\line(3,4){.75}}
			\put(0,0){\line(3,5){.6}}
			\put(0,0){\line(4,1){1}}
			\put(0,0){\line(4,3){1}}
			\put(0,0){\line(4,5){.8}}
			\put(0,0){\line(5,1){1}}
			\put(0,0){\line(5,2){1}}
			\put(0,0){\line(5,3){1}}
			\put(0,0){\line(5,4){1}}
			\put(0,0){\line(5,6){.8333}}
			\put(0,0){\line(6,1){1}}
			\put(0,0){\line(6,5){1}}
		\end{picture}
	\end{center}
	\legend{Fonte: os autores}
\end{figure}

Ou então figuras podem ser incorporadas de arquivos externos, como é o caso da
\autoref{fig_grafico}.

\begin{comment}
	\begin{figure}[!htb]
		\caption{Exemplo}\label{fig_grafico}
		\centering
		\includegraphics[width=5cm]{grafico.pdf}
		\legend{Fonte: \citeonline[p. 24]{araujo2012}}
	\end{figure}
\end{comment}



\section{Tabelas}
As tabelas devem seguir o padrão do \citeonline{ibge1993}, na  \autoref{tabela-ibge} é apresentado um exemplo .

\begin{table}[!htb]
	\IBGEtab{%
		\caption{Um Exemplo de tabela alinhada que pode ser longa   ou curta, conforme parão do IBGE.}
		\label{tabela-ibge}
	}{%
		\begin{tabular}{ccc}
			\toprule
			Nome & Nascimento & Documento \\
			\midrule \midrule
			Maria da Silva & 11/11/1111 & 111.111.111-11 \\
			\midrule
			João Souza & 11/11/2111 & 211.111.111-11 \\
			\midrule
			Laura Vicua & 05/04/1891 & 3111.111.111-11 \\
			\bottomrule
		\end{tabular}
	}{%
		\fonte{Produzido pelos autores.}
		\nota{Esta é uma nota, que diz que os dados são baseados na  regressão linear.}
		\nota[Anotações]{Uma anotação adicional, que pode ser seguida de várias  outras.}
	}
\end{table}



\section{Citações diretas}

Utilize o ambiente \texttt{citação} para incluir citações diretas com mais de três linhas:

\begin{citacao}
	As citaçõs diretas, no texto, com mais de três linhas, devem ser
	destacadas com recuo de 4 cm da margem esquerda, com letra menor que a do texto
	utilizado e sem as aspas. No caso de documentos datilografados, deve-se
	observar apenas o recuo \cite[5.3]{NBR10520:2002}.
\end{citacao}


O ambiente \texttt{citação} pode receber como parâmetro opcional um nome de idioma previamente carregado nas opções da classe. Nesse caso, o texto da citação é automaticamente escrito em itálico e a hifenização ajustada para o idioma selecionado na opção do ambiente. Por exemplo:
\begin{citacao}[english]
	Text in English language in italic with correct hyphenation.
\end{citacao}


Citação simples, com até três linhas, devem ser
incluídas com aspas. Observe que em \LaTeX  as aspas iniciais são diferentes das finais: ``Amor é fogo que arde sem se ver''.


\section{Notas de rodapé}


As notas de rodapé são detalhadas pela NBR 14724:2011 na seção 5.2.1
\footnote{As notas devem ser digitadas ou datilografadas dentro das margens, ficando
	separadas do texto por um espaço simples de entre as linhas e por filete de 5
	cm, a partir da margem esquerda. Devem ser alinhadas, a partir da segunda linha
	da mesma nota, abaixo da primeira letra da primeira palavra, de forma a destacar
	o expoente, sem espaço entre elas e com fonte menor
	
	\citeonline[5.2.1]{NBR14724:2011}.}\footnote{Caso uma séries de notas sejam
	criadas sequencialmente, o \abnTeX  instrui o \LaTeX\ para que uma vígula seja
	colocada após cada número do expoente que indica a nota de rodapé no corpo do
	texto.}\footnote{Verifique se os números do expoente possuem uma vírgula para
	dividi-los no corpo do texto.}.